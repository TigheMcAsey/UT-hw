\documentclass[10.5pt]{article}
\usepackage{amsmath, amsfonts, amssymb,amsthm}
\usepackage[includeheadfoot]{geometry} % For page dimensions
\usepackage{fancyhdr}
\usepackage{enumerate} % For custom lists
\usepackage{tikz-cd}
\usepackage{graphicx}

\fancyhf{}
\lhead{MAT1100 hw1}
\rhead{Tighe McAsey - 1008309420}
\pagestyle{fancy}

% Page dimensions
\geometry{a4paper, margin=1in}

\theoremstyle{definition}
\newtheorem{pb}{}

% Commands:

\newcommand{\set}[1]{\{#1\}}
\newcommand{\abs}[1]{\lvert#1\rvert}
\newcommand{\norm}[1]{\lvert\lvert#1\rvert\rvert}
\newcommand{\tand}{\text{ and }}
\newcommand{\tor}{\text{ or }}

\begin{document}
    \begin{pb}
        \textbf{(a)} We know that since for any element \(x \in G\) that \(C_x\) the centralizer of \(x\) is a subgroup of \(G\), by orbit stabilizer \(\#C_x\#\mathcal{O}_x = \#G\) where \(\mathcal{O}_x\) is the orbit of \(x\) under the conjugation action. It follows that listing the distinct orbits \(\mathcal{O}_{x_i}\),
        \begin{align*}
            \#G = \#\bigsqcup_i \mathcal{O}_{x_i} = \sum_i \#\mathcal{O}_{x_i}
        \end{align*}
        and each \(\mathcal{O}_{x_i} \vert G\) implying that \(\#\mathcal{O}_{x_i} \in \set{1,p,p^2}\), if we assume for contradiction that \(Z(G) = 1\), then from the above class equation \(1 + \sum_{i \geq 2} \#\mathcal{O}_{x_i} = 1 + \sum_{i \geq 2}k_i p = p^2\), taking this equation modulke \(p\) we get a contradiction, so that \(\#Z(G) = p\) or \(p^2\) in the \(p^2\) case we are done, and in the other case we have \(G/Z(G)\) is cyclic, so that any element of \(G\) can be written in the form \(x^ia\), \(a \in Z(G)\), but \((x^ia)(x^jb) = x^{i+j}ab = x^jx^iba = x^jbx^ia\) which shows that \(G\) is abelian, this contradicts \(\#Z(G) = p\), so \(\#Z(G) = \#G\) and \(G\) is abelian. \qed

        \textbf{(b)} A group of order \(p\) is cyclic and generated by any of its nontrivial elements, so that all of its elements aside from \(1\) have order \(p\). Hence \(p-1\) such elements. A group of order \(p^2\) is of the form \(C_p^2\) or \(C_{p^2}\) by the classification of abelian groups. In the former case, we use the fact that \(o(x,y) = \text{lcm}[x,y]\), so as long as either \(x\) or \(y\) have order \(p\) we have an element of order \(p\), this gives \(p^2 - 1\) elements of order \(p\). In the latter case, \(C_{p^2}\) is generated by any element \(k\) with \(\text{gcd}(k,p^2) = 1\) the number of these is \(\varphi(p^2) = p(p-1) = p^2 - p\), so there are \(p^2 - (p^2 - p) - 1 = p-1\) elements of order \(p\). \qed
    \end{pb}
    \begin{pb}
        We can use orbit stabilizer with \(S_9\) acting on the pearls, the stabilizer of the necklace BBBBWWWRR is clearly \(S_4 \times S_3 \times S_2\) has cardinality \(12\cdot4!\), so there are \(\frac{9!}{12\cdot4!}\) necklaces. \qed
    \end{pb}
    \begin{pb}
        \begin{align*}
            X = \set{(g_1,\hdots,g_p) \mid g_i \in G \tand \prod_1^p g_i = 1}
        \end{align*}
        Then we have an action of \(\mathbf{F}_p\) on \(X\) via \(k\cdot(g_i) = (g_{[k + i]})\) where \([n]\) denotes \(n\) mod \(p\). Note that when determining an element of \(X\), the first \(p-1\) choices are free, meaning there are \(n^{p-1}\) choices for the first \(p-1\) coordinates (here \(n = \#G\)), but the last coordinate is fixed as \(\left(\prod_1^{p-1}g_i\right)^{-1}\), so \(X\) has \(n^{p-1}\) elements. In the case where \(g_i = g_j, \forall i,j\) the action is trivial, otherwise the orbit of the action has order \(p\). The cardinality of \(X\) is the sum of the cardinality of the orbits, letting \(r\) be the number of single element orbits, and \(q\) the number of \(p\) element orbits we get \(n^{p-1} = r + qp\) so that since \(p \vert n\) we get \(p \vert (r + qp)\) which implies \(p \vert r\), but \(r \neq 0\) since \((1,\hdots,1)\) has a one element orbit, hence there is another one element orbit \((g,\hdots,g)\), but this means that \(g^p = 1\), so that \(o(g) \vert p\), but \(g \neq 1\) means that \(o(g) \neq 1\) so \(o(g) = p\) as desired.
    \end{pb}
    \begin{pb}
        Since the groups are not commutative they must have composite order, write \(\#G = \prod_1^r p_i\) where \(p_r \geq p_{r-1} \geq \cdots \geq p_1\). Then \(p_1\) connot be \(11\), so \(p_1\) is at most \(7\), moreover if \(p_1 = 7\), then \(G = C_{49}, C_{7} \tor C_7^2\) all of which are abelian, so that \(p_1\) is at most \(5\), if \(p_1 = 5\), then once again we get an abelian group since the only possible factorizations are \(p_1 = p_2 = 5\) which is abelian by question 1, or \(p_2 = 7\), it follows that the subgroup \(N\) of order \(7\) is normal since it has index \(5\), the smallest prime dividing the order of the group, so this group can't be simple. This implies that \(p_1 \in \set{2,3}\).

        Now we note that no group of order \(pq\) for \(p,q\) both primes is simple, this is immediate from Sylows theorem since if \(q > p\), the number of \(q\) sylow subgroups must be one hence normal. Now we can look at the case \(p^2q\), if \(p > q\), we are done since the Sylow-\(p\) group has to be normal, so assume \(q > p\), then there are either \(p^2\) or \(1\) sylow \(q\) subgroups, in the latter case we are done and in the former case, these sylow \(q\) subgroups all need to have intersection \(1\) since they are cyclic so we have \(p^2(q-1) = p^2q - p^2\) elements of order \(q\), the remaining elements must all be in the same sylow \(p\) subgroup having order \(p^2\), so the sylow \(p\) subgroup must be normal in this case, contradicting simplicity.

        To rule out all groups with 3 prime factors we are thus left with the groups of order \(30\) and \(42\), the group of order \(42\) is easy since the sylow-\(7\) subgroup must be normal by Sylow 2. For the group of order 30, we need only consider the case where there are \(6\) sylow-5 subgroups and hence \(24\) elements of order 5, the sylow \(3\) subgroup must be normal otherwise there would be \(10\) sylow 3 subgroups adding \(20\) elements of order \(3\) giving too many elements, so at this point we have ruled out all groups with \(3\) prime factors.

        Four prime factors (not all \(2,3\)) gives us groups of order \(56\) and \(40\) in the \(56\) case we get the sylow-7 subgroup has index \(1\) or \(8\), in the index \(8\) case we get \(48\) elements of order \(7\), so the remaining \(8\) elements must constitute a single sylow-2 subgroup, which must be normal so this case is null. In the order \(40\) case the sylow \(5\) subgroup must be normal.

        Now the problem has been reduced to \(\geq 4\) prime factors all being \(2,3\), for now I will appeal to Burnside's theorem, but I should finish it more satisfyingly later.
    \end{pb}
    \begin{pb}
        \textbf{(a)}
            The types of elements in \(A_5\) are \(1,(a\,b\,c), (a\,b)(c\,d),(a\,b\,c\,d\,e)\) The conjugacy classes are each contained in their conjugacy classes in \(S_5\), i.e. cycle types, and hence the orders are divisors. We compute \(\#\mathcal{O}_1 = 1\), the normalizer of elements of the form \((a\,b)(c\,d)\) are elements of \(A_5\) sending \(\set{a,b} \to \set{a,b}\) and \(\set{c,d} \to \set{c,d}\) or \(\set{a,b} \to \set{c,d}\) and \(\set{c,d} \to \set{a,b}\), of which there are \(4\) elements

        \textbf{(b)} A normal subgroup of \(A_5\) must be a union of conjugacy classes (including 1) with order dividing \(\#A_5 = 60\), by reading the conjugacy class sizes in (a), no such union of conjugacy classes exists.

        \textbf{(c)}

        \textbf{(d)}

        \textbf{(e)}
    \end{pb}
    \begin{pb}
        \textbf{(a)} Suppose \(gx = y\), then for \(h \in \text{Stab}_x\) we have \(ghg^{-1}y = ghx = gx = y\), so that \(ghg^{-1} \in \text{Stab}_y\), by rewriting the equation \(g^{-1}y = x\), we see this conjugation is onto since it has inverse \(h \mapsto g^{-1}hg\) by the same argument. \qed
        
        \textbf{(b)} First suppose the stabilizers are conjugate, then 
    \end{pb}
\end{document}