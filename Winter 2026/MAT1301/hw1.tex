\documentclass[10.5pt]{article}
\usepackage{amsmath, amsfonts, amssymb,amsthm}
\usepackage[includeheadfoot]{geometry} % For page dimensions
\usepackage{fancyhdr}
\usepackage{enumerate} % For custom lists
\usepackage{tikz-cd}
\usepackage{graphicx}

\fancyhf{}
\lhead{MAT1301 hw1}
\rhead{Tighe McAsey - 1008309420}
\pagestyle{fancy}

% Page dimensions
\geometry{a4paper, margin=1in}

\theoremstyle{definition}
\newtheorem{pb}{}
\usepackage{tikz-cd, stackengine}

% Commands:

\newcommand{\set}[1]{\{#1\}}
\newcommand{\gen}[1]{\langle#1\rangle}
\newcommand{\abs}[1]{\left\vert#1\right\vert}
\newcommand{\norm}[1]{\lvert\lvert#1\rvert\rvert}
\newcommand{\tand}{\text{ and }}
\newcommand{\tor}{\text{ or }}
\newcommand{\pd}{\frac{\partial}{\partial x_j}}
\newcommand{\px}{\frac{\partial}{\partial x}}
\newcommand{\py}{\frac{\partial}{\partial y}}
\newcommand{\pz}{\frac{\partial}{\partial z}}
\newcommand{\ppx}{\frac{\partial^2}{\partial x^2}}
\newcommand{\ppy}{\frac{\partial^2}{\partial y^2}}
\newcommand{\ppz}{\frac{\partial^2}{\partial z^2}}
\newcommand{\hess}{\operatorname{Hess}}

\begin{document}
\begin{pb}
    \begin{align*}
        h(t,s) = \begin{cases}
            \gamma(2t) & t \leq \frac{1 - s}{2} \\
            \gamma\left(\frac{1 - s}{2}\right) & t \in \left(\frac{1 - s}{2},\frac{1 + s}{2}\right) \\
            \overline{\gamma}(2t) & t \geq \frac{1 + s}{2}
        \end{cases}
    \end{align*} \qed
\end{pb}
\begin{pb}
    \textbf{((iii) \(\mathbf{\implies}\) (i)):} Let \(x_0 \in X\), then \([1_{x_0}] \in \pi_1(X,x_0)\), so for any \([\gamma] \in \pi_1(X,x_0)\) we have \([\gamma] = [1_{x_0}]\) by assumption, whence \(\pi_1(X,x_0) = \set{[1_{x_0}]} = 0\). \qed

    \textbf{((i) \(\mathbf{\implies}\) (iii)):} Let \(f_1, f_2: S^1 \to X\), where we parameterize \(S_1\) as \(\frac{I}{0\sim 1}\), since \(X\) is path connected, there is some path \(\gamma: I \to X\) with \(\gamma(0) = f_1(0) \tand \gamma(1) = f_2(0)\), then by assumption (i), \([f_1] = [1_{f_1(0)}]\) and \([f_2] = [1_{f_2(0)}]\), now we can write the homotopy between \(1_{f_1(0)} \sim f_1\) and \(1_{f_2(0)} \sim f_2\), which is given by \(h(s,t) = \gamma(s)\). \qed

    \textbf{((ii) \(\mathbf{\implies}\) (i)):} Let \([\gamma] \in \pi_1(X,x_0)\) for some \(x_0 \in X\), then we can define new paths
    \begin{align*}
        &\gamma_1: t \mapsto \gamma(t/2) &\gamma_2: t \mapsto \gamma\left(\frac{1 + t}{2}\right)
    \end{align*}
    then \(\overline{\gamma_1}\) and \(\gamma_2\) satisfy the hypotheses of (ii), which entails \(\gamma_2 \sim \overline{\gamma_1}\), now since \(\gamma = \gamma_1 \cdot \gamma_2\) we find that \(\gamma \sim \gamma_1\cdot \overline{\gamma_1} \sim 1_{x_0}\), whence \([\gamma] = [1_{x_0}]\). Since \([\gamma]\) was arbitrary we conclude \(\pi_1(X,x_0) = 0\). \qed

    \textbf{(() \(\mathbf{\implies}\) (ii)):}
\end{pb}
\begin{pb}
    Define the map
    \begin{align*}
        \psi: \pi_1(X,x_0) &\to \set{S^1 \to X}/\sim \\
            [f] &\mapsto [f]
    \end{align*}
    In words, we forget the base point of \(f\). This is of course well defined up to homotopy. To see that this map is onto, it suffices to check that every map \(f: S^1 \to X\) is homotopic to a map \(f'\) with \(f'(0) = x_0\). Now checking this consider some \(f: S^1 \to X\), and let \(\gamma\) be a path between \(x_0\) and \(f(0)\), then we can write
    \begin{align*}
        f'(t) = \begin{cases}
            \gamma(3t) & t \leq \frac13 \\
            f(3(t - \frac13)) & t \in (\frac13,\frac23) \\
            \gamma(1 - 3(t - \frac23)) & t \geq \frac23
        \end{cases}
    \end{align*}
    Then we can verify explicitly that \(f' \sim f\) by taking first a homotopy of \(f'\)
    \begin{align*}
        h'(t,s) = \begin{cases}
            f'()
        \end{cases}
    \end{align*}
\end{pb}
\end{document}