\documentclass[10.5pt]{article}
\usepackage{amsmath, amsfonts, amssymb,amsthm}
\usepackage[includeheadfoot]{geometry} % For page dimensions
\usepackage{fancyhdr}
\usepackage{enumerate} % For custom lists
\usepackage{tikz-cd}
\usepackage{graphicx}

\fancyhf{}
\lhead{Notes On Geometric Topology}
\rhead{Tighe McAsey}
\pagestyle{fancy}

% Page dimensions
\geometry{a4paper, margin=1in}

\theoremstyle{definition}
\newtheorem{pb}{Exercise}[subsection]
\newtheorem{thm}{Theorem}[section]
\newtheorem{definition}{Definition}[section]
\newtheorem{proposition}{Proposition}[section]
\newtheorem{example}{Example(s)}[section]
\usepackage{tikz-cd, stackengine}

% Commands:

\newcommand{\set}[1]{\{#1\}}
\newcommand{\gen}[1]{\langle#1\rangle}
\newcommand{\abs}[1]{\left\vert#1\right\vert}
\newcommand{\norm}[1]{\lvert\lvert#1\rvert\rvert}
\newcommand{\tand}{\text{ and }}
\newcommand{\tor}{\text{ or }}
\newcommand{\pd}{\frac{\partial}{\partial x_j}}
\newcommand{\px}{\frac{\partial}{\partial x}}
\newcommand{\py}{\frac{\partial}{\partial y}}
\newcommand{\pz}{\frac{\partial}{\partial z}}
\newcommand{\ppx}{\frac{\partial^2}{\partial x^2}}
\newcommand{\ppy}{\frac{\partial^2}{\partial y^2}}
\newcommand{\ppz}{\frac{\partial^2}{\partial z^2}}
\newcommand{\hess}{\operatorname{Hess}}

\begin{document}
    Here are some notes meant for reference for my studies in geometric topology. I will only supply proofs when I think doing so will be edifying, these notes are mostly based from Bruno Martelli's Book as well as Do Carmo's book so proofs can be found there.

    As convention when writing \(C^\alpha(M)\) for a manifold \(M\), \(\alpha \geq 1\), but the \(\alpha = \infty\) case is most interesting. All manifolds are atleast differentiable.
    \section{Riemannian Geometry}
    \subsection{Vector Fields}
    \begin{definition}
        A vectorfield \(X: M \to TM\) can be thought of as a derivation, i.e. if \(X(p) = \sum_1^n a_j(p)\pd\) then \(X: C^\alpha(M) \to C^{\alpha - 1}(M)\) via
        \begin{align*}
            X(f)(p) = \sum_1^n a_j(p)\pd f
        \end{align*}
    \end{definition}
    \begin{proposition}
        If \(\varphi: M \to M\) is a diffeomorphism, and \(v \in T_pM\) with \(\gamma(0) = p \tand \gamma'(0) = v\), then
        \begin{align*}
            (d_p \varphi(v))(f) = v(f\circ \gamma)(p)
        \end{align*}
    \end{proposition}
    \begin{definition}
        The Lie bracket of two vector fields \(X,Y\) denoted \([X,Y]\) is the unique vectorfield \(Z\) satisfying \(Z(f) = (XY-YX)(f)\) for all \(f \in C^\alpha(M)\)
    \end{definition}
    \begin{proposition}
        If \(X,Y\) are vectorfields on \(M\), with \(U \supset \set{p} \subset M\), and \(\varphi_t\) is a flow of \(X\) in \(U\), then
        \begin{align*}
            [X,Y](p) = \lim_{t\to0}\frac{1}{t}(Y - d \varphi_t Y)(\varphi_t(p))
        \end{align*}
    \end{proposition}
    
    \subsection{Riemannian Metrics}
    \begin{definition}
        A Riemannian metric is a positive definite, symmetric bilinear form \(g: TM \otimes TM \to \mathbb{R}\)
    \end{definition}
    \begin{definition}
        A Riemannian Manifold is a pair \((M,g)\) where \(M\) is a manifold and \(g\) is a Riemannian metric.
    \end{definition}
    \begin{proposition}
        A Riemannian metric can be represented in coordinates, given a local frame \((X_j)\), and coframe \((\sigma^j)\) we have
        \begin{align*}
            g(-,-) = \sum_{i \leq j} g(X_i,X_j)\sigma^i \sigma^j
        \end{align*}
    \end{proposition}
    \begin{definition}
        \(f: M \to N\) is a isometry between Riemannian manifolds if \(f\) is a diffeomorphism and \(df^*g_M = g_N\)
    \end{definition}
    \begin{definition}
        \(M\) and \(N\) are said to be locally isometric if for any \(p \in M\) there is some neighborhood \(p \in U\), and some \(V \subset N\) with \(f: U \to V\) an isometry. Note this is not an equivalence relation.
    \end{definition}
    \begin{example}
    \begin{itemize}
            \item Euclidean, \(g_{\text{EUC}}\) has \(g_{ij} = \delta_{ij}\)
            \item \(f: M \to N\), then we can define \(g_M(u,v) = g_N(df u, df v)\)
            \item If a group \(G\) acts on \(M\) smoothly, freely, properly and isometrically then we can define for \(\pi: M \to M/G\)
            \begin{align*}
                g_{M/G}(u,v)_p = g_M(d_p\pi^{-1}(u),d_p\pi^{-1}(v))_{\pi^{-1}(p)}
            \end{align*}
            The case of \(U(1)\) acting on \(\mathbb{C}^{n+1} \setminus \set{0}\) is the Fubini-Study metric on \(\mathbb{P}^n\)
            \item The product metric \(g_{X\times Y}(u,v)_{p,q} = g_X(d\pi_X u, d\pi_X v) + g_Y(d\pi_Y u, d\pi_Y v)\), analogous to dot product.
        \end{itemize}
    \end{example}
    \begin{definition}
        A Riemannian metric on a Lie group is Left (resp. Right) invariant if \(L_x\) (resp. \(R_x\)) is an isometry for all \(x \in G\). It is bi-invariant if it is both left adn right invariant.
    \end{definition}
    \begin{proposition}
        All Lie groups have a left invariant Riemannian metric, namely take one on \(T_eG\) and define
        \begin{align*}
            g_{\text{inv}}(u,v)_x = g(dL_{x^{-1}}u,dL_{x^{-1}}v)_e
        \end{align*}
        Moreover, if \(G\) is compact it has a bi-invariant metric.
    \end{proposition}
    \begin{proposition}
        \(g\) is bi-invariant if and only if for any \(U,V,X \in T_eG\) we have
        \begin{align*}
            g([U,X],V) = - g(U,[V,X])
        \end{align*}
    \end{proposition}
    \begin{definition}
        \(I \subset \mathbb{R}\), and \(\gamma: I \to M\), then for any closed interval \([a,b] \subset I\) we can define the length of the curve \(\gamma\), since \(g\) will give a norm. Namely
        \begin{align*}
            \ell_a^b(\gamma) = \int_a^b \sqrt{g(\gamma'(t),\gamma'(t))}dt
        \end{align*}
    \end{definition}
    \begin{definition}
        It may be desirable to define the volume form in terms of Riemannian metrics, in this case we can write locally \(x: U \subset M \to V \subset \mathbb{R}^N\) as coordinates and
        \begin{align*}
            dVol := \sqrt{\det(g_{ij})}dx_1\cdots dx_n
        \end{align*}
    \end{definition}
    \subsubsection{Exercises}
    \begin{pb}
        Put a Riemannian metric on \(T^n\) so that \(R^n \supset (x_1,\hdots,x_n) \overset{f}{\mapsto} (e^{ix_1},\hdots,e^{ix_n})\) is a local isometry, show this is isometric to the flat torus.
    \end{pb}
    \begin{proof}
        We use that the group action of \(\mathbb{Z}^n\) by translation has differential \(1\), so we can just use the metric given in the above examples. This makes \(f\) an isometry tautologically.
    \end{proof}
    \begin{pb}
        Give an immersion \(f: T^n \to \mathbb{R}^{2n}\) which is isometric to the flat Torus.
    \end{pb}
    \begin{proof}
        \begin{align*}
            f: (\theta_1,\hdots,\theta_n) \mapsto (\cos\theta_1,\sin\theta_1,\hdots,\cos\theta_n,\sin\theta_n)
        \end{align*}
        Then we cna simply evaluate
        \begin{align*}
            df\vert_{(\theta_1,\hdots,\theta_n)} = \begin{pmatrix} -\sin \theta_1 & 0 & \cdots & 0 \\ \cos\theta_1 & 0 & \cdots & 0 \\ \vdots& \ddots & \vdots & \vdots \\ 0 & 0 & \cdots & -\sin\theta_n \\ 0 & 0 & \cdots & \cos\theta_n\end{pmatrix}
        \end{align*}
        The flat metric gives us
        \begin{align*}
            g_{ij}= \delta_{ij}
        \end{align*}
        And the immersion gives us
        \begin{align*}
            g \left(d_{\underline{\theta}}f(\frac{\partial}{\partial \theta_i}),d_{\underline{\theta}}f(\frac{\partial}{\partial \theta_j})\right)_{f(\underline{\theta})} = g_{\text{EUC}}(-\sin\theta_i e_{2i-1} + \cos\theta_i e_{2i},-\sin\theta_j e_{2j-1} + \cos\theta_j e_{2j}) = \delta_{ij}
        \end{align*}
    \end{proof}
    \begin{pb}
        The lie group of Affine transformations can be given by \(\mathbb{R}\times (0,\infty)\), with \(A(t) = yt + x\), then the lie group structure is composition. Show that \(g_{11} = g_{22} = \frac{1}{y^2}\) gives a left invariant metric. Note that using this metric, and identifying \((x,y) \sim x+iy = z\) we get that the \(SL(2,\mathbb{R})\) metric acts via isometry on \(G\).
    \end{pb}
    \begin{proof}
        We simply compute \(d_{(x,y)}L_{(A,B)} = \begin{pmatrix} B & 0 \\ 0 & B \end{pmatrix}\), so that
        \begin{align*}
            &((L_{(A,B)})_*g_{(x,y)})_{11} = ((L_{(A,B)})_*g_{(x,y)})_{22} = B^2/(By)^2 = \frac{1}{y^2} \\
            &((L_{(A,B)})_*g_{(x,y)})_{12} = g_{12} = 0
        \end{align*}
    \end{proof}
    \begin{pb}
        Show that "Locally isometric" is not an equivalence relation
    \end{pb}
    \begin{proof}
        Let \(\eta\) be a bump function on \(U \subset \mathbb{R}^n\), the define \(g\) on \(\mathbb{R}^n\) as \(g_{\text{EUC}} + \eta g_{\text{EUC}}\), it is clear we can't have a local isometry with flat \(\mathbb{R}^n\) on \(\partial U\). Existence in the reverse direction is obvious, since we can just map diffeomorphically away from \(U\).
    \end{proof}
    \begin{pb}
        Show that there exists a Bi-invariant Riemannian metric on any compact connected Lie Group.
    \end{pb}
    \begin{proof}
        
    \end{proof}

\end{document}