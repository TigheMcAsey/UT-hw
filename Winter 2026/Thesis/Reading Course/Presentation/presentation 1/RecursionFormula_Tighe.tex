\documentclass[20pt]{article}
\usepackage{amsmath, amsfonts, amssymb,amsthm}
\usepackage[includeheadfoot]{geometry} % For page dimensions
\usepackage{fancyhdr}
\usepackage{enumerate} % For custom lists
\usepackage{tikz-cd}
\usepackage{graphicx}

\fancyhf{}
\lhead{Recursion Formula for Siegel Veech Constants}
\rhead{Tighe McAsey}
\pagestyle{fancy}

% Page dimensions
\geometry{a4paper, margin=1in}

\theoremstyle{definition}
\newtheorem{pb}{Exercise}[subsection]
\newtheorem{thm}{Theorem}[section]
\newtheorem{definition}{Definition}[section]
\newtheorem{proposition}{Proposition}[section]
\newtheorem{example}{Example(s)}[section]
\newtheorem{ques}{Question}[section]
\newtheorem{ans}{Answer}[section]
\newtheorem{recall}{\large{Recall}}[section]
\newtheorem{goal}{\large{Goal}}[section]
\newtheorem{detail}{\large{Detail(s)}}[section]
\newtheorem{concept}{\large{Concept}}[section]
\usepackage{tikz-cd, stackengine}

% Commands:

\newcommand{\set}[1]{\{#1\}}
\newcommand{\gen}[1]{\langle#1\rangle}
\newcommand{\abs}[1]{\left\vert#1\right\vert}
\newcommand{\norm}[1]{\lvert\lvert#1\rvert\rvert}
\newcommand{\tand}{\text{ and }}
\newcommand{\tor}{\text{ or }}
\newcommand{\pd}{\frac{\partial}{\partial x_j}}
\newcommand{\px}{\frac{\partial}{\partial x}}
\newcommand{\py}{\frac{\partial}{\partial y}}
\newcommand{\pz}{\frac{\partial}{\partial z}}
\newcommand{\ppx}{\frac{\partial^2}{\partial x^2}}
\newcommand{\ppy}{\frac{\partial^2}{\partial y^2}}
\newcommand{\ppz}{\frac{\partial^2}{\partial z^2}}
\newcommand{\hess}{\operatorname{Hess}}

\title{Recursion Formula for Siegel Veech Constants \\ \large Following Section 8 of Eskin, Masur, Zorich}
\author{Tighe McAsey}

\begin{document}
    \maketitle
    \section{Review and Vision}

    \large{We begin by recalling the purpose of our investigations, namely the computation of Siegel Veech constants. Our approach following EMZ is to compute volumes of Strata. From which we can recover the Siegel Veech constants.}

    \begin{recall}

        \large{\emph{[Can either skip over this or go over quickly]}}
        
        
        \large{\begin{itemize}
            \item Given a surface (of genus \(g\)) with abelian differential \((S,\omega)\), \(S\) is given as a surface of translation with a finite number of conic singularities \(\set{z_1,\hdots,z_k}\), with multiplicities \(m_1,\hdots,m_k\)
            \item \(\alpha\) (a partition of \(2g-2\)) is a vector recording the conic angles at the singularities.
            \item A configuration \(\mathcal{C}\) of multiplicity \(p\) records the orders of two zeroes \(z_1 \tand z_2\) joined by \(p\) saddle connections (recall generically these are homologous since by definition they have the same holonomy), \(\mathcal{C} = (m_1,m_2,a_1,\hdots,a_{p-1},a_1',\hdots,a_{p-1}')\) where \(m_1\) is the cone angle at \(z_1\), \(m_2\) is the cone angle at \(z_2\) and the angle between \(\gamma_j \tand \gamma_{j+1}\) at \(z_1\) is \(2\pi(a_j + 1)\) and the angle between them at \(z_2\) is \(2\pi(a_j' + 1)\).
            \item We are working in the strata of the form \(\mathcal{H}_1(\alpha)\) which is the subset of the space of abelian differentials \((S,\omega)\) such that the zeroes of \(\omega\) have configuration \(\alpha\), with \(S\) having unit surface area.
            \item Local coordinates on \(\mathcal{H}^1(\alpha)\) are given by choosing a basis \(\gamma_1,\hdots,\gamma_n\) (\(n = 4g+2 k - 2\)) for the relative homology \(H_1(S,\set{z_1,\hdots,z_k};\mathbb{C})\) for which each basis element is the homology class of a saddle connection. Then coordinates are given by:
            \begin{align*}
                (S,\omega) \mapsto \left(\int_{\gamma_1}\omega, \hdots,\int_{\gamma_n}\omega\right) \in \mathbb{C}^n \leftrightsquigarrow \mathbb{R}^{2n}
            \end{align*}
            \item The measure \(\mu\) on \(\mathcal{H}_1(\alpha)\) is given by lebesgue measure in these coordinates, normalized so that \(\mu(I^{2n}) = 1\).
            \item We are interested in counting the asymptotics of the number of saddle connections of generic surfaces in \(\mathcal{H}_1(\alpha)\). Namely the number of saddle connections having configuration \(\mathcal{C}\) and length less than \(L\), under the image of the developing map taking \(\gamma \mapsto \text{hol}(\gamma) \in \mathbb{R}^2\) this set is denoted as \(V_{\mathcal{C}}(S)\cap B_L\).
            \item \(\mathcal{H}_1^\epsilon(\alpha)\) is defined as the subset of the unit strata with configuration \(\alpha\) having a saddle connection of length at most \(\epsilon\).
        \end{itemize}}
    \end{recall}
    \begin{recall}
        \large{
        \begin{itemize}
            \item Siegel Veech constants are defined as follows (existence of such a constant is a result of Eskin and Masur)
            \begin{align*}
                c(\alpha,\mathcal{C}) := \lim_{L\to\infty}\frac{\# \left(V_{\mathcal{C}}(S)\cap B_L\right)}{\pi L^2}
            \end{align*}
            \item Last time we saw Richard present the proof of the formula for connected components of stratum
            \begin{align*}
                c(\alpha,\mathcal{C}) = \lim_{\epsilon \to 0} \frac{1}{\pi \epsilon^2} \frac{\text{Vol}(\mathcal{H}_1^\epsilon(\alpha,\mathcal{C}))}{\text{Vol}(\mathcal{H}_1(\alpha))}
            \end{align*}
        \end{itemize}}
    \end{recall}
    \begin{goal}
        \large{Further develop this methodology of computing Siegel-Veech constants, namely we would like to understand how to compute \(\text{Vol}(\mathcal{H}_1^\epsilon(\alpha,\mathcal{C}))\).}
    \end{goal}

    \section{Approach and Setup}

    \begin{detail}
        \begin{itemize}
            \item Roughly I will be covering the simplest case for computing \(\text{Vol}(\mathcal{H}_1^\epsilon(\alpha,\mathcal{C}))\)
            \item We continue to consider connected Strata
            \item I am only considering saddle connections of multiplicity 1, i.e. \(\mathcal{C} = (m_1,m_2)\)
            \item Later we will consider the picture for higher multiplicity?
        \end{itemize}
    \end{detail}

    \begin{example}
        Saddle connection of multiplicity 1:

        \begin{figure}[h]
            \begin{center}
                \includegraphics{Figures/MultiplicityOneSaddleConnection.png}
            \end{center}
        \end{figure}

        Saddle Connection of multiplicity \(> 1\):

        \begin{figure}[h]
            \begin{center}
                \includegraphics[width=\linewidth]{Figures/Multiplicity5SaddleConnection.png}
            \end{center}
        \end{figure}
    \end{example}

    \begin{concept}[Principle Boundary] 
        By shrinking a saddle connection \(\gamma\) (say between \(z_1\) and \(z_2\)) on a surface of type \(\alpha = (m_1,\hdots,m_k)\), we collapse to a surface of type \(\alpha' = (m_1 + m_2, m_3, m_4, \hdots, m_k)\), \(\mathcal{H}_1(\alpha')\) is called the principle boundary of \(\mathcal{H}_1(\alpha)\).
        \newpage

        \begin{figure}[h]
            \begin{center}
                \includegraphics{Figures/PrincipleBoundary.png}
            \end{center}
        \end{figure}
    \end{concept}
    
    \begin{goal}
        To understand the substance of \textbf{Lemma 8.1 (EMZ)} and its proof.

        Roughly the idea is that given a surface with a short saddle connection, we can map it to its principle boundary. Assuming this saddle connection is short, the surface should not change too much, so that (given the data of the saddle connection) we could recover the original surface.

        Assuming enough geometric information is preserved we hope to recover \(\text{Vol}(\mathcal{H}_1^\epsilon(\alpha,\mathcal{C}))\) in terms of \(\text{Vol}(\mathcal{H}_1(\alpha'))\).
    \end{goal}

    \begin{thm}[EMZ Lemma 8.1 --  Imprecise version]
        Yes, Enough information is preserved.

        Namely, up to some error we have
        \begin{align*}
            \mathcal{H}_1^\epsilon(\alpha,\mathcal{C}) &\to \mathcal{H}_1(\alpha') \times B_\epsilon \\
            (S,m_1,m_2) &\mapsto (S',m,\gamma)
        \end{align*}
        is a covering map (of degree \(m+1\)), and importantly the measure decomposes \(\mu = \mu' \times \text{Lebesgue}\)
    \end{thm}

    \begin{detail}
        The error corresponds to requiring all saddle connections other than the one being collapsed to have length atleast \(\epsilon\). This is essential for the covering map to be well defined, since otherwise a stratum could have 2 distinct boundaries

        \begin{figure}[h]
            \begin{center}
                \includegraphics[width=\linewidth]{Figures/SurfaceWith2Boundaries.png}
            \end{center}
        \end{figure}
    \end{detail}

    \section{Proof of Covering and Precise Restatement}

    \begin{proof}
        The covering map is given by collapsing a saddle connection, as previously specified, it is actually easier to understand through looking at the local inverse.

        SETUP:
        \begin{itemize}
            \item the order of the zero \(x\) we are spliting is \(m \geq 2\) and \(m_1,m_2 \in \mathbb{Z}_{>0}\) with \(m = m_1 + m_2\)
            \item \(\gamma \in B_{\epsilon} \subset \mathbb{R}^2\), with \(\abs{\text{hol}(\gamma)} = 2\delta\)
            \item Assume that all other saddle connections and closed geodesics of \(S'\) have length atleast \(2 \epsilon\) [\emph{This corresponds to our assumption ensuring that the covering map is well defined, since the surface in our local inverse will have saddles of length atleast \(\epsilon\) due to this assumption}]
            \item Let \(\set{x,z_1,\hdots,z_\ell}\) be the zeroes of \(\omega'\) the differential on \(S'\), since we will modify a small neighborhood of \(x\), we take saddle connections whos classes generate the base of our homology \(H_1(S,\set{x,z_1,\hdots,z_\ell};\mathbb{Z})\) avoiding \(B_\epsilon(x)\), if \(x\) is the only zero this can be done, if there are other zeroes then we require exactly one saddle (\(\beta\)) to have \(x\) as its endpoint
            \item Q: Why can we/do we have to do this? A: Recall since \((S,\set{x,z_1,\hdots,z_\ell})\) is a good pair (Hatcher) the relative homology is equivalent to the homology of the quotient space, in which case each saddle has a nontrivial homology class (e.g. look at the holonomy), and if none of \(\gamma_1,\hdots,\gamma_n\) are saddle connections to \(x\), then \(\beta\) will have an independent homology class (once again e.g. the holonomy can be linearly independent). Conversely if we include \(\beta\), then any other saddle connection including \(x\) is homologous to a linear combination of saddle connections avoiding it (Proof - Just draw the 2-simplex).
        \end{itemize}

        Proof:

        \begin{figure}[h]
            \begin{center}
                \includegraphics{Figures/BreakingUpAZero.png}
            \end{center}
        \end{figure}

        Glue together \(2m+2\) half disks of radius \(\epsilon\), start by gluing the opposing discs along the new saddle connection \(\gamma\)

        Considerations:

        \begin{itemize}
            \item This process is ambiguous up to choosing where \(\gamma\) is added, this depends on the choice of disks to be glued along gamma, of which there are \(m+1\) pairs, each pair corresponds to 2 gluings by mirroring, but by marking which point is \(x_1\) and \(x_2\), fixing the holonomy of \(\gamma\) gives one choice. This is where the covering index of \(m+1\) comes from.
            
            \begin{figure}[h]
            \begin{center}
                \includegraphics{Figures/CoveringIndex.png}
            \end{center}
        \end{figure}

            \item Since we perform this construction locally, the rest of the surface i.e. \(B_\epsilon^c\) remains unchanged, namely the holonomy is unchanged on the homology basis of \(S'\) (which includes the homology basis of \(S\)) apart from \(\text{hol}(\beta)\) being changed by a factor of \(-\gamma/2\). By the triangle inequality all of the lifts lie in \(H^{\epsilon,\epsilon}_1(\alpha,m_1,m_2)\)
        \end{itemize}

        Now we consider the case of collapsing zeroes, here we clearly see the error in the above being a covering map, the cover is ramified and maps only to a ``large'' subset of \(\mathcal{H}^1(\alpha')\)

        SETUP:

        \begin{itemize}
            \item Let \(S \in \mathcal{H}^{\epsilon,3 \epsilon}_1(\alpha)\) i.e. \(S\) has a saddle connection \(\gamma\) connecting zeroes \(x_1, x_2\) of orders \(m_1\) and \(m_2\), \(\text{hol}(\gamma) \in B_\epsilon\), and all other saddle connections of \(S\) have length atleast \(3 \epsilon\).
            \item Considering the relative homology \(H_1(S,\set{x_1,x_2,z_1,\hdots,z_\ell})\), similarly to adding a saddle connection, we choose our basis for relative homology away from \(B_\epsilon(\text{midpoint of } \gamma)\), we furthermore take \(\gamma\) in this homology basis, similarly to last time if \(\ell > 0\) we will need to take a saddle conection \(\beta\) connecting one of the \(z_j\) to \(x_1\) in our basis.
        \end{itemize}

        Proof: We break up \(B_\epsilon(\text{midpoint of } \gamma)\) into \(2m+2\) half disks, in the exact same way as breaking up a zero. We then collapse the saddle connection, while leaving the manifold the same outside of the \(\epsilon\) ball, equivalently we are reversing the breaking up process.

        \textcolor{red}{REFER BACK TO IMAGE}

        Considerations:

        \begin{itemize}
            \item Since this construction is local, the holonomy of the basis vectors for \(H_1(S',x,z_1,\hdots,z_\ell)\) is unchanged apart from \(\beta\) which gets replaced by a curve to \(x\), hence the holonomy of \(\beta\) is changed by at most \(\frac12\text{hol}(\gamma)\)
            \item From this we know that the every saddle connection on \(S'\) has length atleast \(2 \epsilon\)
        \end{itemize}
    \end{proof}

    Failure of this map to be a cover:

    \begin{itemize}
        \item To deal nicely with covering \(\mathcal{H}_1(\alpha')\) we will actually work with (and cover) a large (in sense of measure) subset of \(\mathcal{H}_1(\alpha')\), so that our coordinates are simple, and our map is surjective on this set.
        \item We also restricted the image of our mapping into \(\mathcal{H}_1^{> 2 \epsilon}(\alpha')\), which should be accounted for.
        \item Finally, we would like to work with a singular coordinate system on \(\mathcal{H}_1^{>2 \epsilon}(\alpha')\), this makes it straightforward to see that the image of the map lies in \(\mathcal{H}_1^{>2 \epsilon}(\alpha') \times B_\epsilon\)
        \item We can do this on a set \(\mathcal{F} \subset \mathcal{H}^{>2 \epsilon}_1(\alpha')\) with \(\mu(\mathcal{H}^{>2 \epsilon}_1(\alpha') \setminus \mathcal{F})\) arbitrarily small.
        \item First notice that if \(\mathcal{F}\) is simply connected, then we can choose a single coordinate chart. Why?
        \item Choosing a basis for homology in saddle connections (i.e. a chart) at a point, we may extend to an open subset containing that point. We may further extend globally assuming that these choices are compatible, \(\mathcal{F}\) being simply connected implies compatibility.
        \item Well definedness is equivalent to for every path \(\gamma: I \to \mathcal{F}\), covering \(\gamma(I)\) with compatible charts, we get a consistent choice of chart at \(\gamma(1)\). It in fact suffices to check this for closed paths, since if \(\gamma\) and \(\gamma'\) are paths to the same point, and \(\gamma\gamma'^{-1}\) does not effect the chart, then the chart on \(\gamma'\) is the same as the chart on \(\gamma = \gamma\gamma'^{-1} \gamma'\). The argument for closed loops is as follows: Assume \(\gamma: S^1 \to \mathcal{F}\), then by simply connectedness \(\gamma \sim 1\) by the homotopy \(H\), extend the finite cover of \(\gamma(S^1)\) by compatible charts to one on \(H(S^1 \times I)\), the chart at \(\gamma(0)\) is invariant under the homotopy and is the original chart when \(t = 1\) since \(H\) is the constant map.
        \item How do we take such an arbitrarily large simply connected set? This follows from \(\mu'\) being a borel measure (it is locally lebesgue so is of course defined by outer measure), so by regularity and finiteness there is a compact set \(K\) with \(\mu'(\mathcal{H}_1^{>2 \epsilon}(\alpha')) - \mu'(K)\) arbitrarily small.
        \item Now we would like to cover \(K\) with finitely many disjoint coordinate charts \(U_1,\hdots,U_j\), with \(K \setminus \bigcup_1^j U_i\) having arbitrarily small volume, then we may connect these coordinate charts with a spanning tree, since each chart looks like a hollow ball, this is a simply connected subset with volume \(> \mu'(K) -\delta\) for any \(\delta > 0\).
        \item To define these coordinate charts \(U_i\) take an arbitrary finite open cover by charts \(V_1,\hdots,V_r\), taking a suboordinate partition of unity we can define \(U_i = \set{\eta_i > \eta_\ell}\) for all \(\ell\). We can choose the partition of unity so that \(\eta_i - \eta_\ell\) is a submersion for each \(i,\ell\), this implies that \(\left(\bigcup U_i\right)^c\) is measure zero by Sard's theorem.
        \item In this case \(\mathcal{F} = \bigsqcup U_i \bigcup T\) where \(T\) is a spanning tree connecting them is simply connected.
        \item Since the volume of \(\mathcal{F}\) is arbitrarily close to that of \(\mathcal{H}^{>2 \epsilon}_1(\alpha')\)  we have by Lemma 7.1 of EMZ that
        \begin{align*}
            \mu'(\mathcal{H}_1(\alpha') - \mathcal{F}) = O(\epsilon^2)
        \end{align*}

        \item Since \(\mathcal{H}_1^{\epsilon,3 \epsilon}\) maps onto \(\mathcal{F} \times B_\epsilon\) we are almost done establishing the covering is of degree \(m+1\), however our argumant for degree did not take into account possible ramification points which we address now.
        \item The ramification points of the cover correspond to the surfaces \((S',\omega') \in \mathcal{F}\) with multiple of the \(m+1\) surfaces \((S,\omega)\) achieved by the saddle construction being isomorphic.
        \item The set of points which are images of ramification points has measure zero. Proof being, suppose that \(S_1, S_2\) which both arise from the adding a saddle connection on \(S'\) are isomorphic. Then the isomorphism \(S_1 \to S_2\) must send the (unique) saddle connection \(\gamma_1\) of length less than \(\epsilon\) on \(S_1\) to the short saddle connection \(\gamma_2\) on \(S_2\), and hence an open ball around \(\gamma_1\) to an open ball around \(\gamma_2\), hence the compliments of these open balls are isomorphic. This induces an automorphism on \(S'\), but the set of flat surfaces with automorphisms is measure zero.
        \item Q: Why is this set measure zero? Ans: There are finitely many topological automorphisms (i.e. induce automorphisms on homology) (Hurwitz bound?), a fixed (homnology class of) automorphisms \(f\) satisfies for each of the saddle \(\int_{\gamma_k}f^*\omega = \int_{\gamma_k}\omega\), so that \(\text{hol}(\gamma_k)\) are fixed by \(f_*\), so that the set of automorphic surfaces lie in the eigenspace of \(f_*\), if \(f_* \neq 1_*\) this is a proper linear subspace and is thus lower dimensional.
        \item So we have established that we have a ramified cover \(\mathcal{H}_1^{\epsilon,3 \epsilon}(\alpha) \to \mathcal{F} \times B_\epsilon\) ramified over a set of measure zero.
    \end{itemize}

    \section{Measure Theoretic Considerations}
    Topologically we have defined the covering map we are interested in, but in order to compute volumes we require the measure to decompose nicely over the map. Most of the work in this aspect is done by considering \(\mathcal{F} \subset \mathcal{H}_1^{>2 \epsilon}(\alpha')\), since we can work in a single chart.

    \begin{itemize}
        \item Taking a homology basis for \(\mathcal{F}\), the saddle connections covering some \(S' \in \mathcal{F}\) have the same monodromy (apart from \(\beta \rightsquigarrow \beta'\)). It is easy to see in notation that in coordinates the covering map gives us
        \begin{align*}
            \left(\int_{\gamma_1} \omega, \hdots, \int_{\gamma_{n-1}} \omega, \int_\beta \omega, \int_{\gamma}\omega\right) \mapsto \left(\left(\int_{\gamma_1} \omega, \hdots, \int_{\gamma_{n-1}} \omega, \int_{\beta '} \omega\right),\int_{\gamma}\omega\right)
        \end{align*}
        
        \item The only thing preventing the volume element from splitting is \(\beta \rightsquigarrow \beta'\), but we previously controlled this error as \(\text{hol}(\beta') = \text{hol}(\beta) - \frac{1}{2}\text{hol}(\gamma)\), so we can check directly the error in the volume computation.
        \begin{align*}
            \mu(\mathcal{H}^{\epsilon}_1(\alpha,\mathcal{C})) &\overset{\text{Prop 7.1}}{=} \mu(\mathcal{H}_1^{\epsilon,3 \epsilon}(\alpha, \mathcal{C})) + \mathcal{O}(\epsilon^4) \\ 
            \mu(\mathcal{H}_1^{\epsilon,3 \epsilon}(\alpha, \mathcal{C})) &= \int_{\mathcal{H}_1^{\epsilon,3 \epsilon}(\alpha,\mathcal{C})} \left(\text{hol}(\gamma_1),\hdots,\text{hol}(\gamma_{n-1}),\text{hol}(\beta),\text{hol}(\gamma)\right)d\mu \\
            &= \int_{\mathcal{H}_1^{\epsilon,3 \epsilon}(\alpha,\mathcal{C})} \left(\text{hol}(\gamma_1),\hdots,\text{hol}(\gamma_{n-1}),\text{hol}(\beta) - \frac12 \text{hol}(\gamma),\text{hol}(\gamma)\right)d\mu \\ &+ \int_{\mathcal{H}_1^{\epsilon,3 \epsilon}(\alpha,\mathcal{C})} \left(\text{hol}(\gamma_1),\hdots,\text{hol}(\gamma_{n-1}), \underbrace{\frac12 \text{hol}(\gamma),\text{hol}(\gamma)}_{\mathcal{O}(\epsilon^4)}\right)d\mu \\
            &= (m+1)\int_{\gamma \in B_\epsilon}d\text{Lebesgue}(\gamma)\int_{\mathcal{F}} \left(\text{hol}(\gamma_1),\hdots,\text{hol}(\gamma_{n-1}),\text{hol}(\beta)\right)d\mu' + \mathcal{O}(\epsilon^4) \\
            &= (m+1)\pi \epsilon^2 \mu'(\mathcal{F}) + \mathcal{O}(\epsilon^4) \\
            &= (m+1)\pi \epsilon^2 \left(\mu'(\mathcal{H}_1(\alpha')) + \mu'(\mathcal{H}_1(\alpha') \setminus \mathcal{F})\right) + \mathcal{O}(\epsilon^4)
        \end{align*}
        Then \(\mu'(\mathcal{H}_1(\alpha') \setminus \mathcal{F}) = \mathcal{O}(\epsilon^2)\) by (EMZ Prop. 7.1), so that
        \begin{align*}
            \mu(\mathcal{H}_1^\epsilon(\alpha,\mathcal{C})) = (m+1)\pi \epsilon^2 \mu'(\mathcal{H}_1(\alpha')) + \mathcal{O}(\epsilon^4)
        \end{align*}

        This finally resolves the issue of computing the limit,
        \begin{align*}
            c(\alpha,\mathcal{C}) = \lim_{\epsilon \to 0} \frac{\mu(\mathcal{H}_1^\epsilon(\alpha,\mathcal{C}))}{\pi \epsilon^2 \mu(\mathcal{H}_1(\alpha))} = \frac{(m+1)\mu'(\mathcal{H}_1(\alpha'))}{\mu(\mathcal{H}_1(\alpha))}
        \end{align*}
    \end{itemize}

    \section{One Final Technical Issue}

    I have actually been cheating this entire time. The covering map/ surgery may not be volume preserving.

    \begin{itemize}
        \item Don't fret too much, all of our previous work is not all for nought. The prior arguments still preserve the fact our map is
        \begin{align*}
            \mathcal{H}_1^\epsilon(\alpha) \to \mathcal{H}(\alpha') \times B_\epsilon
        \end{align*}
        and the corresponding more precise version.
        \item So topologically our map still makes sense since if the surface which is a result of the surgery is \((S',\omega')\), since we can rescale \(\omega'\) to \(r\omega'\) so that our surface as unit area.
        \item Unfortunately, this breaks our earlier measure theoretic argument, the conclusion is still true but now we need to integrate in the total space of the strata.
        \item To recover the unit strata volumes we integrate over the cone, i.e. if \(A \subset \mathcal{H}_1(\alpha)\) Then
        \begin{align*}
            \text{Cone}(A) := \set{(S,r\omega) \mid (S,\omega) \in A, r \in (0,1]}
        \end{align*}
        So that for \(n = \dim(\mathcal{H}_1(\alpha))\), \(n = 4g + 2k - 2\) we get \[\mu\otimes\text{Lebesgue}(\text{Cone(A)}) = \int_0^1 r^{n-1}dr\mu(A) = \frac{1}{n}\mu(A)\]
        \item Repeating the above argument for the cone and taking \(\mathcal{F}\) to be its rescaled image in \(\mathcal{H}_1(\alpha')\), we get
        \begin{align*}
            \mu(\mathcal{H}^{\epsilon}_1(\alpha,\mathcal{C})) &= n\cdot \mu\otimes\text{Lebesgue}(\text{Cone}(\mathcal{H}^{\epsilon,3 \epsilon}_1(\alpha,\mathcal{C}))) + \mathcal{O}(\epsilon^4) \\
            &= n(m+1)\mu'(\mathcal{F})\int_0^1 r^{n' - 1}\int_{B(\epsilon r)}d\gamma dr + \mathcal{O}(\epsilon^4) \\
            &= n(m+1)\mu'(\mathcal{F})\int_0^1 r^{n'-1}\pi \epsilon r^2 + \mathcal{O}(\epsilon^4) \\
            &= \frac{n(m+1)}{\underbrace{n' + 2}_{n' = n - 2}}\mu'(\mathcal{F}) + \mathcal{O}(\epsilon^4) \\
            &=(m+1)\mu'(\mathcal{F}) + \mathcal{O}(\epsilon^4)
        \end{align*}
    \end{itemize}

    \section{Example: \(\Sigma_3\)}

\end{document}