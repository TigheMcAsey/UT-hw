\documentclass[10.5pt]{article}
\usepackage{amsmath, amsfonts, amssymb,amsthm}
\usepackage[includeheadfoot]{geometry} % For page dimensions
\usepackage{fancyhdr}
\usepackage{enumerate} % For custom lists
\usepackage{tikz-cd}
\usepackage{graphicx}

\fancyhf{}
\lhead{MAT1002 hw2}
\rhead{Tighe McAsey - 1008309420}
\pagestyle{fancy}

% Page dimensions
\geometry{a4paper, margin=1in}

\theoremstyle{definition}
\newtheorem{pb}{}
\usepackage{tikz-cd, stackengine}

% Commands:

\newcommand{\set}[1]{\{#1\}}
\newcommand{\gen}[1]{\langle#1\rangle}
\newcommand{\abs}[1]{\left\vert#1\right\vert}
\newcommand{\norm}[1]{\lvert\lvert#1\rvert\rvert}
\newcommand{\tand}{\text{ and }}
\newcommand{\tor}{\text{ or }}
\newcommand{\pd}{\frac{\partial}{\partial x_j}}
\newcommand{\px}{\frac{\partial}{\partial x}}
\newcommand{\py}{\frac{\partial}{\partial y}}
\newcommand{\pz}{\frac{\partial}{\partial z}}
\newcommand{\ppx}{\frac{\partial^2}{\partial x^2}}
\newcommand{\ppy}{\frac{\partial^2}{\partial y^2}}
\newcommand{\ppz}{\frac{\partial^2}{\partial z^2}}
\newcommand{\hess}{\operatorname{Hess}}

\begin{document}
    \begin{pb}
        \begin{align*}
            \Im(\sigma) &= \Im\left(\frac{(a\tau + b)(c\overline{\tau}+d )}{\norm{c\tau + d}^2}\right) = \Im\left(\frac{ac\norm{\tau}^2 + bd + ad\tau + bc \overline{\tau}}{\norm{c\tau + d}^2}\right) \\ &= \frac{1}{\norm{c\tau + d}^2}\Im(ad\tau + bc \overline{\tau}) = \frac{1}{\norm{c\tau + d}^2} (ad-bc)\Im(\tau) = \frac{1}{\norm{c\tau + d}^2}\Im(\tau)
        \end{align*}
        Then since \(\norm{c\tau + d}^2 > 0\) and \(\Im(\tau) > 0\) we conclude that \(\Im(\sigma) > 0\).

        Now define 
        \begin{align*}
            f: \mathbb{C} &\to \mathbb{C}/(\mathbb{Z}+\sigma \mathbb{Z}) \\
            z &\mapsto \frac{z}{c\tau + d} + \mathbb{Z}+\sigma \mathbb{Z}
        \end{align*}
        To see this descends to a holomorphic map on the torus, we need to check it is periodic with respect to \(\mathbb{Z} + \tau \mathbb{Z}\), so we want to check that \(f(z + n \tau + m) = f(z)\) for any \(n,m \in \mathbb{Z}\). Since \(f\) is linear it will be sufficient to show that \(f(\tau)\) and \(f(1)\) both lie in the lattice \(\mathbb{Z} + \sigma \mathbb{Z}\). This is just a computation,
        \begin{align}
            \tau = (ad - bc)\tau + bd - bd = d(a \tau + b) - b(c\tau + d) \\
            1 = (ad-bc) + ac \tau - ac \tau = -c(a \tau + b) + a(c \tau + d)
        \end{align}
        So that (1) gives us \(f(\tau) = d\sigma - b\) and (2) gives \(f(1) = -c \sigma + a\) both are in \(\mathbb{Z} + \sigma \mathbb{Z}\), so that \(f\) descends to the torus \(X_\tau\). To see that \(f\) is a biholomorphism just take \(\mathbb{C} \to X_\tau\) via \(z \mapsto (c \tau + d)z\), this descends to a holomorphic map on \(X_\sigma\) since \(\sigma \mapsto \tau\), and \(1 \mapsto c\tau + d\) are both in \(\mathbb{Z} + \tau \mathbb{Z}\), moreover this is clearly the inverse of \(f\). \qed
    \end{pb}
\end{document}