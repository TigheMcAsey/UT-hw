\documentclass[10.5pt]{article}
\usepackage{amsmath, amsfonts, amssymb,amsthm}
\usepackage[includeheadfoot]{geometry} % For page dimensions
\usepackage{fancyhdr}
\usepackage{enumerate} % For custom lists
\usepackage{tikz-cd}
\usepackage{graphicx}

\fancyhf{}
\lhead{MAT1002 hw1}
\rhead{Tighe McAsey - 1008309420}
\pagestyle{fancy}

% Page dimensions
\geometry{a4paper, margin=1in}

\theoremstyle{definition}
\newtheorem{pb}{}
\usepackage{tikz-cd, stackengine}

% Commands:

\newcommand{\set}[1]{\{#1\}}
\newcommand{\gen}[1]{\langle#1\rangle}
\newcommand{\abs}[1]{\left\vert#1\right\vert}
\newcommand{\norm}[1]{\lvert\lvert#1\rvert\rvert}
\newcommand{\tand}{\text{ and }}
\newcommand{\tor}{\text{ or }}
\newcommand{\pd}{\frac{\partial}{\partial x_j}}
\newcommand{\px}{\frac{\partial}{\partial x}}
\newcommand{\py}{\frac{\partial}{\partial y}}
\newcommand{\pz}{\frac{\partial}{\partial z}}
\newcommand{\ppx}{\frac{\partial^2}{\partial x^2}}
\newcommand{\ppy}{\frac{\partial^2}{\partial y^2}}
\newcommand{\ppz}{\frac{\partial^2}{\partial z^2}}
\newcommand{\hess}{\operatorname{Hess}}

\begin{document}

    \begin{pb}
        We first find an expression for \(\zeta(s)\Gamma(s)\) on \(\Re(s) > 1\), to do so we will do a substitution \(u = x/n\), valid for any \(n \in \mathbb{Z}_{>0}\)
        \begin{align*}
            \Gamma(s) = \int_0^\infty e^{-nu}(nu)^{s-1} ndu = n^s \int_0^\infty u^{s-1}(e^{-u})^n du
        \end{align*}
        Now since I don't like \(u\) I will switch back to \(x\); Multiplying \(n^{-s}\) on both sides and summing over \(n \in \mathbb{Z}_{>0}\) yields
        \begin{align}
            \zeta(s)\Gamma(s) = \sum_1^\infty \int_0^\infty x^{s-1}(e^{-x})^ndx \overset{\text{DCT}}{=} \int_0^\infty x^{s-1}\sum_1^\infty (e^{-x})^n dx = \int_0^\infty x^{s-1}\frac{1}{e^x - 1}dx
        \end{align}
        Where DCT is taken with respect to \(\abs{x^{s-1}}\sum_1^\infty e^{-nx}\). We can use \(\Gamma(s)\Gamma(1-s) = \frac{\pi}{\sin(\pi s)}\) to rewrite (1):
        \begin{align}
            \zeta(s) = \frac{\Gamma(1-s)\sin(\pi s)}{\pi}\int_0^\infty \frac{x^{s-1}}{e^x-1}dx
        \end{align}
        Now we deal with the contour integral, letting \(C(\epsilon)\) denote the curve described in the problem for fixed \(\epsilon \in \mathbb{R}_{> 0}\). The expression \(e^z-1\) has no poles away from \(\set{z \mid e^z = 1} = \set{2\pi ki \mid k \in \mathbb{Z}}\), whence if we take the branch cut of log away from the non-negative reals the integral is not dependent on \(\epsilon\) for \(\epsilon < 2\pi\), since (for \(\delta < \epsilon < 2\pi\)) we have the area enclosed between the two curves is a quotient of holomorphic functions with the denominator non-vanishing in between the curves. This independence is a result of taking seperating the curve \(C(\epsilon) - C(\delta)\) into two curves (see picture), the first of which has integral zero by Cauchy's theorem, and the second being arbitrarily small depending on where we take the cut.
        \begin{align*}
            \int_{C(\epsilon)}f - \int_{C(\delta)}f = \int_{C(\epsilon,\delta,M)} f + \int_{\gamma_1(\epsilon,\delta,M)} f + \int_{\gamma_2(\epsilon,\delta,M)} f
        \end{align*}
        
        \begin{figure}[h]
            \begin{center}
                \includegraphics[width=0.8\linewidth]{Contourfig.pdf}
            \end{center}
        \end{figure}
        
        Then for \(f\) holomorphic away from he real line, the \(C(\epsilon,\delta, M)\) term vanishes. Notice now that using the standard arclength inequality for large \(M\) we have
        \begin{align*}
            \abs{\int_{\gamma_j(\epsilon,\delta,M)} \frac{(-z)^{s-1}}{e^z - 1}dz} &\leq (\epsilon - \delta)\abs{\frac{M^{\Re(s)-1}}{e^M - 1}} + \abs{\int_M^\infty \frac{(-x- i\epsilon)^{s-1}}{e^{x+i \epsilon} - 1}dx - \int_M^\infty \frac{(-x - i\delta)^{s-1}}{e^{x+i \delta} - 1}dx} \\
            &\leq (\epsilon - \delta)\abs{\frac{M^{\Re(s)-1}}{e^M - 1}} + 2\int_M^\infty\frac{\abs{x + i \epsilon}^{\Re(s) - 1}}{\abs{e^x} - 1}dx
        \end{align*}
        The right hand side clearly converges to zero. as \(M \to \infty\) using basic limits of exponentials and DCT. This gives the desired invariance.
        \begin{align*}
            \abs{\int_{C(\epsilon)} \frac{(-z)^{s-1}}{e^z - 1}dz - \int_{C(\delta)} \frac{(-z)^{s-1}}{e^z - 1}dz} \leq \abs{\int_{\gamma_1(\epsilon,\delta,M)} \frac{(-z)^{s-1}}{e^z - 1}dz} + \abs{\int_{\gamma_2(\epsilon,\delta,M)} \frac{(-z)^{s-1}}{e^z - 1}dz} = 0
        \end{align*}
        Now, we can compute the value of the integral along this curve by letting \(\epsilon \to 0\), to get \(C(0)\), a ray from \(\infty\) to 0 where \((-z)^{s-1} = x^{s-1}e^{-(s-1)\pi i}\), and a ray from \(0\) to \(\infty\) where \((-z)^{s-1} = x^{s-1}e^{(s-1)\pi i}\), to see that we can indeed pass to this limit, once again decompose \(C = C(\epsilon)\) into three curves, with \(C_1\) the ray in the upper half plane, \(C_2\) the ray in the lower half plane and \(C_3\) the circular portion, then once again using the arc length inequality and the fact that \(e^z - 1 = \mathcal{O}(z)\)
        \begin{align*}
            \abs{\int_{C_3}\frac{(-z)^{s-1}}{e^z-1}} \leq 2\pi \epsilon\sup_{\abs{z} = \epsilon}\frac{\abs{z^{s-1}}}{\abs{e^z - 1}} = 2\pi \epsilon \mathcal{O}(\epsilon^{\Re(s) - 2}) = 2\pi \mathcal{O}(\epsilon^{\Re(s) - 1}) \overset{\Re(s) > 1}{\longrightarrow} 0
        \end{align*}
        We can check convergence on \(C_1\) explicitly
        \begin{align*}
            &\abs{\int_0^\infty\frac{x^{s-1}e^{-(s-1)\pi i}}{e^x - 1}dx - \int_\epsilon^\infty \frac{(-x-i \epsilon)^{s-1}}{e^{x + i \epsilon} - 1}dx} \\ \leq &\abs{\int_0^\epsilon \frac{x^{s-1}e^{-(s-1)\pi i}}{e^x - 1}dx} + \abs{\int_\epsilon^\infty\frac{x^{s-1}e^{-(s-1)\pi i}}{e^x - 1}dx - \int_\epsilon^\infty \frac{(-x-i \epsilon)^{s-1}}{e^{x + i \epsilon} - 1}dx} \\
            \leq &\abs{\int_0^\epsilon \frac{x^{s-1}e^{-(s-1)\pi i}}{e^x - 1}dx} + \abs{\int_\epsilon^\infty \frac{(e^{x + i \epsilon} - 1)x^{s-1}e^{-(s-1)\pi i} - (e^x - 1) (\sqrt{x^2 + \epsilon^2})^{(s-1)}e^{i(s-1)\arctan\frac{\epsilon}{x} - (s-1)\pi i}}{(e^{x} - 1)(e^{x + i \epsilon} - 1)}}
        \end{align*}
        Convergence as \(\epsilon \to 0\) of the big ugly term to zero is actually simple from convergence of each of the terms in the two expressions in the product. Convergence of the first term follows from \(\Re(s) > 1\), so writing the bounds of integration as \(\chi_{(0,\epsilon)}\) we can just apply DCT to the absolute value of the integrand. The proof of convergence for \(C_2\) is similar to \(C_1\).

        Now we finally established that (due to taking the limit in \(\epsilon\) and invariance with respect to \(\epsilon\))
        \begin{align}
            \int_C \frac{(-z)^{s-1}}{e^z - 1}dz &= -\int_0^\infty \frac{x^{s-1}e^{-(s-1)\pi i}}{e^x - 1}dx + \int_0^\infty \frac{x^{s-1}e^{(s-1) \pi i}}{e^x - 1}dx \\
            &= \int_0^\infty \frac{x^{s-1}}{e^x - 1}2i\sin((s-1)\pi) = -\int_0^\infty \frac{x^{s-1}}{e^x - 1}2i\sin(s\pi)
        \end{align}
        Multiplying (2) by \(1 = \frac{\int_C \frac{(-z)^{s-1}}{e^z - 1}dz}{-\int_0^\infty \frac{x^{s-1}}{e^x - 1}2i\sin(s\pi)}\) yields the desired equality
        \begin{align}
            \zeta(s) = -\frac{\Gamma(1-s)}{2\pi i}\int_C \frac{(-z)^{s-1}}{e^z - 1}dz
        \end{align}
        Now, using the right side of (5) as an analytic continuation, we first compute \((e^z - 1)^{-1} = \frac{1}{z} - \frac12 + \frac{z}{12} + \mathcal{O}(z^2)\), since we know it has a simple pole at zero, hence is meromorphic this expression just comes from evaluating the systems of equations given for the coefficients for \((a_{-1}z^{-1} + a_0 + a_1z + \cdots)(\sum_0^\infty \frac{z^n}{n!}) = 1\). This is everything we need to evaluate \(\zeta(0)\). First using a similar separation of curves into two parts as was used for invariance, we find that \(\int_C\frac{1}{z(e^z - 1)}dz = \int_{\gamma}\frac{1}{z(e^z - 1)}dz\) where \(\gamma\) is a piecewise \(C^1\) closed curve, this allows us to use the residue theorem
        \begin{align}
            \zeta(0) &= \frac{\Gamma(1)}{2\pi i}\int_\gamma \frac{1}{z(e^z - 1)}dz = \frac{1}{2\pi i}\int_\gamma\frac{1}{z}\left(\frac{1}{z} - \frac12 + \frac{z}{12} + \mathcal{O}(z^2)\right)dz \\
            &= \text{Res}\left(\frac{1}{z}\left(\frac{1}{z} - \frac12 + \frac{z}{12} + \mathcal{O}(z^2)\right)\right) = -\frac12
        \end{align}
        \qed
    \end{pb}
    \begin{pb}
        \textbf{(a)} Let \(\epsilon\) be small enough so that \(B_{2 \epsilon}(z_0) \subset D\), then
        \begin{align*}
            \int_{\partial D} \frac{f(w)}{w-z_0}dw - \int_{\partial B_\epsilon(z_0)} \frac{f(w)}{w-z_0}dw \overset{\text{Stokes}}{=} \int_{D \setminus B_\epsilon(z_0)} \frac{\partial}{\partial \overline{w}}\frac{f(w)}{w-z_0}d \overline{w}\wedge dw = 0
        \end{align*}
        By Cauchy Riemann since \(\frac{f(w)}{w-z_0}\) is holomorphic in \(D \setminus B_\epsilon(z_0)\). From this, we cna prove the base case which is Cauchy's integral formula.
        \begin{align*}
            \int_{\partial D}\frac{f(w)}{w-z_0}dw = \int_{B_\epsilon(z_0)} \frac{f(w)}{w-z_0}dw = \int_0^{2\pi}f(z_0 + \epsilon e^{it})i dt
        \end{align*}
        Which holds for arbitrary \(\epsilon\), but since \(f\) is bounded (on \(D\)) and continuous we have
        \begin{align*}
            \int_{\partial D}\frac{f(w)}{w-z_0}dw &= \lim_{\epsilon \to 0}\int_0^{2\pi}f(z_0 + \epsilon e^{it})i dt = \int_0^{2\pi} \lim_{\epsilon \to 0}f(z_0 + \epsilon e^{it})i dt = 2\pi i f(z_0)
        \end{align*}
        Which proves the desired result in the case of \(n = 0\)
        \begin{align}
           f(z_0) = \frac{1}{2\pi i}\int_{\partial D}\frac{f(w)}{w-z_0}dw
        \end{align}
        Now, we can proceed by induction and simply take derivatives.
        \begin{align*}
            2\pi i\lim_{h \to 0} \frac{1}{h}(f^{(n)}(z_0 + h) - f^{(n)}(z_0)) &= \lim_{h \to 0} \frac{1}{h} \int_{\partial D} \frac{f(w)}{(w-z_0-h)^{n+1}} - \frac{f(w)}{(w-z_0)^{n+1}} dw\\
            &= \lim_{h\to 0}\int_{\partial D} \frac{1}{h}\frac{f(w)(w-z_0)^{n+1} - f(w)(w-z_0-h)^{n+1}}{(w-z_0-h)^{n+1}(w-z_0)^{n+1}}dw \\
            &= \lim_{h\to 0} \int_{\partial D}\frac{f(w)(w-z_0)^{n}}{(w-z_0-h)^{n+1}(w-z_0)^{n+1}} + \mathcal{O}(h)dw
        \end{align*}
        Then we can clearly apply DCT for \(\abs{h} < \frac12 \abs{w - z_0}\), which gives us
        \begin{align}
            \lim_{h \to 0} \frac{1}{h}(f^{(n)}(z_0 + h) - f^{(n)}(z_0)) = \frac{1}{2\pi i}\int_{\partial D}\frac{f(w)}{(w-z_0)^{n+2}}dw
        \end{align}

        \textbf{(b)} Let \(z \in \mathbb{C}\), then by equation (9) in part (a)
        \begin{align*}
            f'(z) = \frac{1}{2\pi i}\int_{\abs{w-z} = R}\frac{f(w)}{(w - z)^2}dw
        \end{align*}
        Applying the standard arc-length inequality yields
        \begin{align}
            \abs{f'(z)} \leq \frac{1}{2\pi}2\pi R \frac{\sup_\mathbb{C} \abs{f}}{R^2} = \frac{\sup_\mathbb{C}\abs{f}}{R} \overset{R \to \infty}{\longrightarrow} 0
        \end{align}
        Therefore \(f' \equiv 0\), for any two points in \(\mathbb{C}\), to see this implies \(f\) is constant we use the fundamental theorem of calculus, if \(z_0,z_1 \in \mathbb{C}\) take \(\gamma\) to be the straight line starting at \(z_0\) and ending at \(z_1\) so that
        \begin{align*}
            \abs{f(z_1) - f(z_0)} = \abs{\int_\gamma f'(z)dz} \leq \ell(\gamma)\sup_\mathbb{C}\abs{f'} = 0
        \end{align*} \qed

        \textbf{(c)} Suppose a polynomial \(P\) does not have a root. Then \(\frac{1}{P}\) is entire, hence \(1/P\) is constant by Liouville's theorem. We conclude any polynomial without a root is of degree zero. \qed
    \end{pb}
    \begin{pb}
        Assume for contradiction that \(f\) has an essential singularity at \(z_0\), but there exists some \(a \in \mathbb{C}\) and \(\delta > 0\) such that for some \(\epsilon > 0\) we have \(f(D^*_\epsilon)(z_0) \cap B_\delta(a) = \emptyset\). Now define a new function \(g(z) = \frac{1}{f(z) - a}\), since \(\abs{f(z) - a} \geq \delta\) on \(D_\epsilon^*(z_0)\) we find that \(g\) is holomorphic and nonvanishing on this punctured annulus with modulus bound above by \(\frac{1}{\delta}\). Then we can define on \(D_\epsilon(z_0)\)
        \begin{align}
        (z-z_0)^2g(z) = \begin{cases}
            (z-z_0)^2g(z) & z \neq z_0 \\
            0 & z = z_0
        \end{cases}
        \end{align}
        This is holomorphic away from \(z_0\), thus is holomorphic since
        \begin{align*}
            \lim_{h\to 0} \frac{(z_0 + h-z_0)^2g(z_0 +h)}{h} = \lim_{h\to 0}hg(z_0 + h) = 0
        \end{align*}
        where the last equality is a consequence of \(g\) being bounded. Therefore we can write
        \begin{align*}
            g(z)(z-z_0)^2 = \sum_0^\infty a_k(z-z_0)^k
        \end{align*}
        which gives the following expression on \(D_\epsilon^*\) for \(b_k = a_{k+2}\)
        \begin{align}
            g(z) = \sum_{-2}^\infty b_k(z-z_0)^k = \sum_0^\infty b_k(z-z_0)^k
        \end{align}
        Where \(b_{-2} = b_{-1} = 0\) follows from \(g\) being bounded near \(z_0\). Moreover since \(g\) is nonvanishing, we must have some \(b_N \neq 0\), it follows that
        \begin{align}
            \lim_{z\to z_0}(z-z_0)^{N+1}(f(z) - a) = \lim_{z \to z_0}\frac{(z-z_0)^{N+1}}{g(z)} \overset{(12)}{=} \lim_{z\to z_0} \frac{z-z_0}{b_N + \mathcal{O}(z-z_0)} = 0
        \end{align}
        Now we can show directly from the definition that \(f\) has a pole
        \begin{align*}
            \lim_{z\to z_0}(z-z_0)^{N+1}f(z) = \lim_{z\to z_0}(z-z_0)^{N+1}(f(z) - a) \overset{(13)}{=} 0
        \end{align*}
        Which is the desired contradition. \qed
    \end{pb}
    \begin{pb}
        Denote the annulus in question as \(A\), the outer disc as \(D_{r_2}\) and the inner disc as \(D_{r_1}\), then for \(z_0 \in A^\circ\), we can let \(\gamma\) be a ray from \(D_{r_1}\) to \(D_{r_2}\) not intersecting \(z_0\). It follows that \(\partial D_{r_2} - \gamma + \gamma - \partial D_{r_1}\) is a closed piecewise \(C^1\) curve in \(A\) around \(z_0\), so that by Cauchy's integral formula (8):
        \begin{align}
            f(z_0) = \int_{\partial D_{r_2} - \gamma + \gamma - \partial D_{r_1}} \frac{f(w)}{w-z_0}dw = \int_{\partial D_{r_2}}\frac{f(w)}{w-z_0}dw - \int_{\partial D_{r_1}} \frac{f(w)}{w-z_0}dw
        \end{align}
        Now we can apply algebraic manipulations, since on \(\partial D_{r_2}\) we have \(\abs{w} > \abs{z_0}\) we get the following uniformly convergent power series expansion
        \begin{align}
            \frac{f(w)}{w-z_0} = \frac{f(w)}{w}\left(\frac{1}{1 - \frac{z_0}{w}}\right) = \frac{f(w)}{w} \sum_0^\infty \left(\frac{z_0}{w}\right)^k
        \end{align}
        Similarly, on \(\partial D_{r_1}\) \(\abs{z_0} > \abs{w}\), so that there is a uniformly convergent power series expansion given by
        \begin{align}
            -\frac{f(w)}{w-z_0} = \frac{f(w)}{z_0}\left(\frac{1}{1-\frac{w}{z_0}}\right) = \frac{f(w)}{z_0}\sum_0^\infty\left(\frac{w}{z_0}\right)^k
        \end{align}
        Combining (14), (15) and (16) along with uniform convergence in (15) and (16) furnishes
        \begin{align}
            2\pi if(z_0) &= \int_{\partial D_2}\frac{f(w)}{w} \sum_0^\infty \left(\frac{z_0}{w}\right)^kdw + \int_{\partial D_1}\frac{f(w)}{z_0}\sum_0^\infty\left(\frac{w}{z_0}\right)^kdw \\
            &= \sum_0^\infty z_0^k\int_{\partial D_{r_2}} \frac{f(w)}{w^{k+1}} dw + \sum_0^\infty \frac{1}{z_0^{k+1}}\int_{\partial D_{r_2}} f(w)w^kdw \\
            &= \sum_0^\infty z_0^k \int_{\partial D_{r_2}} \frac{f(w)}{w^{k+1}}dw + \sum_{-\infty}^1 z_0^k \int_{\partial D_{r_1}} \frac{f(w)}{w^{k+1}}dw
        \end{align}
        Now letting \(r \in (r_1,r_2)\), we can use the same trick with a ray  \(\gamma\) used in (14) to apply Cauchy's theorem and find that
        \begin{align}
           &\sum_0^\infty z_0^k \int_{ \partial D_r} \frac{f(w)}{w^{k+1}}dw - \sum_0^\infty z_0^k \int_{\partial D_{r_2}} \frac{f(w)}{w^{k+1}}dw = 0 \\ \tand &\sum_{-\infty}^1 z_0^k \int_{\partial D_{r_1}} \frac{f(w)}{w^{k+1}}dw - \sum_{-\infty}^1 z_0^k \int_{\partial D_r} \frac{f(w)}{w^{k+1}}dw = 0
        \end{align}
        Where application of Cauchy's theorem follows by \(\sum_0^\infty z_0^k \int_{ \partial D_r} \frac{f(w)}{w^{k+1}}dw\) analytic on \(B_{r_2}(0)\) and similarly \(\sum_{-\infty}^1 z_0^k \int_{\partial D_r} \frac{f(w)}{w^{k+1}}dw\) analytic on \(\overline{B_{r_1}(0)}^c\). Now we can substitute into (19) using (20) and (21 to obtain
        \begin{align}
            f(z_0) = \sum_{-\infty}^\infty z_0^k \frac{1}{2\pi i}\int_{\partial D_r}\frac{f(w)}{w^{k+1}}dw
        \end{align}
        \qed
    \end{pb}
    \begin{pb}
        From the chain rule we have \(f \circ (w \mapsto 1/w)\) is holomorphic on \(\mathbb{C} \setminus \set{0}\). Suppose for contradiction that \(f\circ (w \mapsto 1/w)\) has an essential singularity at zero, then by Casorati-Weierstrass
        \(f(\overline{B_\epsilon(0)}^c) = f\circ(w\mapsto\frac{1}{w})(B_\frac{1}{\epsilon}(0) \setminus \set{0})\) is dense in \(\mathbb{C}\), this gives us a contradiction since \(f(B_\epsilon(0))\) is open by the open mapping principle, hence \(f(B_\epsilon(0)) \cap f(\overline{B_\epsilon(0)}^c) \neq \emptyset\) which contradicts injectivity of \(f\). It follows that \(f\circ (w\mapsto \frac{1}{w})\) has a pole at \(0\), and hence a Laurent series. Now write
        \begin{align*}
            f \circ (w\mapsto\frac{1}{w}) = \sum_{-N}^\infty a_kz^k \implies f(w) = \sum_{-N}^\infty a_kz^{-k}
        \end{align*}
        \(f\) being entire implies that \(a_j = 0\) for \(j \geq 1\), hence \(f = \sum_0^N a_{-k}z^k\), since \(f'(z) = \sum_1^n a_{-k}z^{k-1}\) is nonvanishing, by the fundamental theorem of algebra \(f'\) must be a constant nonzero polynomial, hence \(a_{-k} = 0\) for \(k > 1\) and \(f = a_{-1}z + a_0\) as desired. \qed
    \end{pb}
    \begin{pb}
        \textbf{(i)} Let \([z_0:\cdots:z_n] \in \mathbb{P}^n\), then \((z_0,\hdots,z_n) \neq 0\), hence atleast one coordinate is nonzero, label that coordinate \(j\) to see that \([z_0:\cdots:z_n] \in U_j\) by definition. To check \(\varphi\) is a homeomorphism we can check it is a continuous and open bijection. Injectivity follows from the map \(\varphi_j\) choosing the unique equivalence class representative with \(j\)-th entry equal to one. To see surjectivity let \(z = (z_0,\hdots,1,\hdots,z_n) \in \mathbb{C}^n\), then \(z = \varphi_j([z_0:\cdots:1:\cdots:z_n])\). By definition of the quotient topology \(\varphi_j\) is continuous if and only if \(\varphi_j\circ \pi\) is, same for open. To check this simply write \(\varphi_j \circ \pi(z_0,\hdots,z_n) = \left(z_0/z_j,\hdots,z_n/z_j\right)\), since we are on \(\set{z_j \neq 0}\) this map is smooth, and its Jacobian has a copy of \(\frac{1}{z_j}1_{n\times n}\), thus this map is a submersion, since submersions are open maps this map is smooth, open and a bijection hence a homeomorphism.

        \textbf{(ii)} On \(\varphi_j(U_i \cap U_j)\) we have
        \begin{align}
            \varphi_i \varphi_j^{-1}(z_0,\hdots,1,\hdots,z_n) = \left(\frac{z_0}{z_i},\hdots,\frac{1}{z_i},\hdots,\frac{z_n}{z_i}\right)
        \end{align}
        So each of the coordinate functions is of the form \(1\) (which is not actually considered in the image by identification with \(\mathbb{C}^n\)), \(\frac{1}{z_i}\) or \(z_k/z_i\) for \(k \neq i\). From here it is immediate, since \(\frac{1}{z_i}\) is holomorphic on \(\set{z_i \neq 0} \supset \varphi_i(U_i \cap U_j)\) we get for \(\ell \neq k,i\) that \[\frac{\partial}{\partial \overline{z_\ell}} \frac{1}{z_i} = \frac{1}{z_i} \frac{\partial}{\partial \overline{z_\ell}} 1 = 0 = z_k/z_i \frac{\partial}{\partial \overline{z_\ell}} 1 = \frac{\partial}{\partial \overline{z_\ell}} z_k/z_i\]
        and by holomorphicity or \(\frac{1}{z_i}\), as well as \(z_k\)
        \begin{align*}
            \frac{\partial}{\partial \overline{z_k}} \frac{z_k}{z_i} = 1/z_i \frac{\partial}{\partial \overline{z_k}} z_k = 0 \tand \frac{\partial}{\partial \overline{z_i}} \frac{1}{z_i} = 0 = z_k \frac{\partial}{\partial \overline{z_i}} \frac{1}{z_i} = \frac{\partial}{\partial \overline{z_i}} \frac{z_k}{z_i}
        \end{align*} \qed
        
        % , it suffices to check on the basis, so let \(z \in \mathbb{C}^n\) and consider \(B_\epsilon(z) \subset \mathbb{C}^n\), then
        % \begin{align}
        %     (\varphi_j\circ \pi)^{-1}(B_\epsilon(z)) = \bigcup_{\lambda \in \mathbb{C}^\times} \lambda \cdot B_\epsilon(z) \label{eq:projopen}
        % \end{align}
        % An arbitrary element \(w \in (\varphi_j\circ \pi)^{-1}(B_\epsilon(z))\) can be written in the form \(w = \lambda\cdot(z_0 + \delta_0,\hdots,1,\hdots, z_n + \delta_n)\) with \(\abs{\mathbf{\delta}} < \epsilon \) by \eqref{eq:projopen}, also note that for some \(r > 0\) we have \(B_r(z + \delta) \subset B_\epsilon(z)\) which of course entails \(\bigcup_{\lambda \in \mathbb{C}^\times}\lambda\cdot B_r(z + \delta) \subset (\varphi_j\circ \pi)^{-1}(B_\epsilon(z))\). Now for \(v \in \mathbb{C}^{n+1}\) we can write
        % \begin{align}
        %     w + v = C\left(\frac{z_0 + \delta_0}{1 + v_j/\lambda} + \frac{v_0}{\lambda + v_j},\hdots,1,\hdots,\frac{z_n + \delta_n}{1 + v_j/\lambda} + \frac{v_n}{\lambda + v_j}\right)
        % \end{align}
        % Where \(C = \frac{\lambda^2}{\lambda + v_j}\), from this formula it is immediately apparent that \(\lim_{v \to 0} (w + v) = C(z + \delta)\), and since the \(j\)-th entry is fixed at \(C\) for all \(v\), we have for sufficiently small \(v\) that \(w + v \in C\cdot B_r(z + \delta)\). Finally to check that the map is open, note that \(\varphi_j\circ\pi(z) = q(z/z_j)\), where \(q\) denotes the projection \(\mathbb{C}^{n+1} \to \mathbb{C}\), we once again check on basis elements, so consider \(z = (z_0,\hdots,z_n) \in \mathbb{C}^{n+1}\) with \(z_j \neq 0\) and take \(B_\epsilon(z)\). Then if \(w \in \varphi_j\circ \pi(B_\epsilon(z))\), we can write \(w = \frac{z + v}{(z + v)_j}\) for \(\abs{v} < \epsilon\), then for any \(u \in B_{\frac{\epsilon - \abs{v}}{\abs{(z-v)_j}}}(w)\)
    \end{pb}
    \begin{pb}
        I will explain it for \(\pi_N\), \(\pi_S\) is clearly the same by symmetry (however we flip the convention \((x,y,0) \leftrightsquigarrow x - iy\) to ensure holomorphicity). Stereographic projection geometrically is embedding \(S^2 \hookrightarrow \mathbb{R}^3\) (i.e. defining \(S^2\) as the equation in question), where we identify \(\mathbb{R}^2\) with \(\mathbb{C}\) in the obvious way \((x,y,0) \leftrightsquigarrow x + iy\), then we associate to each point \((x_1,x_2,x_3)\) in \(S^2 \setminus N\) the unique point where the  \(\mathbb{C} \leftrightsquigarrow \mathbb{R}^2 \times \set{0}\) such that the line through \(N\) and \((x_1,x_2,x_3)\) intersects \(\mathbb{C}\). The equation for \(\pi_N\) is simply parameterizing in terms of \((x_1,x_2,x_3)\) the point where the line meets \(\mathbb{C}\). Concretely we have the following parameterization
        \begin{align}
            \ell(t) = t(x_1,x_2,x_3 - 1) + (0,0,1) \label{eq:stereo}
        \end{align}
         We get the point of projection by solving for \(t_0\) such that \(\ell(t_0) \in \mathbb{R}^2 \times\set{0}\), this is \(t_0 = \frac{1}{1-x_3}\), now plugging this into \eqref{eq:stereo} gives
         \begin{align}
            \ell(t_0) = \left(\frac{x_1}{1-x_3},\frac{x_2}{1-x_3},0\right) \leftrightsquigarrow \frac{x_1 + ix_2}{1-x_3} \overset{\text{def}}{=} \pi_N(x_1,x_2,x_3)
         \end{align}
         From this geometric realization we can also compute \(\pi_S^{-1}\) as being the unique point on \(S^2 \setminus S\) which intersects the line through \((x,-y,0) \leftrightsquigarrow x+iy\), this gives the parameterized line
         \begin{align}
            \ell'(t) = t(x,-y,1) + (0,0,-1)
         \end{align}
         so solving for \(t_0\) where \(\ell'(t_0) \in S^2\) we get
         \begin{align}
            t_0^2(x^2 + y^2 + 1) - 2t_0 = 0 \label{eq:inversester}
         \end{align}
         Where the zeroes of \eqref{eq:inversester} correspond to \(S\) and
         \begin{align}
            \pi_S^{-1}(x+iy) = \left(\frac{2x}{x^2 + y^2 + 1},\frac{-2y}{x^2+y^2+1},\frac{1-x^2-y^2}{x^2+y^2+1}\right)
         \end{align}
         Now we can apply \(\pi_N\) to get
         \begin{align}
            \pi_N\pi_S^{-1}(x+iy) = \frac{\frac{2(x-iy)}{x^2+y^2+1}}{1-\frac{1-x^2-y^2}{1+x^2+y^2}} = \frac{x-iy}{x^2+y^2} = \frac{\overline{z}}{\abs{z}^2} = \frac{1}{z}
         \end{align}
         Which is indeed holomorphic on \(\mathbb{C}\setminus \set{0}\). \qed
    \end{pb}
    \begin{pb}
        Note that the reason we only need to check on the charts \(\mathbb{P}^1 \setminus \set{[0:1]}\), and \(S^2 \setminus N\) are that away from \(S\) we have that \(f\) can be written in charts as \(\phi_{\tilde{S}}\circ f\circ\pi_S^{-1} = 1_{\mathbb{C}^\times}\). To begin with we compute \(\phi_{\tilde{N}}\circ f \circ \pi_N^{-1}\) we write \(z = x+iy\)
        \begin{align}
            \phi_{\tilde{N}}\circ f \circ \pi_N^{-1}(x+iy) &= \phi_{\tilde{N}}\circ\phi_{\tilde{S}}^{-1}\circ\pi_S\left(\frac{2x}{x^2+y^2+1},\frac{2y}{x^2+y^2+1},\frac{-2}{1+x^2+y^2}+1\right) \\
            &= \phi_{\tilde{N}}\circ\phi_{\tilde{S}}^{-1}(1/z) = \phi_{\tilde{N}} [1/z:1] = z
        \end{align}
        This of course extends to the identity map from \(\mathbb{C} \to \mathbb{C}\), so if we take \(f(S) = [1:0]\), the extended map on \(S^2 \setminus N \to \mathbb{P}^1 \setminus\set{[1:0]}\) charts is just the identity map \(z \mapsto z\) which is holomorphic, so that every point in \(S^2\) is contained in an open set with (the extension of) \(f\) holomorphic in coordinates on that open set. Moreover, by the holomorphic inverse function theorem (the extension of) \(f\) is biholomorphic, since the derivative in either pair of charts is just given by \(1\) which is non-vanishing. \qed
    \end{pb}
\end{document}