\documentclass[10.5pt]{article}
\usepackage{amsmath, amsfonts, amssymb,amsthm}
\usepackage[includeheadfoot]{geometry} % For page dimensions
\usepackage{fancyhdr}
\usepackage{enumerate} % For custom lists
\usepackage{tikz-cd}
\usepackage{graphicx}

\fancyhf{}
\lhead{MAT1300 hw6}
\rhead{Tighe McAsey - 1008309420}
\pagestyle{fancy}

% Page dimensions
\geometry{a4paper, margin=1in}

\theoremstyle{definition}
\newtheorem{pb}{}
\usepackage{tikz-cd, stackengine}

% Commands:

\newcommand{\set}[1]{\{#1\}}
\newcommand{\gen}[1]{\langle#1\rangle}
\newcommand{\abs}[1]{\lvert#1\rvert}
\newcommand{\norm}[1]{\lvert\lvert#1\rvert\rvert}
\newcommand{\tand}{\text{ and }}
\newcommand{\tor}{\text{ or }}
\newcommand{\pd}{\frac{\partial}{\partial x_j}}
\setcounter{MaxMatrixCols}{20}

\begin{document}
    \begin{pb}
        We use from the notes the existence of the oriented intersection number, defined when \(f: M \to N\), and \(Z \subset N\) is a submanifold with \(f = f_0 \sim f_1\) and \(f_1 \pitchfork Z\), \(I(f,Z) = \sum_{p \in f_1^{-1}(Z)} \text{orientation}(p)\), which is congruent to \(I_2(f,Z)\) mod 2. 
        
        Now since \(n := \dim M = \dim N\), we can define \(\deg f = I(f,\set{p})\) for \(p \in M\), we need to show that this is well defined for arbitrary \(p\), it reduces to \(\deg_2 f\) since \(I(f,\set{p}) = I_2(f,\set{p}) \text{mod }2\). Since the oriented intersection number is a homotopy invariant, we can take a point \(p\), and assume \(f \pitchfork \set{p}\), I will show that \(\deg f\) is locally constant near \(p\), and hence since \(N\) is connected is constant on \(N\).
        Since \(f \pitchfork \set{p}\), we have that \(d_qf\) is an isomorphism for \(q \in f^{-1}\set{p}\), so by the inverse function theorem \(f\) is a local diffeomorphism at each \(q \in f^{-1}(p)\), moreover \(f^{-1}\set{p}\) is closed in \(M\), hence compact, and since each has an open neighborhood mapping diffeomorphically to an open neighborhood of \(p\), this implies that there must be finitely many such points. Now taking \(f^{-1}\set{p} = \set{q_1,\hdots,q_r}\), we can take the intersection of the neighborhoods giving a local diffeomorphism to get \(U \supset \set{p}\), such that \(f^{-1}(U) = \bigsqcup_1^r V_i\), and \(f\vert_{V_i}: V_i \overset{\cong}{\longrightarrow} U\). This proves that the number of points in the preimage is constant, to finish the proof, fix orientations of \(M \tand N\) with corresponding sections \(s_M, s_N\), assuming WLOG that \(s_N > 0\). it will suffice to show that \((\det d_{q_i}f) s_M(q_i) > 0\) implies \((\det d_x f)s_M(x) > 0\) for all \(x \in V_i\) (and the same for if \((\det d_{q_i}f)s_M < 0\), but the proof is the same so we just prove the first case) since this shows that the orientation of points is locally constant on each \(V_i\). Now suppose that \((\det d_{q_i}f) s_M(q_i)\) is positive, then since \(f\vert_{V_i}\) is a diffeomorphism, we get \(df\vert_{V_i}: TV_i \overset{\cong}{\longrightarrow} TU\) which induces \(\det df\vert_{V_i}:\Lambda^n TV_i \overset{\cong}{\longrightarrow} \Lambda^n TU\), so in particular \(\det df\vert_{V_i}\) is a linear isomorphism (non-vanishing) on each fiber and since it is smooth, and we have \((\det df\vert_{V_i})s_M\) positive at some point in \(V_i\) which is connected by connectedness of \(M\), we find that it is everywhere positive by the IVT as desired. \qed
    \end{pb}
    \begin{pb}
        We first show that \(NM\) is the trivial bundle. Consider the embedding \(\varphi: \mathbb{R}^d \hookrightarrow S^d\), via the stereographic projection, and denote \(\varphi(\mathbb{R}^d)^c = \set{\infty}\), then under this embedding \(M\) is taken to a compact subset of \(S^d\), so we can apply the Jordan-Brouwer separation theorem to \(\varphi(M) \subset S^d\), this furnishes open sets \(U, V \subset S^d\) with \(\overline{U}, \overline{V}\) compact connected submanifolds of \(S^d\) so that \(\partial \overline{U} = \partial \overline{V} = \varphi(M)\), suppose that \(\infty \not \in U\) (otherwise choose \(V\)), then \(X:= \varphi^{-1}(\overline{U})\) is a compact connected submanifold of \(\mathbb{R}^d\) with \(\partial X = M\), now to use the orthogonal compliment of bundles we fix a Riemmannian metric on \(X\), or simply notice that \(X\) inherits a metric from \(\mathbb{R}^d\). From the construction of collars, there exists a smooth function \(\chi: X \to [0,\infty)\) with the property that \(\chi^{-1}(0) = M\), and for each \(p \in M\) there is some \(v \in T_pX \setminus T_pM\) so that \(d_p\chi(v) \neq 0\), using the direct sum decomposition \(T_pX = T_p M \oplus (T_pM)^\perp\) we can write \(v = v_M + v_\perp\), then since \(\chi\vert_M = 0\), we get \(d_p\chi\vert_M = 0\), and hence \(d_p\chi v_M = 0\), which gives us \(d_p\chi v_\perp \neq 0\), and therefore \(d_p\chi: (T_pM)^\perp \to T_0[0,\infty)\) is nonzero and hence an isomorphism. Since \(d\chi: (TM)^\perp \to T[0,\infty)\) is a smooth bundle map, with image \(T_0[0,\infty)\), we get a smooth real valued map \(f\) which is an isomorphism at each point, then \(F: (TM)^\perp \to M \times \mathbb{R}\) via \((x,v) \mapsto (x,f(x,v))\) is a smooth bundle isomorphism, we conclude that \(NM \cong (TM)^\perp \cong M \times \mathbb{R}\) is trivial.

        Now since \(T \mathbb{R}^d\vert_M\) is orientable, we have \(\Lambda^n T \mathbb{R}^d\vert_M \cong M \times \mathbb{R}\), and from the proof involving Jordan-Brouwer separation we get \(\Lambda NM \cong NM \cong M \times \mathbb{R}\), now we can use the direct sum decomposition \(T \mathbb{R}^d \cong TM \oplus NM\), which induces \(\Lambda^d T \mathbb{R}^d\vert_M \cong \Lambda^{d-1}TM \oplus \Lambda NM\), the following diagram of isomorphisms shows that \(\Lambda^{d-1}TM\) is trivial, and hence \(TM\) is orientable.
        \begin{equation*}
            \begin{tikzcd}
                \Lambda^{d-1}TM \arrow[dr, dashed, "\cong"] \arrow[r,"\cong"] & \Lambda^{d-1}TM \otimes M \times \mathbb{R} \arrow[r,"\cong"] &\Lambda^{d-1}TM \otimes \Lambda NM \arrow[d,"\cong"] \\
                &M \times \mathbb{R} &\Lambda^d T \mathbb{R}^d\vert_M \arrow[l, "\cong"]
            \end{tikzcd}
        \end{equation*} \qed
    \end{pb}
    \begin{pb}
        \textbf{Lemma.} If \(G\) a (finite) discrete group acts on an orientable manifold \(M\) such that the action is smooth, free and proper, such that for each \(g \in G\) we have \(\det (d g) > 0\) on \(M\), then there is an induced orientation on \(M/G\).
        \begin{proof}
            For convenience, take the section \(s: M \to \Lambda^n TM\) so that \(s > 0\). Let \(q: M \to M/G\) be the quotient map induced by the group action, and let \(\set{V_\alpha}_{\alpha \in \mathcal{A}}\) be an open cover for \(M\) with \(q^{-1}(V_\alpha) = \bigsqcup_1^r U_\alpha^i\) and \(q\vert_{U_\alpha^i}: U_\alpha^i \overset{\cong}{\longrightarrow} V_\alpha\). Now let \(\set{\eta_\alpha}_\mathcal{A}\) be a partition of unity subordinate to the \(V_\alpha\), we consider the following diagram, and local invertibility of \(q\) to define a section \(V_\alpha \to \Lambda^n TV_\alpha\) which is either everywhere positive or negative, since \(q\vert_{U_\alpha^i}\) is a diffeomorphism, it induces an isomorphism of tangent bundles \(\det dq\vert_{U_\alpha^1}\), since this is a smooth map which is everywhere non-zero, it is in particular everywhere positive or negative.
            \begin{equation*}
                \begin{tikzcd}
                    U_\alpha^1 \arrow[d,"q\vert_{U_\alpha^1}"] \arrow[r,"s"] & \Lambda^nTU \arrow[d, "\det(dq\vert_{U_\alpha^1})"] \\
                    V_\alpha & \Lambda^n TV_\alpha
                \end{tikzcd}
            \end{equation*}
            So that we get the section \(\overline{s}: x \mapsto \sum_\alpha \eta_\alpha \det dq\vert_{U_\alpha^1}(s(q\vert_{U_\alpha^1}^{-1}(x)))\), to check it is everywhere non-zero suppose that \(\eta_{\alpha_1}(x),\hdots,\eta_{\alpha_s}(x) > 0\), notice that for each \(j\) letting \(y_k = q\vert_{U^1_{\alpha_j}}^{-1}(x)\) we have on some neighborhood of \(y_j\)
            \[q\vert_{U^1_{\alpha_j}} = q\vert_{U_{\alpha_1}^1}\left(q\vert^{-1}_{U^1_{\alpha_1}}q\vert_{U^1_{\alpha_j}}\right) = q\vert_{U_{\alpha_1}^1}g_j\]
            for some \(g_j \in G\), this gives us that (using functoriality of \(\det\))
            \begin{align*}
                \det d_{y_j}q\vert_{U_{\alpha_j}^1} = \det d_{y_j}q\vert_{U_{\alpha_1}^1}g_j = \left(\det d_{y_1}q\vert_{U_{\alpha_1}^1}\right)\left(\det d_{y_j}g_j\right)
            \end{align*}
            Since \(\det dg_j > 0\), this implies that each term of the sum \(\sum_\alpha \eta_\alpha \det dq\vert_{U_\alpha^1}(s(q\vert_{U_\alpha^1}^{-1}(x)))\) is a positive multiple of \(\det dq\vert_{U_{\alpha_1}^1}(s(q\vert_{U_\alpha^1}^{-1}(x)))\), then since each term is nonzero and has the same sign this suffices to show nowhere vanishing at the point \(x\), and since \(x\) was arbitrary, the section is nowhere vanishing.
        \end{proof}

        Denote \(j:S^d \to S^d\) as the antipodal map. When \(d\) is odd, we have an isotopy \(1_{S^d} \sim j\) via \(H((z_1,\hdots,z_{\frac{d+1}{2}}),t) = (e^{i\pi t}z_1,\hdots,e^{i\pi t}z_{\frac{d+1}{2}})\), denoting \(H(x,t)\) as \(j_t(x)\) we find each induced \(\det dj_t: \Lambda^d TS^d \to \Lambda^n TS^d\) is non-vanishing by virtue of being an embedding, moreover by IVT and smoothness in \(t\), we find that \(\det dj_t > 0\) on \(M\) for all \(t\), and hence the antipodal map satisfies the conditions of the lemma, applying the lemma we find \(\mathbb{RP}^d = S^d/(\mathbb{Z}/ 2 \mathbb{Z})\) has an induced orientation for odd \(n\).

        In the case that \(d\) is even, suppose for the sake of contradiction that \(\mathbb{RP}^d\) is orientable. We first check that the antipodal map \(j\) is indeed orientation reversing on \(S^d\). Consider \(S^d\) as embedded in \(\mathbb{R}^{d+1}\) via the standard embedding, we get the decomposition \(T \mathbb{R}^{d+1}\vert_{S^d} = TS^d \oplus (TS^d)^\perp\), since \(T \mathbb{R}^{d+1}\vert_{S^d}\), letting \(s\) be a representative of an orientation for \(S^d\)  we can also take nonvanishing the outward normal section \(n: S^d \to (S^d)^\perp\) via \(n(x) = x\), this gives an orientation on \((S^d)^\perp\) since we have the canonical isomorphism \(\Lambda (S^d)^\perp \cong (S^d)^\perp\). We have that \(j\) extends to the map \(\hat{j}: x \mapsto -x\) on \(\mathbb{R}^{d+1}\), and using \(\mathbb{R}\)- coordinates, we can simply compute \(\det dj = -1^{d+1} = -1\), then the isomorphism of tangent spaces gives us
        \begin{align*}
            \Lambda^{d+1} T \mathbb{R}^{d+1}\vert_{S^d} \cong \Lambda^{d+1} TS^d \oplus (TS^d)^\perp \cong \Lambda^d TS^d \otimes \Lambda (TS^d)^\perp
        \end{align*}
        So that we can identify via this correspondence a orientation \(t\) of \(\mathbb{R}^{d+1}\), writing fiberwise \(t = s \otimes n\), we get that 
        \[(\det dj) t (x) = (\det dj)(s(x))\otimes (\det dj)n(x)\]
        Then \((\det dj)n(x) = dj n(x) = -n(x) = n(-x)\) (note this makes sense because \(j(x) = -x\), so the fibers are consistent), so this can be rewritten as
        \begin{align*}
            -t(-x) = (\det dj)(s(x))\otimes n(-x)
        \end{align*}
        wich tells us that \((\det dj)(s(x)) = -s(-x)\), so that \((\det dj) s = -s\) which suffices to show the antipodal map is orientation reversing (if you don't accept that the family of orientations on \(S^d\) for \(d\) odd is smooth, then take this argument to show the antipodal map is orientation preserving for odd \(d\), since \(\det d\hat{j} = (-1)^{d+1} = 1\) when \(d\) is odd). Now let \(s\) be an orientation on \(\mathbb{RP}^d\), from this we get the section \((\det dq)^{-1}s: S^d \to \Lambda^d S^d\), which is non-vanishing since \(s\) is nonvanishing and \(q\) is a local diffeomorphism, moreover we have \(q\circ j = q\) and also note since \((dj)^2 = 1\) we get \(\det dj= (\det dj)^{-1}\), so that 
        \begin{align*}
            (\det dj)(\det dq)^{-1} = (\det dj)^{-1}(\det dq)^{-1} = (\det dq \cdot dj)^{-1} = (\det d(q \circ j))^{-1} = (\det dq)^{-1}
        \end{align*}
        this implies that \(j\) preserves the orientation of \((\det dq)^{-1}s\), contradicting our earlier proof that \(j\) is orientation reversing on an arbitrary orientation of \(S^d\). \qed
    \end{pb}
\end{document}