\documentclass[10.5pt]{article}
\usepackage{amsmath, amsfonts, amssymb,amsthm}
\usepackage[includeheadfoot]{geometry} % For page dimensions
\usepackage{fancyhdr}
\usepackage{enumerate} % For custom lists
\usepackage{tikz-cd}
\usepackage{graphicx}

\fancyhf{}
\lhead{MAT1300 hw8}
\rhead{Tighe McAsey - 1008309420}
\pagestyle{fancy}

% Page dimensions
\geometry{a4paper, margin=1in}

\theoremstyle{definition}
\newtheorem{pb}{}
\usepackage{tikz-cd, stackengine}

% Commands:

\newcommand{\set}[1]{\{#1\}}
\newcommand{\gen}[1]{\langle#1\rangle}
\newcommand{\abs}[1]{\left\vert#1\right\vert}
\newcommand{\norm}[1]{\lvert\lvert#1\rvert\rvert}
\newcommand{\tand}{\text{ and }}
\newcommand{\tor}{\text{ or }}
\newcommand{\pd}{\frac{\partial}{\partial x_j}}
\setcounter{MaxMatrixCols}{20}

\begin{document}
    \begin{pb}
        If compactly supported cohomology were homotopy invariant, then we would require \(H^*_c(\mathbb{R}^n) \cong H_c^*(\set{\text{pt.}})\) for all \(n \in \mathbb{Z}_{\geq 0}\) since \(\mathbb{R}^n \simeq \set{\text{pt.}}\) for all \(n\). To further explicate this, Consider for each \(n\), the homotopy
        \begin{align*}
            H: \mathbb{R}^n \times [0,1] &\to \mathbb{R}^n \\
            (x,t) &\mapsto x(1-t)
        \end{align*}
        interpolates the maps \(1_{\mathbb{R}^n}\) and the zero map. Then we could take \(g: \set{0} \hookrightarrow \mathbb{R}^n\), and \(f: \mathbb{R}^n \to \set{0}\), so that \(gf = 1_{\set{0}}\), and \(fg\) is the zero map which we already showed is homotopy equivalent to \(1_{\mathbb{R}^n}\).

        Thus if compactly supported cohomology were a homotopy invariant we would have
        \begin{align*}
            H_c^* (\mathbb{R}^n) \cong H_c^* (\set{0})
        \end{align*}
        choosing \(n = 1 = *\), we get \(\mathbb{R} \cong 0\) (as \(\mathbb{R}\)-vector spaces) by the poincare lemma. This is a contradiction. \qed
    \end{pb}
    \begin{pb}
        \textbf{(a)} Let \(\eta_U\), \(\eta_V\) be a partition of unity subordinate to \(U,V\), then we can define the following maps making the sequence short exact:
        \begin{equation*}
            \begin{tikzcd}
                0 \arrow[r] & \Omega_c^*(U\cap V) \arrow[r] &\Omega_c^*(U)\oplus\Omega_c^*(V) \arrow[r] &\Omega_c^*(M) \arrow[r] &0 \\
                &\omega \arrow[r] &(\eta_V\cdot \omega, \eta_U\cdot \omega) \\
                & & (\omega,\nu) \arrow[r] & (\eta_U\cdot\omega,\eta_V\cdot\nu)
            \end{tikzcd}
        \end{equation*}
        The first map has \(\text{supp} (\eta_V\cdot\omega) \subset \text{supp}(\omega) \supset \text{supp} (\eta_U\cdot\omega)\), so there are no issues with the compact support, similarly for the second map \(\text{supp}(\eta_U\cdot\omega + \eta_V\cdot\nu) \subset \text{supp}(\omega) \cup \text{supp}(\nu)\) which is a union of two compact sets hence compact.

        To see the first map is an injection let \(\omega \in \Omega_c^p(U\cap V)\) (for some \(p\)) and suppose that \((\eta_V\cdot \omega,\eta_U\cdot \omega) \equiv 0\), then \(\eta_V\cdot \omega \equiv 0\) on \(U\) and \(\eta_U\cdot \omega \equiv 0\) on \(V\), this of course implies \((\eta_V\cdot \omega)\vert_{U\cap V} \equiv 0\) and \((\eta_U\cdot \omega)\vert_{U\cap V} \equiv 0\) on \(U\cap V\), so the following easy computation shows injectivity,
        \begin{align*}
            (\eta_V\cdot \omega)\vert_{U\cap V} + (\eta_U\cdot \omega)\vert_{U\cap V} = \eta_V\vert_{U\cap V}\cdot \omega + \eta_U\vert_{U\cap V}\cdot \omega = (\eta_U\vert_{U\cap V} + \eta_V\vert_{U\cap V})\cdot\omega = \omega
        \end{align*}
        Now checking surjectivity of the second map, Let \(\omega \in \Omega_c^p(M)\) for some \(p\), then we have \(\omega\vert_U \in \Omega_c^p(U)\) and \(-\omega\vert_V \in \Omega_c^p(V)\), then I claim that the image of \((\omega\vert_U,-\omega\vert_V) = \eta_U\cdot\omega\vert_U + \eta_V\cdot\omega\vert_V = \omega\). To check this, it suffices to check equivalence pointwise, so we can simply check on each of the sets \(U \cap V^c\), \(V \cap U^c\) and \(U \cap V\), to see it on \(U \cap V^c\) we have \(\eta_V = 0\), and \(\eta_U = 1\) so that \(\eta_U\cdot\omega\vert_U + \eta_V\cdot\omega\vert_V = \omega\vert_U\) on this set, but since \(U \cap V^c \subset U\), this is the same thing as \(\omega\) here. Checking on \(V \cap U^c\) is similar, finally on \(U \cap V\), we have
        \begin{align*}
            \eta_U\cdot\omega\vert_U + \eta_V\cdot\omega\vert_V = (\eta_U + \eta_V) \omega\vert_{U \cap V} = \omega\vert_{U\cap V}
        \end{align*}
        which is of course just \(\omega\) on \(U\cap V\), this shows surjectivity.
        
        Finally, we need to check that \(\ker ((\omega, \nu) \mapsto \eta_U\cdot \omega - \eta_V\cdot\omega) = \text{Im}(\omega \mapsto (\eta_V\cdot \omega, \eta_U\cdot\omega))\), checking the image is a subset of the kernel, is straightforward since composing both maps we get
        \begin{align*}
            \omega \mapsto \eta_U\cdot\eta_V(\omega - \omega) = 0
        \end{align*}
        Now to check that all elements of the kernel are of this form, suppose \((\omega,\nu) \mapsto 0\), then wherever \(\eta_V = 0\), we have \(\eta_U\cdot \omega - \eta_V\cdot \nu = \omega\), but since we are assuming this is zero we must have \(\text{supp}\,\omega \subset \text{supp}\,\eta_V\), the same argument shows that \(\text{supp}\,\nu \subset \text{supp}\,\eta_U\). Now we define the following form \(\alpha\) on \(U \cap V\)
        \begin{align*}
            \alpha = \begin{cases}
                \eta_V^{-1}\cdot\omega & \eta_U,\eta_V > 0 \\
                \omega & \eta_U = 0 \\
                \nu & \eta_V = 0
            \end{cases}
        \end{align*}
        Then \(\text{supp}\,\alpha \subset \text{supp}\, \omega \cup \text{supp}\,\eta\) is compact, to verify that \(\alpha\) is indeed smooth, note that away from \(\eta_V = 0\), this is clear, to see that its smooth near \(\eta_V = 0\), we can use the kernel condition to give us \(\eta_U\cdot \omega = \eta_V\cdot\nu\), when \(\eta_U,\eta_V > 0\) this gives us \(\eta_V^{-1}\omega = \eta_U^{-1}\nu\), but this shows that we are smooth near \(\eta_V = 0\). Finally, we compute
        \begin{align*}
            \eta_V \cdot \alpha &= \begin{cases}
                \omega & \eta_V > 0\\
                0 &\eta_V = 0
            \end{cases} = \omega
        \end{align*}
        since \(\text{supp}\,\omega \subset \text{supp}\,\eta_V\), similarly
        \begin{align*}
            \eta_U \alpha &= \begin{cases}
                \eta_U \eta_V^{-1}\cdot \omega & \eta_V, \eta_U > 0 \\
                0 & \eta_U = 0 \\
                \nu & \eta_V = 0
            \end{cases} = \begin{cases}
                \eta_V \eta_V^{-1}\cdot \nu & \eta_V, \eta_U > 0 \\ 0 & \eta_U = 0 \\ \nu & \eta_V = 0
            \end{cases} \\ &= \nu
        \end{align*}
        Once again, the last equality follows from \(\text{supp}\,\nu \subset \text{supp}\,\eta_U\). This suffices to show exactness of the previous sequence. \qed

        \textbf{(b)} Since Theorem 1.4, Lecture 22 in the notes holds for any short exact sequence of cochain complexes, it in particular holds for
        \begin{equation*}
            \begin{tikzcd}
                0 \arrow[r] & \Omega_c^*(U\cap V) \arrow[r] &\Omega_c^*(U)\oplus\Omega_c^*(V) \arrow[r] &\Omega_c^*(M) \arrow[r] &0
            \end{tikzcd}
        \end{equation*}
        The induced long exact sequence is exactly the desired one. \qed
    \end{pb}
    \begin{pb}
        I will denote \(\rho\) to be the covering map \(\mathbb{R}^2 \to \mathbb{T}^2\).

        \textbf{(a)} Since \(\mathbb{T}^2 \hookrightarrow \mathbb{R}^3\), we get by problem 2 of homework 6 it is orientable, since its a codimension 1 submanifold of \(\mathbb{R}^3\), since its orientable it has a trivial determinant bundle so we can just take the section \(t(x) = (x,1)\). Now we can choose an orientation on \(\mathbb{R}^2\) using the section \(s:x \mapsto (x,s'(x))\) where \(s'(x) = \pi(\det d\rho(1))\) (here \(\pi\) denotes the map throwing out the base point), \(s\) is smooth from construction, and is nonvanishing since \(\rho\) is a local diffeomorphism, which implies that \(\det d\rho\) is an isomorphism on each fiber, but now by construction we have \(\rho\) orientation preserving, since \(\pi\det d\rho (s) = (\pi\det d\rho(1))^2 > 0\).

        Now that we have shown \(\rho\) is orientation preserving, we can use the fact that \(A: \mathbb{R}^2 \to \mathbb{R}^2\) is orientation preserving when \(\det A = 1\) and orientation reversing when \(\det A = -1\) alongside the following diagram, to conclude that the top path is orientation reversing when \(\det A = -1\), and orientation preserving when \(\det A = 1\) (for extra justification use functoriality of determinant and the fact that \(\det \rho > 0\)), by commutativity of the diagram, this means that \(\phi_A\) is orientation preserving iff \(\det A = 1\).
        \begin{equation*}
            \begin{tikzcd}
                \mathbb{R}^2 \arrow[d,"\rho"] \arrow[r,"A"] & \mathbb{R}^2 \arrow[d,"\rho"] \\
                \mathbb{T}^2 \arrow[r,"\phi_A"] & \mathbb{T}^2
            \end{tikzcd}
        \end{equation*} \qed

        \textbf{(b)} In part (a), we verified that \(SL_2(\mathbb{Z}) \to \Gamma_{1,0}\) via the map, it remains to check its a homomorphism. The following diagram commutes by definition of \(\phi_{X}\) being induced by \(X\) for each \(X \in SL_2(\mathbb{Z})\)
        \begin{equation*}
            \begin{tikzcd}
                \mathbb{R}^2 \arrow[d,"\rho"] \arrow[r,"A"] & \mathbb{R}^2 \arrow[d,"\rho"] \arrow[r,"B"] & \mathbb{R}^2 \arrow[d,"\rho"] \\
                \mathbb{T}^2 \arrow[r,"\phi_A"] & \mathbb{T}^2 \arrow[r,"\phi_B"] & \mathbb{T}^2
            \end{tikzcd} \rightsquigarrow \begin{tikzcd}
                \mathbb{R}^2 \arrow[d,"\rho"] \arrow[r,"BA"] & \mathbb{R}^2 \arrow[d,"\rho"] \\
                \mathbb{T}^2 \arrow[r,"\phi_B\phi_A"] & \mathbb{T}^2
            \end{tikzcd}
        \end{equation*}
        But by definition the induced map in the right diagram is \(\phi_{BA}\), so that \(\phi_B\phi_A = \phi_{BA}\), this is exactly the homomorphism property.

        \textbf{(c)} If \(A = B\), then \(\phi_A = \phi_B\), so clearly they are isotopic and we are done. Conversely, assume \(\phi_A \simeq \phi_B\), we want to write down how \(\phi_A\) acts on cohomology.

        We start by defining \(1\)-forms on \(\mathbb{T}^2\), consider the standard one forms \(dx,dy\) on \(\mathbb{R}^2\), we get well defined forms from the restriction (it will suffice to check just for \(dx\)). To check this, take open sets \(U_1,\hdots,U_s \subset \mathbb{R}^2\) with \(\bigcup_1^s \rho(U_j) = \mathbb{T}^2\), and \(\rho \vert_{U_j}\) is a diffeomorphism to its image, denote the inverse as \(\rho^{-1}_j\). The existence of such sets follows from \(\rho\) being a covering map and we can guarantee finitely many since \(\mathbb{T}^2\) is compact. Now let \(\eta_j\) be a partition of unity subordinate to the \(U_j\), then we can define \(dx' = \left(\sum_1^s \eta_j\cdot\rho_j^{-1}\right)^*dx\), then since \(\rho\) is a local diffeomorphism any tangent vector in \(T \mathbb{T}^2\) is locally of the form \(d\rho(v)\) for \(v \in T \mathbb{R}^2\), then
        \begin{align*}
            dx'(d\rho(v)) = \left(\sum_1^s \eta_j\cdot\rho_j^{-1}\right)^*dx d\rho(v) = (dx) (d \sum_1^s \eta_j\cdot\rho_j^{-1})(d\rho)(v) = (dx) \sum_1^s (d\eta_j)(d\rho_j^{-1})(d\rho)(v)
        \end{align*}
        Then since \(dx\) is translation invariant, and \((d\rho_j^{-1})(d\rho) = d(x \mapsto x + g)\) for some \(g \in \mathbb{Z}^2\), there are no issues with what point we are taking \(dx\) at, so that the right hand side further simplifies to \((dx) d\left(\sum_1^s \eta_j\right)(v) = dx(v)\), so the form descends to \(\mathbb{T}^2\). Writing \(f = \sum_1^s \eta_j\cdot\rho_j^{-1}\), the computation also verifies that \(\rho^*f^* dx = dx\), and the same will hold for \(dy\).  In order to write down how \([\phi^*_A]\) acts on cohomology, we will first determine how \(\phi^*_A\) acts on \(dx',dy'\), to do so first let \(\alpha \partial_x + \beta \partial_y = v \in T \mathbb{R}^2\)
        \begin{align*}
            A^*dx(v) &= (dx) (dA) (v) = dx ((a \alpha + b \beta)\partial_x + (c \alpha + d \beta)\partial_y) = a \alpha + b \beta \\
            A^*dy(v) &= (dy)(dA)(v) = dy((a \alpha + b \beta)\partial_x + (c \alpha + d \beta)\partial_y) = c \alpha + d \beta
        \end{align*}
        So that \(A^*dx = adx + bdy\) and \(A^*dy = cdx + ddy\). Now, we can use this to compute \(\phi_A^* dx'\) and \(\phi_A^* dy'\), by noticing that \(\phi_A \rho = \rho A\), so that \(\rho^* \phi_A^* = A^*\rho^*\) as well as noticing since \(\rho f =1_{\mathbb{T}^2}\) we get \(f^*\rho^* = 1_{\mathbb{T}^2}^*\), from here we can compute
        \begin{align*}
            \phi_A^*dx' &= 1_{\mathbb{T}^2}^* \phi_A^*dx' = f^*\rho^*\phi_A^*dx' = f^*A^*\rho^*dx' = f^* A^* \rho^* f^* dx = f^* A^* dx = f^* (adx + bdy)  \\ 
            &= a\circ f \cdot f^*dx + b\circ f \cdot f^*dy = a dx' + bdy' \\
            \phi_A^*dy' &= 1_{\mathbb{T}^2}^* \phi_A^*dy' = f^*\rho^*\phi_A^*dy' = f^*A^*\rho^*dy' = f^* A^* \rho^* f^* dy = f^* A^* dy = f^* (cdx + ddy)  \\ 
            &= c\circ f \cdot f^*dx + d\circ f \cdot f^*dy = c dx' + ddy'
        \end{align*}
        Now assuming that \(\set{[dx'],[dy']}\) is a basis for \(H^1(\mathbb{T}^2)\) what we just computed shows that \[[\phi_A^*]^1: H^1(\mathbb{T}^2) \overset{A^{\text{T}}}{\longrightarrow} H^1(\mathbb{T}^2)\]
        with respect to this basis, and therefore if \(\phi_A \simeq \phi_B\), then \(A^T = [\phi_A^*]^1 = [\phi_B^*]^1 = B^T\), but this of course implies that \(A = B\).

        Now, checking that indeed \([dx'],[dy']\) is a basis, we use that \(S^1 \cong \mathbb{R}/ \mathbb{Z}\), and that \(d \theta\) is given by \(dx\) using the analogous 1-dimensional restriction for what we did on the Torus, we can write the pull-back here as \(\hat{f}\), and the quotient map as \(q\), using the same computations as for the Torus this gives us \(\rho^* \hat{f}^* dx = dx\). To see that \(H^1(S^1) = [d\theta]\), recall that \(H^1(S^1) \cong \mathbb{Z}\), and using stokes theorem we know that if \(d \theta\) were exact it would have zero integral since the circle has no boundary. Now it is straightforward to compute
        \begin{align*}
            \int_{S^1} d\theta = \int_0^1 \rho^* d\theta = \int_0^1 \rho^* \hat{f}^* dx = \int_0^1 dx = 1
        \end{align*}
        So that indeed \(H^1(S^1) = [d\theta]\). I will denote the constant function \(S^1 \to \set{1}\) as \(1\), we use the Kunneth Theorem (from the notes Lecture 23) to find that the following is an isomorphism:
        \begin{align*}
            \mathbb{Z}^2 \cong (H^*(S^1)\otimes H^*(S^1))^1 = \gen{[d\theta]\otimes [1]}\oplus \gen{[1]\otimes [d\theta]} \overset{[\pi_1^* \wedge \pi_2^*]\oplus [\pi_1^* \wedge \pi_2^*]}{\longrightarrow} H^1(\mathbb{T}^2)
        \end{align*}
        Then by construction we get that \([\pi_1^* \wedge \pi_2^*] (d\theta \otimes 1) \overset{\text{def}}{=} [\pi_1^*(d\theta) \wedge \pi_2^*(1)] =  [dx']\) and \([\pi_1^* \wedge \pi_2^*](1 \otimes d\theta) \overset{\text{def}}{=} [\pi_1^*(1) \wedge \pi_2^*(d\theta)] = [dy']\), which verifies that these indeed form a basis for the first cohomology. \qed
        % \textbf{(a)} We first check that the map \(\rho\) is orientation preserving, a quick way would be that the action of \(\mathbb{Z}^2\) is orientation preserving, which makes the induced covering map orientation preserving. If this is not satisfactory I will prove by hand. To do so define the  following section \(s: \mathbb{R}^2 \to \bigwedge^2 T \mathbb{R}^2 \cong \mathbb{R}^2 \times \mathbb{R}\) to be \(x \mapsto (x,1)\), (this is a representative for the standard orientation on \(\mathbb{R}^2\)), then since \(\mathbb{T}^2\) is compact and \(\rho\) is a covering map, we have open sets \(U_1,\hdots,U_r\) with \(\mathbb{T}^2 \subset \bigcup_1^r \rho(U_j)\) and \(\rho_j := \rho \vert_{U_j}\) is a diffeomorphism onto its image. Take \(\eta_j\) to be a partition of unity subordinate to the \(U_j\), then define
        % \begin{align*}
        %     &t: \mathbb{T}^2 \to \bigwedge^2T\mathbb{T}^2 &
        %     x \mapsto \sum_1^r \eta_j(x)\cdot(\det d\rho_j)(s \rho_j^{-1}(x))
        % \end{align*}
        
        % We first check that the volume form \(dx\wedge dy\) descends to \(\mathbb{T}^2\), to do so take open sets \(U_1,\hdots,U_s \subset \mathbb{R}^2\) with \(\bigcup_1^s \rho(U_j) = \mathbb{T}^2\), and \(\rho \vert_{U_j}\) is a diffeomorphism to its image, denote the inverse as \(\rho^{-1}_j\). The existence of such sets follows from \(\rho\) being a covering map and we can guarantee finitely many since \(\mathbb{T}^2\) is compact. Now let \(\eta_j\) be a partition of unity subordinate to the \(U_j\), then we can define \(dx'\wedge dy' = \sum_1^s \eta_j\cdot (\rho_j^{-1})^*dx\wedge dy\), to check this indeed gives us a volume form, we check nowhere vanishing, this is straightforward once we notice that \(dx\wedge dy\) is translation invariant, so that \((\rho^{-1}_j)^* dx\wedge dy\) is independent of \(j\), and since \(\rho\) is a local diffeomorphism, it gives us an isomorphism of tangent spaces, so that we can write for any \(v \in T\mathbb{T}^2\), that \(v = d\rho(u)\), so for \(w \in \bigwedge^2T\mathbb{T}^2\), we have \(w = dp(u)\wedge dp(v)\), so that
        % \begin{align*}
        %     \left(\sum_1^s \eta_j\cdot (\rho_j^{-1})^*dx\wedge dy\right)(w) = \sum_1^s \eta_j\cdot(dx\wedge dy)(u \wedge v) = (dx\wedge dy)(u \wedge v)\sum_1^s \eta_j = (dx\wedge dy)(u \wedge v)
        % \end{align*}
        % Thus this form is nowhere vanishing since \(dx\wedge dy\) is nowhere vanishing. Finally it is immediate that \(\rho\) preserves orientation in this case, since 
    \end{pb}
\end{document}