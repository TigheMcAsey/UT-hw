\documentclass[10.5pt]{article}
\usepackage{amsmath, amsfonts, amssymb,amsthm}
\usepackage[includeheadfoot]{geometry} % For page dimensions
\usepackage{fancyhdr}
\usepackage{enumerate} % For custom lists

\fancyhf{}
\lhead{MAT1000 hw2}
\rhead{Tighe McAsey - 1008309420}
\pagestyle{fancy}

% Page dimensions
\geometry{a4paper, margin=1in}

\theoremstyle{definition}
\newtheorem{pb}{}

% Commands:

\newcommand{\set}[1]{\{#1\}}
\newcommand{\abs}[1]{\lvert#1\rvert}
\newcommand{\norm}[1]{\lvert\lvert#1\rvert\rvert}
\newcommand{\tand}{\text{ and }}
\newcommand{\tor}{\text{ or }}

\begin{document}
    \begin{pb}\textbf{(Folland 1.3.6)}
        That \(\overline{\mu}\) is a measure is clear, since \(\overline{\mu}(\emptyset) = \mu(\emptyset) = 0\), \(\text{Im}\overline{\mu} = \text{Im}\mu\) (this shows it is positive) and if \(\set{A_i}_1^\infty\) are disjoint sets in \(\overline{M}\), then each \(A_i\) can be written as \(E_i \cup F_i\) (for \(F_i\)) contained in null sets \(N_i\) so that \(\overline{\mu}\bigcup_1^\infty A_i = \overline{\mu}(\bigcup_1^\infty E_i \bigcup_1^\infty F_i)\), then \(\bigcup_1^\infty F_i \subset \bigcup_1^\infty N_i\) is a null set, so \[\overline{\mu}(\bigcup_1^\infty E_i\bigcup_1^\infty F_i) \overset{\text{defn.}}{=}\mu(\bigcup_1^\infty E_i) = \sum_1^\infty \mu(E_i) = \sum_1^\infty \overline{\mu}(E_i\cup F_i)\]
        Suppose \(N \in \overline{\mathcal{M}}\) with \(\overline{\mu}(N) = 0\) and \(F \subset N\), then \(N = N_1 \cup N_2\) with \(N_1 \in \mathcal{M}\), and \(N_2 \subset N_3 \in \mathcal{M}\), so that \(N \subset N_1 \cup N_3\) which is a \(\mu\)-measurable set, hence \(F\) is contained in the null set \(N_1 \cup N_3\) and is \(\mu\)-measurable. Finally to see uniqueness of \(\overline{\mu}\), suppose \(\mu'\) is another extension of \(\mu\) to \(\overline{\mathcal{M}}\), then for some \(E \cup F \in \overline{M}\) we have \(\mu(E) = \overline{\mu}(E \cup F) \neq \mu'(E\cup F)\), hence \(\mu'(F) > 0\), but then \(F \subset N \in \mathcal{M}\) where \(N\) is \(\mu\) null, so that \(0 < \mu'(F) \leq \mu(N) = 0\). \qed
    \end{pb}
    \begin{pb}\textbf{(Folland 1.3.7)}
        Positivity follows from each \(\mu_j\) and \(a_j\) positive, suppose \(\set{E_i}_1^\infty\) are disjoint, if any of the \(\mu_j(\bigcup_1^\infty E_i) = \infty\) then \(\sum_{i=1}^\infty\sum_{j=1}^n a_j\mu_j(E_i) \geq a_j \sum_{i=1}^\infty \mu(E_i) = a_j\mu_j(\bigcup_1^\infty E_i) = \infty\) and additivity is trivial, otherwise we can interchange sums since they converge in absolute value
        \begin{align*}
            &\sum_1^n a_j\mu_j(\emptyset) = \sum_1^n 0 = 0 \\
            &\sum_1^n a_j\mu_j(\bigcup_1^\infty E_i) = \sum_{j=1}^n a_j \sum_{i=1}^\infty \mu_j(E_i) = \sum_{i=1}^\infty \sum_{j=1}^n a_j\mu_j(E_i)
        \end{align*} \qed
    \end{pb}
    \begin{pb}\textbf{(Folland 1.3.8)}
        for any \(N\), we have \(\bigcup_{n=1}^N \bigcap_{j=n}^\infty E_j \subset E_k\) for all \(k \geq N\), hence \(\mu(\bigcup_{n=1}^N \bigcap_{j=n}^\infty E_j) \leq \liminf \mu(E_k)\). By continuity from below we have \(\lim_{N\to\infty}\mu(\bigcup_{n=1}^N \bigcap_{j=n}^\infty E_j) = \mu(\liminf E_k)\), but the limit is bounded above by \(\liminf \mu(E_k)\), so that \(\mu(\liminf E_K) \leq \liminf\mu(E_k)\).
        
        For all \(N\), we have \(\bigcap_{n=1}^N\bigcup_{j=n}^\infty E_j \supset E_k\) for \(k \geq n\), hence \(\mu(\bigcap_{n=1}^N\bigcup_{j=n}^\infty E_j) \geq \limsup \mu(E_k)\), since \(\mu(\bigcup_1^\infty E_i) < \infty\) we can invoke continuity from above to conclude \[\mu(\limsup(E_k)) = \lim_{n\to\infty}\mu(\bigcap_{n=1}^N\bigcup_{j=n}^\infty E_j) \geq \limsup \mu(E_k)\] \qed
    \end{pb}
    \begin{pb}\textbf{(Folland 1.3.9)}
        We can decompose the sets of interest as follows:
        \begin{align*}
            E = (E \setminus F) \sqcup (E \cap F), \quad F = (F \setminus E) \sqcup (F \cap E), \quad E \cup F = F \cap E \sqcup (E \setminus F) \sqcup (F \setminus E)
        \end{align*}
        The result follows from additivity on disjoint sets,
        \begin{align*}
            \mu(E) + \mu(F) = \mu(E \setminus F) + \mu(E \cap F) + \mu(F \setminus E) + \mu(F \cap E) = \mu(E \cup F) + \mu(E \cap F)
        \end{align*} \qed
    \end{pb}
    \begin{pb}\textbf{(Folland 1.3.10)}
        That \(\mu_E\) is nonnegative follows from \(\mu\) nonnegative. \(\empty = \empty \cap E\) so \(\mu_E(\emptyset) = 0\). Finally if \(\set{A_i}_1^\infty\) are disjoint sets, then so are \(\set{A_i\cap E}_1^\infty\), hence
        \begin{align*}
            \mu_E(\bigcup_1^\infty A_i) = \mu(E \cap \bigcup_1^\infty A_i) = \mu(\bigcup_1^\infty E \cap A_i) = \sum_1^\infty \mu(E\cap A_i)
        \end{align*} \qed
    \end{pb}
    \begin{pb}\textbf{(Folland 1.3.11)}
        Suppose \(\set{E_i}_1^\infty\) are disjoint sets, then let \(F_n = \bigcup_1^n E_i\), it follows that \[\mu(\bigcup_1^\infty E_i) = \mu(\bigcup_1^\infty F_n) = \lim_{n\to\infty}\mu(F_n) = \lim_{n\to\infty}\sum_1^n \mu(E_i)\]

        In the second case, let \(K_n = \bigcap_1^n E_n^c\), it follows that \(\mu K_1 \leq \mu X\) so we can use continuity from above.
        \begin{align*}
            \mu(\bigcup_1^\infty E_i) &= \mu(X) - \mu(\bigcap_1^\infty E_i^c) = \mu(X) - \mu(\bigcap_1^\infty K_n) = \mu(X) - \lim_{n\to\infty}\mu(K_n) = \mu(X) - \lim_{n\to \infty} \mu\left(\left(\bigcup_1^n E_n\right)^c \right) \\
            &= \mu(X) - \left(\lim_{n\to\infty} \mu(X) - \sum_1^n \mu(E_i)\right) = \lim_{n\to\infty} \sum_1^n \mu(E_i)
        \end{align*} \qed
    \end{pb}
    \begin{pb}\textbf{(Folland 1.3.12)}
        
        \textbf{(a)} \(E \Delta F = (E \setminus F) \sqcup (F \setminus E)\), hence \(\mu(E \setminus F) = \mu(F \setminus E) = 0\). It follows that
        \begin{align*}
            \mu(F) \leq \mu(E) + \mu(F \setminus E) = \mu(E) \tand \mu(E) \leq \mu(F) + \mu(E \setminus F) = \mu(F)
        \end{align*} \qed

        \textbf{(b)} reflexivity follows from \(\mu(E \Delta E) = \mu(\emptyset) = 0\), symmetry follows from \(E \Delta F = F \Delta E\), finally transitivity follows from the observation \(H \setminus F \subset (H \setminus E) \cup (E \setminus F)\), hence \(\mu(H \Delta E) = \mu(E \Delta F) = 0\) implies \(\mu(H \setminus F) \leq \mu(H \setminus E)+ \mu(E \setminus F) = 0\) and \(\mu(F \setminus H) \leq \mu(F \setminus E) + \mu(E \setminus H) = 0\) which gives us that \(\mu(H \Delta F) = \mu(H \setminus F) + \mu(F \setminus H) = 0\), proving transitivity. \qed

        \textbf{(c)} \(\rho(E,F) = 0 \iff E \sim F\), and \(\rho\) is nonnegative, symmetry follows from symmetry of \(\Delta\), so \(\rho\) will define a metric if it satisfies the triangle inequality. But as in the previous question \(H \setminus F \subset (H \setminus E) \cup (E\setminus F)\), applying this inequality the other way this implies that \(\mu(H\Delta F) \leq \mu(H \Delta E) + \mu(E \Delta F)\), this proves the triangle inequality for \(\rho\). \qed
    \end{pb}
    \begin{pb}\textbf{(Folland 1.3.13)}
        Suppose that \(\mu\) is not semifinite, then there is some \(E \in \mathcal{M}\), such that for all \(F \subset E\) we have \(\mu(F) = \infty\). Suppose \(X = \bigcup_1^\infty E_i\), then \(E_i \cap E \neq \emptyset\) for some \(i\), then \(\infty = \mu(E_i \cap E) \leq \mu(E_i)\), so that \(X\) cannot be a countable union of sets having finite measure. \qed
    \end{pb}
    \begin{pb}\textbf{(Folland 1.3.14)}
        Let \(C = \sup\set{\mu(F) \mid F \subset E \tand \mu(F) < \infty}\) and suppose for contradiction that \(C < \infty\), then let \(F_n\) be a sequence such that \(\lim_{n\to\infty}\mu(F_n) = C\), it follows that \(\mu(\bigcup_1^n F_j) \geq \mu(F_n)\), hence \(\lim_{n\to\infty}\mu(\bigcup_1^n F_j) = C\), using continuity from below we see that in fact \(\mu(\bigcup_1^\infty F_n) = C\). Then \(\mu(E \setminus \bigcup_1^\infty F_n) = \infty\), so \(E \setminus \bigcup_1^\infty F_n\) has some subset \(A\) with \(0 < \mu(A) < \infty\), but then
        \begin{align*}
            C \geq \mu(A\bigcup_1^\infty F_n) = \mu(\bigcup_1^\infty F_n) + \mu(A) > \mu(\bigcup_1^\infty F_n) = C
        \end{align*} \qed
    \end{pb}
    \begin{pb}\textbf{(Folland 1.3.15)}
        \textbf{(a)} \(\mu_0 \geq 0\) and \(\mu_0(\emptyset) = 0\) are obvious, Now let \(\set{E_i}_1^\infty \subset \mathcal{M}\) be disjoint sets, if \(\mu_0(E_j) = \infty\) for some \(j\), then \(\mu_0(E_j) \leq \mu_0(\bigcup_1^\infty E_i)\) (since every finite measure subset of the former is also a finite measure subset of the latter) and we are done. So assuming each \(\mu_0(E_j) < \infty\)l, let \(\epsilon > 0\), then we can choose for each \(E_j\) some \(\mu\)-measurable \(F_j \subset E_j\) such that \(\mu_0(E_j) \geq \mu(F_j) \geq \mu_0(E_j) - \epsilon2^{-j}\), so that
        \begin{align*}
            \mu_0(\bigcup_1^\infty E_i) \geq \limsup \mu(\bigcup_1^n F_j) \geq \sum_1^\infty \mu_0(E_i) - \epsilon2^{-i} = -\epsilon + \sum_1^\infty \mu_0(E_i)
        \end{align*}
        since \(\epsilon\) was arbitrary this concludes the inequality. To see the converse inequality, let \(F \subset \bigcup_1^\infty E_i\) such that \(\mu(F) < \infty\), then \(\sum_1^\infty \mu(E_i \cap F) \leq \sum_1^\infty \mu_0(E_i)\), taking a sequence \((F_n)_1^\infty\), such that \(\lim_{n\to\infty}\mu(F_n) = \mu_0(\bigcup_1^\infty E_i) \tand F_n \subset \bigcup_1^\infty E_i\) we conclude that
        \begin{align*}
            \mu_0(\bigcup_1^\infty E_i) = \lim_{n\to\infty}\mu(F_n) = \lim_{n\to\infty}\sum_1^\infty \mu(F_n \cap E_i) \leq \sum_1^\infty \mu_0(E_i)
        \end{align*} so that \(\mu_0\) is a measure. To check that \(\mu_0\) is semifinite, let \(E \in \mathcal{M}\) with \(\mu_0(E) = \infty\), then by definition there is some \(F \subset E\) with \(\mu(F) > 0\), it follows that \(\mu_0(F) = \mu(F) > 0\). \qed

        \textbf{(b)} Suppose \(\mu\) is semifinite, then it is clear for a set \(E\) of finite measure we have \(\mu(E) = \mu_0(E)\) by monotonicity. If \(E\) has infinite measure, then by (Folland 1.3.14) we have \[\mu_0(F) := \sup\set{\mu(F) \mid F \subset E \tand \mu(F) < \infty} = \infty = \mu(F)\]
        \qed

        \textbf{(c)} Define \(\nu\) as follows,
        \begin{align*}
            \nu(E) := \begin{cases}
                0 & E \text{ is \(\mu\)-semi-finite}\\
                \infty & \text{else}
            \end{cases}
        \end{align*}
        Once again positivity and \(\nu(\emptyset) = 0\) are obvious, to check countable additivity let \(\set{E_i}_1^\infty \subset \mathcal{M}\) be disjoint. If \(\nu(\bigcup_1^\infty E_i) = 0\), then \(\bigcup_1^\infty E_i\) is \(\mu\)-semi-finite, hence so is every \(E_i\), so that \(\sum_1^\infty \nu(E_i) = \sum_1^\infty 0\). If \(\nu(\bigcup_1^\infty E_i) = \infty\), then atleast one \(E_i\) is not semifinite because otherwise for any \(F \subset \bigcup_1^\infty E_i\) with \(\mu(F) = \infty \tand \mu_0(F) = 0\) we have \(\infty = \mu(F) = \sum_1^\infty \mu(F \cap E_i)\) and \(0 = \mu_0(F) = \sum_1^\infty \mu_0(F \cap E_i)\), so that for some \(E_i\) we have \(\mu(F \cap E_i) = \infty\), but \(E_i\) is semifinite by assumption, so that \(\mu_0(F\cap E_i) \neq 0\) a contradiction, this shows that atleast one \(E_j\) is not semifinite, so that \(\sum_1^\infty \nu(E_i) \geq \nu(E_j) = \infty = \nu(\bigcup_1^\infty E_i)\). This suffices to show that \(\nu\) is a measure, to see that \(\mu = \mu_0 + \nu\), let \(E \in \mathcal{M}\), if \(E\) is semifinite then either it is finite and \(\mu(E) = \mu(E_0)\) by monotonicity, or it has infinite measure in which case we are done by (Folland 1.3.14). If \(E\) is not semifinite, then \(\infty = \mu(E) = \nu(E) \leq \nu(E) + \mu_0(E)\) and we are done. \qed
    \end{pb}
    \begin{pb} \textbf{(Folland 1.3.16)}
        \textbf{(a)} We can write \(X = \bigcup_1^\infty A_i\) with \(\mu(A_i) < \infty\) by the sigma finite hypothesis, hence if \(E\) is locally measurable , we have \(E = E \cap \bigcup_1^\infty A_i = \bigcup_1^\infty (E \cap A_i) \in \mathcal{M}\) by closure under countable unions. \qed

        \textbf{(b)} Let \(E \in \tilde{\mathcal{M}}\), then for any \(A\) with \(\mu(A) < \infty\) we have \(E^c \cap A = E \cup A^c = (E \cap A) \cup A^c \in \mathcal{M}\). If \(\set{E_i}_1^\infty \subset \mathcal{M}\), then \(A \cap\bigcup_1^\infty E_i = \bigcup_1^\infty (A \cap E_i) \in \mathcal{M}\). \qed

        \textbf{(c)} positivity and \(\tilde{\mu}(\emptyset) = 0\) are clear. Now let \(\set{E_i}_1^\infty \subset \tilde{\mathcal{M}}\) be disjoint, if \(\bigcup_1^\infty E_i \not \in \mathcal{M}\), then atleast one \(E_j \not \in \mathcal{M}\), in which case \(\sum_1^\infty \tilde{\mu}(E_i) \geq \tilde{\mu}(E_j) = \infty = \tilde{\mu}(\bigcup_1^\infty E_i)\). If \(\bigcup_1^\infty E_i \in \mathcal{M}\), and \(\mu(\bigcup_1^\infty E_i) = \infty\), then \(\bigcup_1^\infty \tilde{\mu}(E_i)\) is infinity in either the case all \(E_i \in \mathcal{M}\) or the case some \(E_i \not \in \mathcal{M}\). Finally, if \(\bigcup_1^\infty E_i \in \mathcal{M}\), and \(\mu(\bigcup_1^\infty E_i) < \infty\), then \(E_i = E_i \cap\left(\bigcup_1^\infty E_i \right) \in \mathcal{M}\), so that \[\tilde{\mu}(\bigcup_1^\infty E_i) = \mu(\bigcup_1^\infty E_i) = \sum_1^\infty \mu(E_i) = \sum_1^\infty \tilde{\mu}(E_i)\] \qed

        \textbf{(d)} Suppose \(F \subset N\) where \(N\) is a \(\tilde{\mu}\)-null set, then \(\tilde{\mu}(N) = 0 \neq \infty \implies N \in \mathcal{M}\) and \(\mu(N) = \tilde{\mu}(N) = 0\), it follows by completion of \(\mu\) that \(F \in \mathcal{M} \subset \tilde{\mathcal{M}}\). \qed

        \textbf{(e)} It is clear that \(\underline{\mu}(\emptyset) = 0 \tand \underline{\mu} \geq 0\). Now let \(\set{E_i}_1^\infty \subset \mathcal{M}\) be disjoint, If some \(\underline{\mu}(E_j) = \infty\), then so does \(\underline{\mu}(\bigcup_1^\infty E_j)\), since the former is a subset of the latter. If this is not the case, then let \(\epsilon > 0\) and take \(F_i \in \mathcal{M}\) such that \(\underline{\mu}(E_i) \geq \mu(F_i) \geq \underline{\mu}(E_i) -\epsilon2^{-i}\), so that
        \begin{align*}
            \underline{\mu}(\bigcup_1^\infty E_i) \geq \limsup \mu(\bigcup_1^n F_i) \geq \sum_1^\infty \underline{\mu}(E_i) -\epsilon2^{-i} = -\epsilon + \sum_1^\infty \underline{\mu}(E_i)
        \end{align*}
        since epsilon was arbitrary this gives the inequality. For the converse, if \(F \subset \bigcup_1^\infty E_i\) with \(\mu(F) < \infty\), then \(\sum_1^\infty \mu(E_i \cap F) \leq \sum_1^\infty \underline{\mu}(E_i)\), taking a sequence \(F_n\) with \(\lim_{n\to\infty}\mu(F_n) = \underline{\mu}(\bigcup_1^\infty E_i)\), and \(F_n \subset \bigcup_1^\infty E_i\) (we can do this due to the semifinite assumption), we use the inequality to conclude that
        \begin{align*}
            \underline{\mu}(\bigcup_1^\infty E_i) = \lim_{n \to \infty}\mu(F_n) = \lim_{n \to \infty} \sum_1^\infty \mu(F_n \cap E_i) \leq \sum_1^\infty \underline{\mu}(E_i)
        \end{align*}
        So that indeed \(\underline{\mu}\) is a measure. The fact that \(\underline{\mu}\vert_\mathcal{M} = \mu\) is directly a consequence of monotonicity. Let \(E\) be a \(\underline{\mu}\) locally measurable set, and let \(A \in \mathcal{M} \subset \tilde{\mathcal{M}}\) be such that \(\mu(A) < \infty\), then \(\underline{\mu}(A) = \mu(A) < \infty\), so that by locally measurable assumption we have \(E \cap A \in \tilde{M}\), hence \(E \cap A = (E \cap A) \cap A \in \mathcal{M}\) by definition of \(\tilde{\mathcal{M}}\), this suffices to show that \(E\) is locally measurable with repect to \(\mu\), so that \(E \in \tilde{\mathcal{M}}\), i.e. \(\underline{\mu}\) is saturated. \qed

        \textbf{(f)} It is clear that \(\mu \geq 0\) and \(\mu(\emptyset) = \mu_0(\emptyset) = 0\). Now let \(\set{E_i}_1^\infty \subset \mathcal{M}\) be disjoint. Then
        \begin{align*}
            \mu(\bigcup_1^\infty E_i) = \mu_0(X_1 \cap \bigcup_1^\infty E_i) = \mu_0(\bigcup_1^\infty(X_1 \cap E_i)) = \sum_1^\infty \mu_0(X \cap E_i) = \sum_1^\infty \mu(E_i)
        \end{align*}
        so that \(\mu\) is a measure. Note that \(X_2\) (which is not countable or cocountable) is locally finite, since if \(A\) has finite measure, then \(A \cap X_1\) is finite, hence \(A^c\) is uncountable, so that \(A \cap X_2\) must be countable, so that \(A \cap X_2 \in \mathcal{M}\). Now for any \(E \subset X_2\) such that \(E \in \mathcal{M}\), we have \(\mu(E) = \mu_0(E \cap X_1) = \mu_0(\emptyset) = 0\), so that \(\underline{\mu}(X_2) = 0\) (note here that \(X_1\cup X_2\) is obviously semifinite since any set containing infinitely members of \(X_1\) contains a set with finitely many which is also countable or cocountable respectively so that \(\underline{\mu}\) is a measure on our space), but \(X_2 \not \in \mathcal{M}\) so that
        \begin{align*}
            \tilde{\mu}(X_2) = \infty \neq 0 = \underline{\mu}(X_2)
        \end{align*} \qed
    \end{pb}
    \begin{pb}\textbf{(Folland 1.4.17)}
        \begin{align*}
            \mu^*(E\cap\bigcup_1^nA_i) = \mu^*\left((E\cap\bigcup_1^nA_i)\cap A_n\right) + \mu^*\left((E\cap\bigcup_1^nA_i)\cap A_n^c\right) = \mu^*(A_n) + \mu^*(\bigcup_1^{n-1} A_n)
        \end{align*}
        applying this process recursively we find that \(\mu^*(E\cap\bigcup_1^nA_i) = \sum_1^n \mu^*(E\cap A_i)\), so that by monotonicity we have for any \(n\),
        \begin{align*}
            \mu^*(E\cap\bigcup_1^\infty A_i) \geq \mu^*(E\cap\bigcup_1^nA_i) = \sum_1^n \mu^* (E \cap A_i)
        \end{align*}
        and hence \(\mu^*(E\cap\bigcup_1^\infty A_i) \geq \sum_1^\infty \mu^* (E \cap A_i)\). \qed
    \end{pb}
    \begin{pb}\textbf{(Folland 1.4.18)}

        \textbf{(a)} Let \(\epsilon > 0\) and \(\set{A_j}_1^\infty \subset \mathcal{A}\), such that \(\mu^*(E) \geq \sum_1^\infty \mu_0(A_j) - \epsilon\) and \(E \subset \bigcup_1^\infty A_j\) (existence of such a collection is guaranteed by the definition of outer measure). Take \(A = \bigcup_1^\infty A_j \in \mathcal{A}_\sigma\), then \[\mu^*(A) = \inf\set{\sum_1^\infty \mu_0(B_i) \mid A \subset \bigcup_1^\infty B_i \tand B_i \in \mathcal{A}} \leq \sum_1^\infty \mu_0(A_j) \leq \mu^*(E) + \epsilon\] \qed

        \textbf{(b)} Suppose such a set exists, since \(\mathcal{A} \subset \mathcal{M}\) is a sigma algebra we know that \(B\) is measurable, it follows that for \(F \subset X\), we have 
        \begin{align*}
            \mu^*(F \cap E^c) &= \mu^*(F \cap E^c \cap B^c) + \mu^*(F \cap E^c \cap B) = \mu^*(F \cap (B \setminus E)) + \mu^*(F \cap B^c) \\ 
            &\leq \mu^*(B\setminus E) + \mu^*(F \cap B^c) = \mu^*(F \cap B^c)
        \end{align*}
        so that they are equal, since the other equality follows from monotonicity. Applying this equality, we get
        \begin{align*}
            \mu^*(F) = \mu^*(F \cap B) + \mu^*(F \cap B^c) \geq \mu^*(F \cap E) + \mu^*(F \cap B^c) = \mu^*(F \cap E) + \mu^*(F \cap E^c)
        \end{align*}
        so that \(E\) is measurable.

        Conversely, assume that \(E\) is measurable, and \(\mu^*(E) < \infty\), then let \((A^\sigma_n)\) be a sequence of \(\mathcal{A}_\sigma\) sets, each containing \(E\), such that \(\lim_{n\to\infty} \mu^*(A^\sigma_n) = \mu^*(E)\) which is possible by part (a), then \(\cap_1^\infty A^\sigma_n\) is \(\mathcal{A}_{\sigma \delta}\), and for any \(n\) we have (using measurability in the equality) that
        \begin{align*}
            0 \leq \mu^*(\cap_1^\infty A^\sigma_n \setminus E) \leq \mu^*(A_n^\sigma \setminus E) = \mu^*(A_n^\sigma) - \mu^*(E),\quad \forall n
        \end{align*}
        So by the squeeze theorem \(\mu^*(\cap_1^\infty A^\sigma_n \setminus E) = 0\).
        \qed

        \textbf{(c)} The finiteness condition is only used in the converse, so assume that \(E\) is measurable, and we can write \(X = \bigcup_1^\infty A_i\) for \(A_i \in \mathcal{A}\), then \(X = \bigsqcup_1^\infty B_i\) where \(B_n = A_n \cap_1^{n-1} A_i^c \in \mathcal{A}\). By part (b), we can choose \(E_i\), such that \(E_i\) is \(\mathcal{A}_{\sigma \delta}\), \(E\cap B_i \subset E_i\) and \(\mu^*(E_i \setminus (E \cap B_i)) = 0\) (we can also say \(E_i \subset B_i\) by taking intersection- intersections of \(\mathcal{A}_{\sigma \delta}\) sets are \(\mathcal{A}_{\sigma \delta}\)). Now since each \(E_i \subset B_i\) so that \(\set{E_i}_1^\infty\) are disjoint, it follows that (using here (Folland 1.4.17), because \(E_i\) measurable for all \(i\), since \(E_i\) is \(\mathcal{A}_{\delta \sigma}\))
        \begin{align*}
            \mu^*(\bigcup_1^\infty E_i \setminus E) = \mu^*(\bigcup_1^\infty E_i \cap E^c) = \sum_1^\infty \mu^*(E_i \cap E^c) = \sum_1^\infty \mu^*(E_i \setminus (E \cap B_i)) = \sum_1^\infty 0 = 0
        \end{align*} \qed
    \end{pb}
    \begin{pb}\textbf{(Folland 1.4.19)}
        First suppose that \(E\) is measurable, then
        \begin{align*}
            \mu_0(X) = \mu^*(X) = \mu^*(E \cap X) + \mu^*(E^c \cap X) = \mu^*(E) + \mu^*(E^c) \implies \mu^*(E) = \mu_0(X) - \mu^*(E^c)
        \end{align*}
        Conversely, if \(\mu_*(E) = \mu^*(E)\), then for any measurable \(A \supset E\),
        \begin{align*}
            \mu^*(E) = \mu_0(X) -\mu^*(E^c) = \mu^*(X) - \mu^*(E^c) = \mu^*(A) + \mu^*(A^c) - \mu^*(E^c)
        \end{align*} 
        and furthermore
        \begin{align*}
            \mu^*(E^c) = \mu^*(A \setminus E) + \mu^*(A^c \cap E^c) = \mu^*(A \setminus E) + \mu^*(A^c) \implies \mu^*(A\setminus E) = \mu^*(E^c) - \mu^*(A^c)
        \end{align*}
        Combining these we get for any \(A \supset E\)
        \begin{align*}
            \mu^*(A \setminus E) = \mu^*(E^c) - \mu^*(A^c) = (\mu^*(A) + \mu^*(A^c) - \mu^*(E)) - \mu^*(A^c) = \mu^*(A) - \mu^*(E)
        \end{align*}
        Now by (Folland 1.4.18 (a)), we have some sequence \((A_n^\sigma)_n\) of \(\mathcal{A}_\sigma\) sets, such that \(E \subset A_n^\sigma\) for all \(n\), and \(\mu(A_n^\sigma) \to \mu(E)\). It follows that for all \(n\), we have
        \begin{align*}
            0 \leq \mu^*\left(\left(\bigcap_1^\infty A_n^\sigma\right)\setminus E\right) \leq \mu^*(A_n^\sigma \setminus E) = \mu^*(A_n^\sigma) - \mu^*(E)
        \end{align*}
        Since this holds for all \(n\) taking the limit on the right we get \(\mu^*\left(\left(\bigcap_1^\infty A_n^\sigma\right)\setminus E\right) = 0\), so \(E\) is contained in a \(\mathcal{A}_{\sigma \delta}\) set, such that the measure of the difference set is zero. We are done by (Folland 1.4.18(b)). \qed
    \end{pb}
    \begin{pb}\textbf{(Folland 1.4.20)}

        \textbf{(a)} If \(\set{A_i}_1^\infty \subset \mathcal{M}^*\), and \(E \subset \bigcup_1^\infty A_i\), then from monotonicity and subadditivity we have \(\mu^*(E) \leq \mu^*(\bigcup_1^\infty A_i) \leq \sum_1^\infty \mu^*(A_i)\), hence
        \begin{align*}
            \mu^+(E) = \inf\set{\sum_1^\infty \mu^*(A_i) \mid E \subset \bigcup_1^\infty A_i \tand A_i \in \mathcal{M}^*} \geq \mu^*(E)
        \end{align*}
        Suppose there exists \(E \subset A \in \mathcal{M}^*\) with \(\mu^*(A) = \mu^*(E)\), then \(\mu^*(E) = \mu^*(A) \geq \mu^+(E) \geq \mu^*(E)\). Conversely, If \(\mu^+(E) = \mu^*(E)\), then for any \(n\), there are some \(A_i \in \mathcal{M}^*\) with \(E \subset \bigcup_1^\infty A_i\) such that \(\sum_1^\infty \mu^*(A_i) \leq \mu^*(E) + \frac{1}{n}\), then for each \(n\) we define \(B_n = \bigcup_{n=1}^\infty A_n\), then 
        \[\mu^*(E) \leq \mu^*(B_n) \leq \sum_1^\infty \mu^*(A_i) \leq \mu^*(E) + \frac{1}{n}\]
        It follows that \(\mu^*(E) \leq \mu^*(\bigcap_1^\infty B_n) \leq \mu^*(E) + \frac{1}{n}\) for all \(n\), and hence \(\mu^*(E) = \mu^*(B_n)\). \qed

        \textbf{(b)} If \(\mu^*\) is induced by a pre-measure, then by (Folland 1.4.18 (a)) we find that for any \(E \subset X\) we have some \(A_n \supset E\) with \(\mu^*(A_n) - n^{-1} \leq \mu^*(E)\), hence for any \(n\), \[\mu^*(E) \leq \mu^*(\bigcap_1^\infty A_n) \leq \mu^*(A_n) \leq \mu^*(E) + n^{-1}\]
        So that \(\mu^*(E) = \mu^*(\bigcap_1^\infty A_n) \in \mathcal{M}^* \), and hence we are done by (a). \qed

        \textbf{(c)} Define the outer measure as follows:
        \begin{align*}
            \mu^*: \begin{cases}
                X \mapsto 2 \\
                \set{1} \mapsto 2 \\
                \set{0} \mapsto 1
            \end{cases}
        \end{align*}
        Then since \(\mu^*(X) \neq \mu^*\set{1} + \mu^*\set{0}\) we find that \(\set{X,\emptyset}\) is the sigma algebra of measurable sets. It follows that \(\mu^+(\set{0}) = \mu^*(X) = 2 \neq 1 = \mu^*(\set{0})\). \qed
    \end{pb}
    \begin{pb}\textbf{(Folland 1.4.21)}
        
    \end{pb}
    \begin{pb}\textbf{(Folland 1.4.22)}
        
    \end{pb}
    \begin{pb}\textbf{(Folland 1.4.23)}
        
    \end{pb}
    \begin{pb}\textbf{(Folland 1.4.24)}
        
    \end{pb}
\end{document}