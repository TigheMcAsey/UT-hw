\documentclass[10.5pt]{article}
\usepackage{amsmath, amsfonts, amssymb,amsthm}
\usepackage[includeheadfoot]{geometry} % For page dimensions
\usepackage{fancyhdr}
\usepackage{enumerate} % For custom lists

\fancyhf{}
\lhead{MAT1000 hw2}
\rhead{Tighe McAsey - 1008309420}
\pagestyle{fancy}

% Page dimensions
\geometry{a4paper, margin=1in}

\theoremstyle{definition}
\newtheorem{pb}{}

% Commands:

\newcommand{\set}[1]{\{#1\}}
\newcommand{\abs}[1]{\lvert#1\rvert}
\newcommand{\norm}[1]{\lvert\lvert#1\rvert\rvert}
\newcommand{\tand}{\text{ and }}
\newcommand{\tor}{\text{ or }}

\begin{document}
    \begin{pb}\textbf{(Folland 1.3.6)}
        That \(\overline{\mu}\) is a measure is clear, since \(\overline{\mu}(\emptyset) = \mu(\emptyset) = 0\), \(\text{Im}\overline{\mu} = \text{Im}\mu\) (this shows it is positive) and if \(\set{A_i}_1^\infty\) are disjoint sets in \(\overline{M}\), then each \(A_i\) can be written as \(E_i \cup F_i\) (for \(F_i\)) contained in null sets \(N_i\) so that \(\overline{\mu}\bigcup_1^\infty A_i = \overline{\mu}(\bigcup_1^\infty E_i \bigcup_1^\infty F_i)\), then \(\bigcup_1^\infty F_i \subset \bigcup_1^\infty N_i\) is a null set, so \[\overline{\mu}(\bigcup_1^\infty E_i\bigcup_1^\infty F_i) \overset{\text{defn.}}{=}\mu(\bigcup_1^\infty E_i) = \sum_1^\infty \mu(E_i) = \sum_1^\infty \overline{\mu}(E_i\cup F_i)\]
        Suppose \(N \in \overline{\mathcal{M}}\) with \(\overline{\mu}(N) = 0\) and \(F \subset N\), then \(N = N_1 \cup N_2\) with \(N_1 \in \mathcal{M}\), and \(N_2 \subset N_3 \in \mathcal{M}\), so that \(N \subset N_1 \cup N_3\) which is a \(\mu\)-measurable set, hence \(F\) is contained in the null set \(N_1 \cup N_3\) and is \(\mu\)-measurable. Finally to see uniqueness of \(\overline{\mu}\), suppose \(\mu'\) is another extension of \(\mu\) to \(\overline{\mathcal{M}}\), then for some \(E \cup F \in \overline{M}\) we have \(\mu(E) = \overline{\mu}(E \cup F) \neq \mu'(E\cup F)\), hence \(\mu'(F) > 0\), but then \(F \subset N \in \mathcal{M}\) where \(N\) is \(\mu\) null, so that \(0 < \mu'(F) \leq \mu(N) = 0\). \qed
    \end{pb}
    \begin{pb}\textbf{(Folland 1.3.7)}
        Positivity follows from each \(\mu_j\) and \(a_j\) positive, suppose \(\set{E_i}_1^\infty\) are disjoint, if any of the \(\mu_j(\bigcup_1^\infty E_i) = \infty\) then \(\sum_{i=1}^\infty\sum_{j=1}^n a_j\mu_j(E_i) \geq a_j \sum_{i=1}^\infty \mu(E_i) = a_j\mu_j(\bigcup_1^\infty E_i) = \infty\) and additivity is trivial, otherwise we can interchange sums since they converge in absolute value
        \begin{align*}
            &\sum_1^n a_j\mu_j(\emptyset) = \sum_1^n 0 = 0 \\
            &\sum_1^n a_j\mu_j(\bigcup_1^\infty E_i) = \sum_{j=1}^n a_j \sum_{i=1}^\infty \mu_j(E_i) = \sum_{i=1}^\infty \sum_{j=1}^n a_j\mu_j(E_i)
        \end{align*} \qed
    \end{pb}
    \begin{pb}\textbf{(Folland 1.3.8)}
        for any \(N\), we have \(\bigcup_{n=1}^N \bigcap_{j=n}^\infty E_j \subset E_k\) for all \(k \geq N\), hence \(\mu(\bigcup_{n=1}^N \bigcap_{j=n}^\infty E_j) \leq \liminf \mu(E_k)\). By continuity from below we have \(\lim_{N\to\infty}\mu(\bigcup_{n=1}^N \bigcap_{j=n}^\infty E_j) = \mu(\liminf E_k)\), but the limit is bounded above by \(\liminf \mu(E_k)\), so that \(\mu(\liminf E_K) \leq \liminf\mu(E_k)\).
        
        For all \(N\), we have \(\bigcap_{n=1}^N\bigcup_{j=n}^\infty E_j \supset E_k\) for \(k \geq n\), hence \(\mu(\bigcap_{n=1}^N\bigcup_{j=n}^\infty E_j) \geq \limsup \mu(E_k)\), since \(\mu(\bigcup_1^\infty E_i) < \infty\) we can invoke continuity from above to conclude \[\mu(\limsup(E_k)) = \lim_{n\to\infty}\mu(\bigcap_{n=1}^N\bigcup_{j=n}^\infty E_j) \geq \limsup \mu(E_k)\] \qed
    \end{pb}
    \begin{pb}\textbf{(Folland 1.3.9)}
        We can decompose the sets of interest as follows:
        \begin{align*}
            E = (E \setminus F) \sqcup (E \cap F), \quad F = (F \setminus E) \sqcup (F \cap E), \quad E \cup F = F \cap E \sqcup (E \setminus F) \sqcup (F \setminus E)
        \end{align*}
        The result follows from additivity on disjoint sets,
        \begin{align*}
            \mu(E) + \mu(F) = \mu(E \setminus F) + \mu(E \cap F) + \mu(F \setminus E) + \mu(F \cap E) = \mu(E \cup F) + \mu(E \cap F)
        \end{align*} \qed
    \end{pb}
    \begin{pb}\textbf{(Folland 1.3.10)}
        That \(\mu_E\) is nonnegative follows from \(\mu\) nonnegative. \(\empty = \empty \cap E\) so \(\mu_E(\emptyset) = 0\). Finally if \(\set{A_i}_1^\infty\) are disjoint sets, then so are \(\set{A_i\cap E}_1^\infty\), hence
        \begin{align*}
            \mu_E(\bigcup_1^\infty A_i) = \mu(E \cap \bigcup_1^\infty A_i) = \mu(\bigcup_1^\infty E \cap A_i) = \sum_1^\infty \mu(E\cap A_i)
        \end{align*} \qed
    \end{pb}
    \begin{pb}\textbf{(Folland 1.3.11)}
        Suppose \(\set{E_i}_1^\infty\) are disjoint sets, then let \(F_n = \bigcup_1^n E_i\), it follows that \[\mu(\bigcup_1^\infty E_i) = \mu(\bigcup_1^\infty F_n) = \lim_{n\to\infty}\mu(F_n) = \lim_{n\to\infty}\sum_1^n \mu(E_i)\]

        In the second case, let \(K_n = \bigcap_1^n E_n^c\), it follows that \(\mu K_1 \leq \mu X\) so we can use continuity from above.
        \begin{align*}
            \mu(\bigcup_1^\infty E_i) &= \mu(X) - \mu(\bigcap_1^\infty E_i^c) = \mu(X) - \mu(\bigcap_1^\infty K_n) = \mu(X) - \lim_{n\to\infty}\mu(K_n) = \mu(X) - \lim_{n\to \infty} \mu\left(\left(\bigcup_1^n E_n\right)^c \right) \\
            &= \mu(X) - \left(\lim_{n\to\infty} \mu(X) - \sum_1^n \mu(E_i)\right) = \lim_{n\to\infty} \sum_1^n \mu(E_i)
        \end{align*} \qed
    \end{pb}
    \begin{pb}\textbf{(Folland 1.3.12)}
        
        \textbf{(a)} \(E \Delta F = (E \setminus F) \sqcup (F \setminus E)\), hence \(\mu(E \setminus F) = \mu(F \setminus E) = 0\). It follows that
        \begin{align*}
            \mu(F) \leq \mu(E) + \mu(F \setminus E) = \mu(E) \tand \mu(E) \leq \mu(F) + \mu(E \setminus F) = \mu(F)
        \end{align*} \qed

        \textbf{(b)} reflexivity follows from \(\mu(E \Delta E) = \mu(\emptyset) = 0\), symmetry follows from \(E \Delta F = F \Delta E\), finally transitivity follows from the observation \(H \setminus F \subset (H \setminus E) \cup (E \setminus F)\), hence \(\mu(H \Delta E) = \mu(E \Delta F) = 0\) implies \(\mu(H \setminus F) \leq \mu(H \setminus E)+ \mu(E \setminus F) = 0\) and \(\mu(F \setminus H) \leq \mu(F \setminus E) + \mu(E \setminus H) = 0\) which gives us that \(\mu(H \Delta F) = \mu(H \setminus F) + \mu(F \setminus H) = 0\), proving transitivity. \qed

        \textbf{(c)} \(\rho(E,F) = 0 \iff E \sim F\), and \(\rho\) is nonnegative, symmetry follows from symmetry of \(\Delta\), so \(\rho\) will define a metric if it satisfies the triangle inequality. But as in the previous question \(H \setminus F \subset (H \setminus E) \cup (E\setminus F)\), applying this inequality the other way this implies that \(\mu(H\Delta F) \leq \mu(H \Delta E) + \mu(E \Delta F)\), this proves the triangle inequality for \(\rho\). \qed
    \end{pb}
    \begin{pb}\textbf{(Folland 1.3.13)}
        Suppose that \(\mu\) is not semifinite, then there is some \(E \in \mathcal{M}\), such that for all \(F \subset E\) we have \(\mu(F) = \infty\). Suppose \(X = \bigcup_1^\infty E_i\), then \(E_i \cap E \neq \emptyset\) for some \(i\), then \(\infty = \mu(E_i \cap E) \leq \mu(E_i)\), so that \(X\) cannot be a countable union of sets having finite measure. \qed
    \end{pb}
    \begin{pb}\textbf{(Folland 1.3.14)}
        Let \(C = \sup\set{\mu(F) \mid F \subset E \tand \mu(F) < \infty}\) and suppose for contradiction that \(C < \infty\), then let \(F_n\) be a sequence such that \(\lim_{n\to\infty}\mu(F_n) = C\), it follows that \(\mu(\bigcup_1^n F_j) \geq \mu(F_n)\), hence \(\lim_{n\to\infty}\mu(\bigcup_1^n F_j) = C\), using continuity from below we see that in fact \(\mu(\bigcup_1^\infty F_n) = C\). Then \(\mu(E \setminus \bigcup_1^\infty F_n) = \infty\), so \(E \setminus \bigcup_1^\infty F_n\) has some subset \(A\) with \(0 < \mu(A) < \infty\), but then
        \begin{align*}
            C \geq \mu(A\bigcup_1^\infty F_n) = \mu(\bigcup_1^\infty F_n) + \mu(A) > \mu(\bigcup_1^\infty F_n) = C
        \end{align*} \qed
    \end{pb}
    \begin{pb}
        \(\mu_0 \geq 0\) and \(\mu_0(\emptyset) = 0\) are obvious, 
    \end{pb}
\end{document}