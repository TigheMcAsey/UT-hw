\documentclass[10.5pt]{article}
\usepackage{amsmath, amsfonts, amssymb,amsthm}
\usepackage[includeheadfoot]{geometry} % For page dimensions
\usepackage{fancyhdr}
\usepackage{enumerate} % For custom lists
\usepackage{tikz-cd}
\usepackage{graphicx}
\usepackage{hyperref}

\fancyhf{}
\lhead{MAT1190 Brief Notes}
\rhead{Tighe McAsey - 1008309420}
\pagestyle{fancy}

% Page dimensions
\geometry{a4paper, margin=1in}

\theoremstyle{definition}
\newtheorem{pb}{}
\newtheorem{theorem}{Theorem}
\newtheorem{proposition}{Proposition}
\newtheorem{rmk}{Remark}
\newtheorem{exe}{exercise}
\newtheorem{definition}{definition}
\newtheorem*{example}{Example(s)}

% Commands:

\newcommand{\set}[1]{\{#1\}}
\newcommand{\abs}[1]{\lvert#1\rvert}
\newcommand{\norm}[1]{\lvert\lvert#1\rvert\rvert}
\newcommand{\tand}{\text{ and }}
\newcommand{\tor}{\text{ or }}
\newcommand{\spec}{\text{Spec}}
\title{Concise AG Notes - UofT MAT1190}
\author{Tighe}


\begin{document}
\maketitle
    \section{Lecture Notes}
    \subsection{Lecture 1 (Sept 3, 2025)}
    \begin{theorem}\textbf{(Gelfond-Neymark)}
        A compact topological space is determined by its ring of smooth functions. In particular if the ring \(C(X) := C(X,\mathbb{R})\) and \(C(X) \cong C(Y)\), then \(X \cong Y\).
    \end{theorem}
    \begin{proposition} Each point in \(X\) corresponds to a maximal ideal of \(C(X)\), moreover if \(X\) is compact, then the correspondence is 1-1.
    \end{proposition}
    \begin{proof}
        the evaluation at a point gives a surjective homomorphism \(C(X) \to \mathbb{R}\), the image is a field, hence the kernel is a maximal ideal corresponding to the point.

        Now in the compact case, (assume \(X\) is Hausdorff?), then \(X\) is Hausdorff and compact hence normal. We can use Uhrysohn's lemma to get a function vanishing at \(x\) but not \(y\). Now suppose that for some maximal ideal \(\mathfrak{m} \subset C(X)\) for any point \(p \in X\) there is a continuous function with \(f(p) \neq 0\), then the set \(U_f = \set{x \in X \mid f(x) \neq 0}\) is open , and \(\bigcup_{f \in C(X)}U_f = X\), so we get a finite subcover. Take a linear combination of the functions in this subcover to complete the proof.
    \end{proof}
    \begin{definition}
        The Zariski Topology on \(\spec_{\max}(R)\) is the coarsest topology such that when \(\mathfrak{m} \leftrightarrow x\) \(f: \mathfrak{m} \to f(x)\) is continuous, where the topology on \(\mathbb{R}\) is taken as the cofinite topology. The closed sets in this topology are the vanishing loci of \(f \in C(X)\).
    \end{definition}
    \begin{itemize}
        \item Exercise \ref{HS1.4}, complete Hartshorne exercise 1.4
    \end{itemize}

    %%%%%%%%%%%%%%%%%%%%%%%%%%%%%%%%%%%%%%%%%%%%%%%%%%%%%%%%%%%%%%%%%%

    \subsection{Lecture 2 (Sept 5, 2025)}
    \begin{definition}
        For \(T \subset R_n := k[x_1,\hdots,x_n]\) and \(S \subset k^n\) we define \[V(T) = \set{x \in k^n \mid f(x) = 0, \forall f \in T} \tand I(S) = \set{f \in R_n \mid f(x) = 0, \forall x \in S}\]
    \end{definition}
    \begin{proposition}
        Suppose \(k\) is an uncountable field, and \(L/k\) is an extension with \([L:k] \leq \#\mathbb{N}\), then \(L = k\).
    \end{proposition}
    \begin{proof}
        Suppose not, then let \(x \in L\setminus k\), we find that \(\set{\frac{1}{x-\lambda} \mid \lambda \in k}\) is uncountable, so that there must be an algebraic relation. Thus there exist \(\mu_i \in k\) with \(\sum_1^n \frac{\mu_i}{x - \lambda_i} = 0\), so that \(\sum_1^n \mu_j \prod_{i \neq j}(x - \lambda_i) = 0\), but then \(x\) is algebraic over \(k\), hence \(x \in k\), contradiction.
    \end{proof}
    \begin{theorem}\textbf{(Nullstellensatz - weak form)}
        \(V(T) = \emptyset \implies (T) = R_n\)
    \end{theorem}
    \begin{proof}
        We assume here that \(k\) is uncountable (this is unnecessary- use Noether Normalization). Since \(J := (T) \subset R_n\) is an ideal it is contained in a maximal ideal \(\mathfrak{m}\). Then \(R_n/\mathfrak{m}\) is a field extension of \(k\) with countable dimension, by the previous proposition it is equal to \(k\). It follows that each \(x_i \mapsto a_i \in k\) when taking the quotient \(R_n \to R_n/\mathfrak{m} = k\), it follows that \(I\) vanishes on \((a_1,\hdots,a_n)\), so \(I\) cannot be contained in a maximal ideal.
    \end{proof}
    \begin{theorem}\textbf{(Nullstellensatz)}
        \[IV(J) = \sqrt{J}\]
    \end{theorem}
    \begin{proof}
        By Hilbert's basis theorem, we reduce to the finitely generated case. Let \(f \in IV(\set{f_1,\hdots,f_r})\), then \((1-tf,f_1,\hdots,f_r) \subset R_n[t]\) has no common zero. Then \(g_0(1-tf) + g_1f_1 + \cdots + g_rf_r = 1\), and let \(N = \max_i\set{\deg_tg_i}\). Taking \(t = f^{-1}\), we get \(\sum_1^r g_if_i = 1\), so that for \(h_i = f^Ng_i \in R_n\) we get \(\sum_1^r h_if_i = f^N \in I \implies f \in \sqrt{I}\).
    \end{proof}
    
    The Nullstellensatz gives a bijection
    \begin{align*}
        \set{\text{Affine algebraic varieties}} &\longleftrightarrow \set{\text{Finitely generated reduced k-algebras}} \\
        V(\sqrt{J}) &\longleftrightarrow R_n/\sqrt{J}
    \end{align*}
    Moreover, this is a categorical equivalence
    \begin{align*}
        \text{Var}_k \cong \left(\text{Alg}_k^{\text{reduced}}\right)^{\text{op}}
    \end{align*}
    
    %%%%%%%%%%%%%%%%%%%%%%%%%%%%%%%%%%%%%%%%%%%%%%%%%%%%%%%%%%%%%%%%%%

    \subsection{Lecture 3 (Sept 8, 2025)}

    \begin{definition}
        Let \(\pi: S \to X\) be a local homeomorphism, then \(S\) is called an \'etal\'e space, or a sheaf on \(X\).
    \end{definition}

    \begin{example}
        \begin{enumerate}
            \item \(\empty \to X\)
            \item \(1: X \to X\)
            \item \(I\) a set with the discrete topology and the projection \(X \times I \to X\)
            \item A covering space, more explicitly the mobious covering
            \begin{align*}
                S^1 &\to S^1 \\ z &\mapsto z^2
            \end{align*}
            \item \(U \subset X\) an open set, \(\iota: U \to X\)
            \item If \(x \in X\) is a closed point, then we can construct the space \(X \sqcup_{X \setminus \set{x}}X = X \times \set{1,2}/\sim\) where \((y,1) \sim (y,2)\) when \(y \neq x\). This comes with the codiagonal map \(\nabla: X \sqcup_{X \setminus \set{x}}X \to X\), where \(\nabla\vert_{X \times \set{i}} = 1_X, i \in \set{1,2}\).
            
            This is a generalization of the line with two origins.
            \item \(I \neq \emptyset\), then take \(\underset{X \setminus \set{x}}{\sqcup_I} X \overset{\nabla}{\longrightarrow} X\)
        \end{enumerate}
    \end{example}

    \begin{definition}
        If \(U \subset X\) is an open set, then a section on \(U\) is a continuous map \(s: U \to S\) such that the following commutes:
        \begin{equation*}
            \begin{tikzcd}
                &S \arrow[d,"{\pi}"] \\
                U \arrow[ru,"{s}"] \arrow[r,"{\iota}"] &X
            \end{tikzcd}
        \end{equation*}
        The set of sections is denoted \(S(U) \tor \Gamma(U,S)\). If \(U = X\), then \(s\) is called a global section with notation \(S(X)\) or \(\Gamma(S)\).
    \end{definition}

    \begin{example}\textbf{(Revisited)}
        \begin{enumerate}
            \item \begin{align*}
                S(U) = \begin{cases}
                    1_\emptyset & U = \emptyset \\
                    \emptyset & \text{else}
                \end{cases}
            \end{align*}
            \item \begin{align*}
                S(U) = \set{\iota_U}
            \end{align*}
            \item \begin{align*}
                S(U) = \hom_\text{set}(\pi_0(U),I)
            \end{align*}
            \item \begin{align*}
                S(U) = \set{f: U \to \mathbb{C} \mid f(z^2) = z}
            \end{align*}
            \item \begin{align*}
                S(U) = \begin{cases}
                    \set{\iota} & x \not \in U \\ \set{1,2} & x \in U
                \end{cases}
            \end{align*}
            \item \begin{align*}
                S(U) = \begin{cases}
                    \set{\iota} & x \not \in U \\ I & x \in U
                \end{cases}
            \end{align*}
            This particular example is called the ``sky-scraper sheaf''
        \end{enumerate}
    \end{example}

    \begin{proposition}\label{Holomorphic as sections}
            There is a \'etal\'e space \(\mathcal{H}\) over \(\mathbb{C}_\text{EUC}\) with sections corresponding to holomorphic functions on \(\mathbb{C}\).
    \end{proposition}
    \begin{proof}
        The construction of \(\mathcal{H}\) as a set is given, alongside its topology. Verifying the claim is exercise \ref{Holomorphic as sections exe}.
        
        \begin{align*}
            \mathcal{H} := \bigsqcup_{z_0 \in \mathbb{C}}\left\{\sum_1^\infty c_n(z-z_0)^n \mid \text{the series converges in some neighborhood of }z_0\right\}
        \end{align*}
        And define the topology on \(\mathcal{H}\) as the strongest topology such that for any open set \(U\), and holomorphic \(f: U \to \mathbb{C}\) we have the following map is continuous
        \begin{align*}
            \mathcal{H}f: U &\to \mathcal{H} \\
            z_0 &\mapsto \text{The Taylor expansion of }f \text{ at }z_0
        \end{align*}
    \end{proof}


    %%%%%%%%%%%%%%%%%%%%%%%%%%%%%%%%%%%%%%%%%%%%%%%%%%%%%%%%%%%%%%%%%%


    \subsection{Lecture 4 (Sept 10, 2025)}

    \begin{definition}
        Let \(\pi:S \to X\) be \'etal\'e, then \(S_x := \pi^{-1}(x)\) is called the stalk of \(x\).
    \end{definition}

    \begin{example}
        \begin{enumerate}
            \item \(1: X \to X\), \(S_x = \set{x}\)
            \item \(X \times I \to X\), \(S_x \cong I\)
            \item \(\underset{X \setminus \set{x}}{\sqcup_I} X \overset{\nabla}{\longrightarrow} X\), then \(S_y \cong \begin{cases}
                I & y=x\\ \set{\overline{y}} & y\neq x
            \end{cases}\)
            \item \(\mathcal{H} \to \mathbb{C}\) \(\mathcal{H}_{z_0}\) is locally convergent power series at \(z_0\).
        \end{enumerate}
    \end{example}

    \begin{proposition}
        If \(\pi:S \to X\) is \'etal\'e and \(y \in \pi^{-1}(x)\), then there is an open set \(U \supset \set{x}\) and a section \(s:U \to S\) with \(s(x) = y\). Moreover, given two sections \(s_i \in \Gamma(U_i,S)\) there is some \(V \subset U_1\cap U_2\) containing \(x\), such that \(s_1\vert_V = s_2\vert_V\).
    \end{proposition}
    \begin{proof}
        The proof is exercise \ref{Existence and uniqueness sections on stalks}.
    \end{proof}



    %%%%%%%%%%%%%%%%%%%%%%%%%%%%%%%%%%%%%%%%%%%%%%%%%%%%%%%%%%%%%%%%%%

    \section{Exercises}
    \begin{exe}\label{HS1.4}\textbf{(Hartshorne Exercise 1.4)}
        An algebraically closed field is infinite, moreover the zero sets of polynomials are either \(k\) or a finite subset of \(k\). Consider the closed set \(V(x-y) \subset \mathbb{A}^2\), then it is an infinite set so if \(\mathbb{A}^1\times \mathbb{A}^1 = \mathbb{A}^2\), then it must be of the form \(\mathbb{A}^1 \times F \cup E \times \mathbb{A}^1\), where \(E,F \subset \mathbb{A}^1\) are closed. But for a fixed \(x\) \tor \(y\) we have \(V(x-y)\) has cardinality \(1\) which makes this impossible. \qed
    \end{exe}
    \begin{exe}\label{Holomorphic as sections exe}
        \textbf{(Show that \(\mathcal{H}(U) = \mathbf{\set{f: U \to \mathbb{C} \mid f \text{ holomorphic}}}\))} where we define \(\mathcal{H}\) in proposition \ref{Holomorphic as sections}.

        We first check that it is a local homeomorphism, for \(z \in \mathbb{C}\) and \(U \supset z\), we can take a function \(f\) holomorphic on \(U\), then \(\mathcal{H}f\) has only one taylor expansion for \(f\) at each \(z_0\) and is continuous. Since there is only one Taylor expansion at each \(z_0\) the map \(\pi\) taking Taylor series centered at \(z_0\) to \(z_0\) is injective, \(\pi\) is continuous because for an open set \(V\) we have \(\pi^{-1}(V) = \bigsqcup_{z_0 \in V}\mathcal{H}f(z_0)\), which has open preimage under all of the \(\mathcal{H}f\). Since \(\mathcal{H}f\pi = 1_S, \pi \mathcal{H}f = 1_X\) we are done this step.

        Now \(\pi \circ \mathcal{H}f\vert_U = \iota_U\) and \(\mathcal{H}f\) continuous suffices to show that every holomorphic function is a section. Conversely, suppose \(g: X \to \mathcal{H}\) is not induced by a holomorphic function. The first case is \(g\) maps some \(z_1\) to a taylor expansion around \(z_0 \neq z_1\), this cannot be a section since then the diagram won't commute. In the second case, there are distinct points \(\set{z_\alpha}_{\alpha\in I}\) in the same connected component of \(U\) each Taylor series \(g(z_\alpha)\) determining a different holomorphic function (near that point) \(f_\alpha\), denote the set of points that determines \(f_\alpha\) as \(V_\alpha\), it is immediate that the \(V_\alpha\) are disjoint. We know that \(\mathcal{H} f_\alpha(U)\) is open in \(\mathcal{H}\) for each \(\alpha\) (\ref{check Holomorphic as sections exe}), but if \(g^{-1}(\mathcal{H} f_\alpha(U)) = V_\alpha\) is open for each \(\alpha \in I\), then \(U = \bigsqcup V_\alpha\) is not connected, violating our earlier assumption. Hence for some open \(\mathcal{H} f_\alpha(U)\) we have that \(g^{-1}(\mathcal{H} f_\alpha(U))\) is not open and \(g\) is not continuous. \qed

        \makeatletter\def\@currentlabel{Check!}\makeatother \label{check Holomorphic as sections exe}Check: If two holomorphic functions have the same Taylor series at a point they are equal so \[\mathcal{H}\phi^{-1} (\mathcal{H}f_\alpha(U)) = \begin{cases}
        U \tor \text{the domain of definition for }f &\phi = f_\alpha \\
        \emptyset &\text{else}
        \end{cases}\]
        In either case the preimage is an open set.
    \end{exe}
    \begin{exe}\label{Existence and uniqueness sections on stalks}
        \textbf{(Show the existence and uniqueness of sections for each element in the stalk)}
        Existence is not too bad, since \(\pi\) is a local homeomorphism, hence we can choose some neighborhood \(y \in U\) with \(\pi\vert_U\) a homeomorphism. Then define \(s: \pi(U) \to U\) via \(x \mapsto \pi\vert_U^{-1}(x)\). Now assume that \(s_1,s_2\) are two such sections, 
    \end{exe}
    \appendix
    
    \section{Assigned Readings}

    \section{Misc.}
    \begin{definition}
        A ring or algebra is called reduced when it has no non-zero nilpotents.
    \end{definition}
\end{document}