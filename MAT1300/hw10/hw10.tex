\documentclass[10.5pt]{article}
\usepackage{amsmath, amsfonts, amssymb,amsthm}
\usepackage[includeheadfoot]{geometry} % For page dimensions
\usepackage{fancyhdr}
\usepackage{enumerate} % For custom lists
\usepackage{tikz-cd}
\usepackage{graphicx}

\fancyhf{}
\lhead{MAT1300 hw10}
\rhead{Tighe McAsey - 1008309420}
\pagestyle{fancy}

% Page dimensions
\geometry{a4paper, margin=1in}

\theoremstyle{definition}
\newtheorem{pb}{}
\usepackage{tikz-cd, stackengine}

% Commands:

\newcommand{\set}[1]{\{#1\}}
\newcommand{\gen}[1]{\langle#1\rangle}
\newcommand{\abs}[1]{\left\vert#1\right\vert}
\newcommand{\norm}[1]{\lvert\lvert#1\rvert\rvert}
\newcommand{\tand}{\text{ and }}
\newcommand{\tor}{\text{ or }}
\newcommand{\pd}{\frac{\partial}{\partial x_j}}
\newcommand{\px}{\frac{\partial}{\partial x}}
\newcommand{\py}{\frac{\partial}{\partial y}}
\newcommand{\pz}{\frac{\partial}{\partial z}}
\newcommand{\ppx}{\frac{\partial^2}{\partial x^2}}
\newcommand{\ppy}{\frac{\partial^2}{\partial y^2}}
\newcommand{\ppz}{\frac{\partial^2}{\partial z^2}}
\newcommand{\hess}{\operatorname{Hess}}
\setcounter{MaxMatrixCols}{20}

\begin{document}
    \begin{pb} \textbf{(a)}
        Define \(F(x) = (\det\operatorname{Hess}_x(f))^2 + \sum_1^k \left(\pd f(x)\right)^2\), its clear that both the hessian and sum terms are non-negative (since they are squares of real values). Now assume first that \(F > 0\) on \(U\), if \(f\) has no critical points on \(U\) we are done. Now if \(p\) is a critical point for \(f\), we have that \(d_pf: \mathbb{R}^k \to \mathbb{R}\) is not surjective, where \(d_pf = \begin{pmatrix}
        \frac{\partial}{\partial x_1}f & \cdots & \frac{\partial}{\partial x_k}f
        \end{pmatrix}\), this matrix is of course surjective so long as atleast one \(\pd f \neq 0\), so at a critical point we get \(\pd f = 0\) for all \(j\) whence \(\sum_1^k \left(\pd f(p)\right)^2 = 0\), by our assumption of \(F > 0\), this implies that \((\det\operatorname{Hess}_p(f))^2 > 0\), and since \(p\) was an arbitrary critical point we can conclude that \(f\) is morse. Conversely, if \(f\) is morse, then by the computation above, at any regular value, \(p\), we must have some \(\pd f \neq 0\), which implies that \(\sum_1^k \left(\pd f(p)\right)^2 > 0\), which implies \(F(p) > 0\) by non-negativity of the hessian term. In the case that \(p\) is a critical point, we know that \(\det \hess_pf \neq 0\), so that \((\det\hess_pf)^2 > 0\), and \(F(p) > 0\), since all points are either regular values or critical points we are done. \qed

        \textbf{(b)} Let \(f\) be morse, and \(H\) a homotopy with \(H(x,0) = f(x)\), moreover we can denote \(f_t = H(-,t)\). Then since \(M\) is compact we can pick a finite covering by charts \((V_1,U_1,\phi_1),\hdots,(V_n,U_n,\phi_n)\). Now let \(\eta_i\) be a partition of unity suboordinate to these charts, we can define
        \begin{align*}
            F(x,t) = \sum_1^n \eta_i(x) \left((\det\hess_{\phi_i^{-1}(x)}f_t\circ\phi_i^{-1})^2 + \sum_{j=1}^k (\pd f_t\circ \phi_i^{-1}(x))^2\right)
        \end{align*}
        \(F\) is smooth, due to closure of smooth functions under sums, products and compositions as well as the fact that the hessian and partials vary smoothly with \(t\), this can be seen since the coordinates of these maps are a subset of the coordinates of \(H\circ (\phi^{-1},1_{[0,1]})\), which of course has smooth partials and hessian. To see that \(F(-,0) > 0\) on \(M\), note that since \(f_0\) is morse and \(\phi_i^{-1}\) are diffeomorphisms, each summand \(\left((\det\hess_{\phi_i^{-1}(x)}f_t\circ\phi_i^{-1})^2 + \sum_{j=1}^k (\pd f_t\circ \phi_i^{-1}(x))^2\right) > 0\) by part (a). Since \(M\) is compact, the function \(\hat{F}(t) = \inf_{x \in M}F(x,t)\) is continuous, and it has \(\hat{F}(0) > 0\), since \(F(-,0)\) is continuous and positive on a compact set (which implies it attains its infimum), then by continuity of \(\hat{F}\), there exists some \(\delta > 0\), such that \(t < \delta\) implies \(\hat{F}(t) > 0\). So to show that morse functions are generic, it will suffice to show that \(F(-,t) > 0\) implies that \(f_t\) is morse, since then we get for \(t < \delta\) each \(f_t = H(x,t)\) is morse. To check this, note that since the \(\phi_i^{-1}\) are diffeomorphisms, we have that \(\hess_xf_t\) is invertible iff \(\hess_{\phi_i^{-1}(x)}f_t\circ \phi_i^{-1}\) for each \(i\) (since we have a partition of unity we can always assume \(x \in V_i\)), moreover \(f_t\) is regular if and only if \(f_t \circ \phi_i\) is, so at critical points of \(f_t\), we get that for each \(i\), \(\sum_{j=1}^k (\pd f_t\circ \phi_i^{-1}(x))^2\) vanishes. It follows that \((\det\hess_{\phi_i^{-1}(x)}f_t\circ\phi_i^{-1})^2 > 0\) for all \(i\), when \(x\) is a critical value of \(f\), and since this Hessian is invertible iff the hessian of \(f_t\) is, we find that \(f_t\) has invertible Hessian at all its critical points, assuming \(F(-,t) > 0\) on \(M\) which completes the proof. \qed
    \end{pb}
    \begin{pb}
        \textbf{(a)} Let \(\Delta_{ij} = \set{(x_1,\hdots,x_r \mid x_i = x_j)}\), its clear that by reordering the factors we have each \(\Delta_{ij} \cong \Delta_{1,2}\), moreover \(\Delta_{12} = \Delta \times M^{r-2}\) where \(\Delta\) denotes the diagonal of \(M \times M\), where we take the convention \(M^{r-2} = \emptyset\) if \(r-2 = 0\). Since \(\Delta\) is a submanifold of \(M \times M\), we can realize it as the image of an embedding \(e: \Delta \hookrightarrow M \times M\), so that \(e\times 1_M^{r-2}: \Delta_{ij} \to M^r\) is an embedding and thus each \(\Delta_{ij}\) is a submanifold, it is clear from definition that \(\Delta = \bigcup_{1 \leq i < j \leq r} \Delta_{ij}\). Now suppose that \(X\) is compact \textcolor{red}{TODO}

        \textbf{(b)} Let \(p = (p_1,\hdots,p_r),q = (q_1,\hdots,q_r) \in \operatorname{Conf}_r(M)\), then we can identify these points with the same coordinates in \(M^r\). To see that there is a path between \(p\) and \(q\), we use path connectedness of \(M\), which gives paths \(\gamma'_i: [0,1] \to M\) satisfying \(\gamma'_i(0) = p_i \tand \gamma'_i(1) = q_i\). Now define
        \begin{align*}
            \gamma_1(t) &= (\gamma'_1(t),p_2,\hdots,p_r) \\
            \gamma_2(t) &= (q_1,\gamma_2'(t),p_3,\hdots,p_r) \\
            & \vdots \\
            \gamma_r(t) &= (q_1,\hdots,q_{r-1},\gamma_r'(t))
        \end{align*}
        We can join these paths together continuously (but not necessarily smoothly) by taking
        \begin{align*}
            \gamma(t) = \begin{cases}
                \gamma_1(rt) & t \in [0,\frac{1}{r}) \\
                \gamma_2(r(t - 1/r)) & t \in [\frac{1}{r},\frac{2}{r}) \\
                \vdots \\
                \gamma_r((t - \frac{r-1}{r})) & t \in [\frac{r-1}{r},1]
            \end{cases}
        \end{align*}
        This gives a continuous path between \(p\) and \(q\), which implies the existence of a smooth path, so replace \(\gamma\) with this smooth path. Now since \(\Delta_{ij} \cong \Delta \times M^{r-2}\), we have \(\dim M^r - \dim \Delta_{ij} \geq \dim M = k \geq 2\), and since \(X = [0,1]\) is dimension \(1\), we have a map \(f: X \to M\) is transverse to \(\Delta_{ij}\) iff \(\operatorname{Im} f \cap \Delta_{ij} = \emptyset\). So part (a) tells us that there is some homotopy \(H: [0,1]^2 \to M^r\) with \(H(-,0) = \gamma\), and for any \(\epsilon > 0\), there exists some \(t < \epsilon\) with \(H(-,t) \pitchfork \Delta_{ij}\) for all \(i,j\), but by the dimension argument I just gave, this means that \(H(-,t): [0,1] \to \operatorname{Conf}_r(M)\) for all such \(t\). Now let \(V_p\) and \(V_q\) be connected open sets in \(\operatorname{Conf}_r(M)\) containing \(p\) and \(q\) respectively (note since \(\operatorname{Conf}_r(M)\) is an open subset of \(M\) these can be identified with open subsets of \(M\)), since \(H\) is smooth, we have \(H(0,t) \in V_p\) for all \(t < \delta_p\) for some \(\delta_p > 0\), similarly for \(H(1,t) \in V_q\) we get some \(\delta_q\). Taking \(\delta = \min\set{\delta_p,\delta_q}\), we find that for \(t < \delta\) that \(H(0,t) \in V_p\), and \(H(1,t) \in V_q\), now use genericity to get some \(t\) with \(t < \delta\) and \(H(-,t)\) lying in \(\operatorname{Conf}_rM\), denote \(p' = H(0,t)\) and \(q' = H(1,t)\). Since connected components of manifolds are path connected, we have paths \(\gamma_p, \gamma_q\) in \(\operatorname{Conf}_rM\) connecting \(p\) to \(p'\) and \(q'\) to \(q\). It follows that concatenating gives a smooth path \(f\)
        \begin{align*}
            f(t) = \begin{cases}
                \gamma_p(3t) & t \in [0,1/3) \\
                \gamma(3(t - 1/3)) & t \in [1/3,2/3) \\
                \gamma(3(t-2/3)) & t \in [2/3,1]
            \end{cases}
        \end{align*}
        between \(p\) and \(q\) in \(\operatorname{Conf}_rM\), thus there exists a smooth path between \(p\) and \(q\) in \(\operatorname{Conf}_rM\), since \(p,q\) were arbitrary \(\operatorname{Conf}_rM\) is path connected.

        \textbf{(c)} Part (b) gives the existence of such a path provided that there is some \(q \in \operatorname{Conf}_rM \cap U\), i.e. We only need to check that \(\operatorname{Conf}_rM \cap U \neq \emptyset\). This is easy to see by Sard's theorem, since \((\operatorname{Conf}_rM)^c\) is a union of finitely many sets (the \(\Delta_{ij}\)) with dimension strictly less than \(k^r\). Now letting \(F: U \overset{\cong}{\longrightarrow} B_k\) be the diffeomorphism to the unit ball (and hence \(F^r : U^r \overset{\cong}{\longrightarrow} B_k\) given by \(F\) in each coordinate is a diffeomorphism). Since \(\Delta_{ij} \cap U^r\) is a submanifold of \(U^r\) (since \(U^r\) has full dimension they are automatically transverse, so their intersection is a submanifold), and once again since \(U^r\) has full dimension, we have \(\dim \Delta_{ij} = \dim \Delta_{ij} \cap U^r < \dim U^r\), we then get \(F^r(U^r \cap \Delta_{ij})\) is a submanifold of \(B_k^r\) of strictly smaller dimension, thus having measure \(0\) by sards theorem, this of course implies that 
        \[F^r(U^r \cap (\operatorname{Conf}_rM)^c)F^r(U^r \cap \bigcup_{1 \leq i<j \leq r}\Delta_{ij}) = F^r(\bigcup_{1 \leq i<j \leq r}U^r \cap \Delta_{ij}) = \bigcup_{1 \leq i<j \leq r}F^r(U^r \cap \Delta_{ij})\]
        is a finite union of measure zero sets, hence measure zero, so that since \(U^r\) has positive \(r\)-dimensional measure, there does indeed exist some \(x \in F^r(U^r \setminus (\operatorname{Conf}_rM)^c) = F^r(U^r \cap \operatorname{Conf}_rM)\), so that we can take \(q := F^{-1}(x) \in U^r \cap \operatorname{Conf}_rM\) as desired. \qed

        \textbf{(d)} We proved in the previous subpart, there is a path \(\gamma\) lying in \(\operatorname{Conf}_rM\) connecting \(p\) to some \(q \in U\), this allows us to define an isotopy by taking \(e_t(p_i)\) to be the \(i\)-th coordinate of \(\gamma(t)\), since \(e_t\) has a zero dimensional, compact domain we only need to check that its injective in order to being an embedding, but injectivity follows from \(\gamma(t) \in \operatorname{Conf}_rM\). Hence \(e_t\) is an isotopy of embedding between \(e_0 = 1_{\set{p_1,\hdots,p_r}}\) and \(e_1: p_i \mapsto q_i \in U\). The isotopy extension theorem gives an isotopy of compactly supported diffeomorphism \(h_t\), with \(h_0 = 1\), and \(h_te_0 = e_t\), so that \(h_1(p_i) = q_i \in U\). It follows that \(h_1^{-1}\) is a compactly supported diffeomorphism satisfying \(\set{p_1,\hdots,p_r} \in h_1^{-1}(U) \supset \set{h_1^{-1}(q_i)}_1^r\), moreover its isotopic to the identity via \(H(x,t) = h_{1 - t}^{-1}(x)\), since \(1_M^{-1} = 0\). \qed
    \end{pb}
\end{document}