\documentclass[10.5pt]{article}
\usepackage{amsmath, amsfonts, amssymb,amsthm}
\usepackage[includeheadfoot]{geometry} % For page dimensions
\usepackage{fancyhdr}
\usepackage{enumerate} % For custom lists
\usepackage{tikz-cd}
\usepackage{graphicx}

\fancyhf{}
\lhead{MAT1300 hw3}
\rhead{Tighe McAsey - 1008309420}
\pagestyle{fancy}

% Page dimensions
\geometry{a4paper, margin=1in}

\theoremstyle{definition}
\newtheorem{pb}{}
\usepackage{tikz-cd, stackengine}

% Commands:

\newcommand{\set}[1]{\{#1\}}
\newcommand{\gen}[1]{\langle#1\rangle}
\newcommand{\abs}[1]{\lvert#1\rvert}
\newcommand{\norm}[1]{\lvert\lvert#1\rvert\rvert}
\newcommand{\tand}{\text{ and }}
\newcommand{\tor}{\text{ or }}
\newcommand{\pd}{\frac{\partial}{\partial x_j}}

\begin{document}
    \begin{pb}
        \textbf{(a)} From the definition of a lie group we know that \(\mu: G\times G \to G\) is smooth, then \(\mu_g = \mu \circ \iota_g\), where \(\iota_g: G \to G \times G\) via \(h \mapsto (g,h)\) is the incusion into the product manifold, we have seen previously the inclusion is smooth, so that \(\mu_g = \mu \circ\iota_g\) is smooth. Now we can also see that
        \begin{align*}
            \mu_{g^{-1}}\mu_g = 1_G = \mu_g\mu_{g^{-1}}
        \end{align*}
        and \(\mu_{g^{-1}}\) is smooth for the same reason \(\mu_g\) is, so that \(\mu_g\) is in fact a diffeomorphism, this implies that \(d_e\mu_g\) is an isomorphism.

        \textbf{(b)} 
        
        \textbf{(Lemma)} Let \((\rho,E), (\widehat{\rho},\widehat{E})\) be two vector bundles on the same base space \(M\), and \(F: E \to \widehat{E}\) a smooth bijective map of smooth vector bundles with \(F(x,0) = (x,0)\) (i.e. \(F\) descends to the identity), then \(F\) is a diffeomorphism.
        \begin{proof}
            Being a diffeomorphism is a local property, so for a point \(x \in M\), let \(U\) be an open neighborhood of \(M\) where \(\rho^{-1}(U)\) admits a local trivialization \(\zeta\), moreover there is a second neighborhood \(x \in V \subset U\) where \(\widehat{\rho}^{-1}(V)\) admits a local trivialization \(\widehat{\zeta}\) (since the base manifold is the same by possibly shrinking the neighborhood we can assume that the two bundle charts are equal on \(V\times \set{0}\), this is not necessary but removes a lot of bloat from notation). Then \(\widehat{\zeta}\circ F\circ\zeta^{-1}: M \times \mathbb{R}^n \to M \times \mathbb{R}^n\) is smooth, linear on each fiber and bijective on each fiber, so on \(V\), we can write \(A(x) = \widehat{\zeta}\circ F\circ\zeta^{-1}(x,-)\). Then on the local trivialization \(F\) is given by
            \begin{align*}
                \widehat{\zeta}\circ F\circ\zeta^{-1}(x,v) = (x,A(x)v)
            \end{align*}
            In particular, the Jacobian \(D_{(x,v)} (\widehat{\zeta}\circ F\circ\zeta^{-1})\) is given by
            \begin{align*}
                \begin{pmatrix} 1_n & 0 \\ B(x,v) & A(x) \end{pmatrix}
            \end{align*}
            Bijectivity on each fiber implies that \(A(x)\) is full rank, so that \(\det (D_{(x,v)} \widehat{\zeta}\circ F\circ\zeta^{-1}) = \det A(x) \in \mathbb{R}^\times\), by the inverse function theorem \(\widehat{\zeta}\circ F\circ\zeta^{-1}\) has a local smooth inverse, and hence \(F\) is a diffeomorphism.
        \end{proof}
        
        Since \(T_eG\) is \(n\)-dimensional, we can identify it with \(\mathbb{R}^n\), the following diagram specifies the desired correspondence of vector bundles:
        \begin{equation*}
            \begin{tikzcd}
            G\times \mathbb{R}^n \arrow[d] \arrow[r,"F",shift left=1] &TG \arrow[l,"T", shift left = 1] \arrow[d] \\
            G \arrow[r, "1_G"] & G
            \end{tikzcd}
        \end{equation*}
        Where \(F(g,v) = (g,d_e\mu_g (v))\), 
        
        \emph{(when I originally solved the problem I tried to show \(F\) and the inverse map \(T\) which is not too hard to compute are both smooth, however, after trying to show \(F,T\) are smooth for quite some time I did the following computation which allowed me to see that \(F\) is smooth, this computation does not generalize easily to \(T\), so the lemma is intended to avoid having to do a similar computation for \(T\)).}

        In order to show \(F\) is smooth, it suffices to show that \((g,v) \mapsto d_e\mu_g(v)\) is smooth, here we can use smoothness of \(\mu\), and the identification \(T(G\times G) \longleftrightarrow TG \oplus TG\) by identifying on each fiber, we have previously computed (last homework) that \(d_p \iota_q = \begin{pmatrix}0 \\ 1\end{pmatrix}\) when \(\iota\) denotes inclusion. We have that \(d\mu: T(G\times G) \to TG\) is smooth since \(\mu\) is a smooth map, then
        \begin{align*}
            d\mu((g,v),(h,u)) &= d_{(g,h)}\mu(v,u) \\
            d_e \mu_g = d_e (\mu\circ\iota_g) (v,u) &= (d_{(g,e)}\mu) (d_e\iota_g)(v,u) = d_{(g,e)}\mu(u)
        \end{align*}
        From this computation, we can see that \(d_e \mu_g = d_{(g,e)}\mu(0,u)\) is the restriction of \(d\mu\) to \(G \times \set{0} \times T_eG\), this is clearly a submanifold directly from the definition of it being a linear subspace given by inclusion on the first \(n\) coordinates and last \(n\) coordinates. Thus the restriction of \(d\mu\) to this submanifold is smooth, and is identified with \(d_e \mu_g\). So \(F\) is smooth, and we appeal to the lemma to find that \(T\), the set theoretic inverse for \(F\) is smooth and hence \(F\) is a diffeomorphism.
    \end{pb}
    \begin{pb}
        Let \(f: X \to \mathbb{R}^m\) be a submersion, where \(X\) is a compact smooth manifold. The proof will follow if we can show submersions are open maps, assuming this, since the image of a compact set is compact (by pulling back an open cover along the map) we get that \(f(X) \subset \mathbb{R}^m\) is open, but also \(f(X) \subset \mathbb{R}^m\) is compact hence closed, so since \(X \neq \emptyset\) we have \(f(X) = \mathbb{R}^m\), contradicting compactness.

        It remains to show that a submersion is open, since \(f\) is a submersion, we can cover \(M,N\) with charts \((U_\alpha,V_\alpha,\phi_\alpha)\) and \((U'_\beta,V'_\beta,\varphi_\beta)\) respectively with the property that the following commutes (here \(\pi\) is the projection map onto the first \(n\) coordinates)
        \begin{equation*}
            \begin{tikzcd}
                U_\alpha \arrow[d,"\pi"] \arrow[r,"\phi_\alpha"] &V_\alpha \arrow[d,"f"]\\
                U_\beta' \arrow[r,"\varphi_\beta"] &V_\beta'
            \end{tikzcd}
        \end{equation*}
        Now let \(E \subset X\) be open, and write \(E_\alpha := V_\alpha \cap E\), then
        \begin{align*}
            f(E) = \bigcup_\alpha f(E_\alpha) = \bigcup_{\alpha,\beta} \varphi_\beta\pi\phi_\alpha^{-1}(E_\alpha)
        \end{align*}
        But \(\varphi_\beta\pi\phi_\alpha^{-1}\) is a composition of open maps hence open, so that \(f(E)\) is open which suffices to show \(f\) is open.
    \end{pb}
    \begin{pb}
        \textbf{(a)} From the iterated construction we get
        \begin{align*}
            \overline{[a,b,c,d,e,f,g,h]} &= [\overline{(a,b,c,d,e,f)},(-g,-h)] = [\overline{(a,b,c,d)},(-e,-f),(-g,-h)] \\
            &= [\overline{(a,b)},(-c,-d),(-e,-f),(-g,-h)] = [\overline{a}, -b, (-c,-d),(-e,-f),(-g,-h)]
        \end{align*}
        from this taking the \(i\)-th coordinate to be \(1\) and the others zero we see \(\overline{e_i} = -e_i\). \qed

        \textbf{(b)} We first note that \(A\) maps real values to real values and imaginary values to imaginary values. This can be seen since if \(\overline{x} = x \tand \overline{y} = -y\), then
        \begin{align*}
            &\overline{A(x)} = A(\overline{x}) = A(x) &\overline{A(y)} = A(\overline{y}) = A(-y) = -A(y)
        \end{align*}
        And moreover, for arbitrary \(x\), \(A(x)A(1) = A(x)\) implies \(A(1) = 1\) so linearity suffices to show that \(A(\alpha) = \alpha\) for \(\alpha \in \mathbb{R}\), once again by linearity we see that this implies \(\text{Re}(A(x)) = A(\text{\text{Re}(x)})\) for all \(x\). It follows that \(A\) is orthogonal, i.e. preserves the inner product
        \begin{align*}
            \gen{A (x), A(y)} = \text{Re}(A(x)\overline{A(y)}) = \text{Re}(A(x)A(\overline{y})) = \text{Re}(A(x \overline{y})) = \text{Re}(x \overline{y}) = \gen{x,y}
        \end{align*}
        Since \(A\) preserves the imaginary octonions, it makes sense to restrict \(A\) to acting on \(\text{Im}(\mathbb{O})\), so that identifying \(A\) with its image in \(O(7)\) is well defined, since the inner product on \(\text{Im}(\mathbb{O})\) induces the standard norm on \(\mathbb{R}^7\) (computation provided below), we know from the polarization identity that the inner products are the same since they can be recovered from the norm, so that the image of \(A\) in \(O(7)\) is still orthogonal in the euclidean sense.

        I include here the computation of equivalence of norms using the multiplication table:
        Consider the octonion given by \(a = (a_ie_i)_0^7\), of course we are only interested in the case of \(a_0 = 0\), so we have \(a = (a_ie_i)_1^7\)
        \begin{align*}
            a \overline{a} = \left(\sum_1^7 a_i e_i\right)\left(\sum_1^7-a_je_j\right) = \sum_1^7 a_i^2 + \sum_{i < j} a_ia_j e_ie_j + \sum_{i > j}a_ia_j e_ie_j
        \end{align*}
        We can read from the off diagonal of the octonions multiplication table that for \(i,j > 0\) and \(i \neq j\) that \(e_ie_j = -e_je_i\), this kills the two sums on the right to give us \(\sum_1^7 a_i^2\) the euclidean norm as desired. \qed

        \textbf{(c)} From the multiplication table we have for any \(i\), \(e_i \overline{e_i} = e_0\), it follows from definitions that \(\gen{e_i,e_i} = 1\) for \(i = 1,2,4\). Now reading from the table,
        \begin{align*}
            e_1 \overline{e_2} = -e_3, \quad e_1 \overline{e_4} = - e_5, \quad e_2 \overline{e_4} = -e_6, \quad (e_1e_2)\overline{e_4} = -e_7
        \end{align*}
        these all have zero real part, so that by taking the inner product we get zero, this suffices to show its a special triple. \qed

        \textbf{(d)} 
        
        Although not stated in the question, \(V_n(\mathbb{R}^m)\) is orthonormal sets of \(n\) vectors in \(\mathbb{R}^m\), it is also defined in the notes as a quotient of the orthogonal group by a group action \(O(m)/O(m-n)\). It follows that \(V_3(\mathbb{R}^7)\) can be realized as \(O(7)/O(4)\), since \(O(4)\) is a lie group of dimension \(6\), and \(O(7)\) is a lie group of dimension \(21\), the quotient \(V_3(\mathbb{R}^7)\) is a \(21 - 6 = 15\) dimensional manifold, its also important here that we are identifying \(O(4)\) as the elements fixing the first second and fourth columns \(e_1,e_2,e_4\), taking these columns in particular is important for the projection to work out in part (e).

        The only additional condition of a special triple that isnt in \(V_3(\mathbb{R}^7)\) is the equation \(\gen{xy,z} = 0\), so it will suffice to check that \(0\) is a regular value for
        \begin{align*}
            \hat{F}: V_3(\mathbb{R}^7) &\to \mathbb{R} \\
            (x,y,z) &\mapsto \gen{xy,z}
        \end{align*}
        where again we are using the identification of the imaginary octonions with \(\mathbb{R}^7\). The smoothness of \(\hat{F}\) is immediate by multilinearity. It is somewhat hard to deal with \(V_3(\mathbb{R}^7)\), but we can show that it is a submanifold of \(\mathbb{R}^{21}\), to do so use the regular value theorem with
        \begin{align*}
            F': \text{Mat}_{3\times 7}(\mathbb{R}) &\to \mathbb{R}^6 \\
            \begin{pmatrix} v_1&v_2&v_3 \end{pmatrix} &\mapsto (\norm{v_1}^2,\norm{v_2}^2,\norm{v_3}^2,\gen{v_1,v_2},\gen{v_1,v_3},\gen{v_2,v_3})
        \end{align*}
        \(F'\) is polynomial hence smooth. The total derivative looks like (in \(1 \times 7\) blocks)
        \begin{align*}
            d_{(v_1,v_2,v_3)}F' = \begin{pmatrix} 2v_1 & 0 & 0 \\ 0 & 2v_2 & 0 \\ 0&0& 2v_3 \\ v_2 & v_1 & 0 \\ v_3 & 0 & v_1 \\ 0 & v_3 & v_2 \end{pmatrix}
        \end{align*}
        it is straightforward to see that \((1,1,1,0,0,0)\) is a regular value for \(F'\) by the vectors being nonzero at these points (independence of rows then follows easily from orthogonality). This realizes \(V_3(\mathbb{R}^7)\) as a \(15 = 21 - 6\) dimensional manifold \(F'^{-1}\set{(1,1,1,0,0,0)}\). Now we want to show that \(d_{(e_1,e_2,e_4)}\hat{F}\) is surjective, we will check later that using the action of \(G_2\) this gives surjectivity at all points. In order to avoid overloading notation on \(e_i\) I will use \((f_i)_1^{21}\) to denote the basis on \(\mathbb{R}^{21}\). Then we can define the path \(\gamma: (-\epsilon,\epsilon) \to V_3(\mathbb{R}^7)\) via \(\gamma(t) = (e_1,e_2,e_3\sin t + e_4\cos t)\), it is clear by definition that \(\gamma(-\epsilon,\epsilon)\) does indeed lie in \(V_3(\mathbb{R}^7)\), moreover we have \(\gamma(0) = (e_1,e_2,e_4)\) and \(\gamma'(t) = (0,0,e_3) = f_{17} \in T_{(e_1,e_2,e_4)}V_3(\mathbb{R}^7)\). Now we can take 
        \begin{align*}
            \left.\frac{d}{dt}\right\vert_{t=0}F\circ\gamma(t) = \left.\frac{d}{dt}\right\vert_{t=0} \gen{e_1e_2,e_3\sin t + e_4\cos t} = \left.\frac{d}{dt}\right\vert_{t=0}e_3(\overline{e_3}\cos t + \overline{e_3}\sin t) = \left.\frac{d}{dt}\right\vert_{t=0} \cos t + \sin t = 1
        \end{align*}
        so that indeed we have \(d_{(e_1,e_2,e_4)}\hat{F} = d_{(e_1,e_2,e_4)}F\vert_{T_{(e_1,e_2,e_4)}V_3(\mathbb{R}^7)}\) is nonzero, hence surjective. This gives that \(\hat{F}\) is a submersion at the point \((e_1,e_2,e_4)\), we need to check it for the rest of \(X = \hat{F}^{-1}(0)\), here we can use the theorem that for any other point \((x,y,z) \in X\) we have some \(A \in G_2\) with \(A(e_1,e_2,e_4) = (x,y,z)\). It follows that 

        \textbf{(e)} Define the map using the theorem by defining \(A(x,y,z)\) to be the unique map transforming \((e_1,e_2,e_4) \mapsto (x,y,z)\)
        \begin{align*}
            \Phi: X &\to O(7) \\
            (x,y,z) &\mapsto A_{(x,y,z)}
        \end{align*}
        That \(\Phi\) is injective is an immediate consequence of the theorem, surjectivity onto \(G_2\) is also straightforward, since \(G_2\) preserves norms, products and inner products (which are all of the special triple conditions) so all elements of \(G_2\) send special triples to special triples, whence \(G_2\) elements are all in the image of \(X\), determined by their action on \(e_1,e_2,e_4\). We can also take \(O(7)\) now with respect to the basis given by the special triple \((e_1,e_2,e_4)\), and note that this just amounts to flipping the signs on basis elements \(e_5,e_6\) and \(e_7\), the reason for doing this is to clean up notation that \(\Phi(e_1,e_2,e_4) = 1\). We also have that \(A\) is multilinear hence smooth. We check that it indeed defines an imersion, to do so consider the quotient map induced by the action of \(O(4)\) given by \(\pi: O(7) \to V_3{\mathbb{R}^7}\), then \(\pi\) is smooth, and \(\pi\circ \Phi = 1_X\) (recall that \(X\) is a submanifold, then the composition maps into it), it follows that at any special triple we have
        \begin{align*}
            (d_{A(x,y,z)}\pi)(d_{(x,y,z)}\Phi) = 1_{T_{(x,y,z)}X}
        \end{align*}
        injectivity of \(1_{T_{(x,y,z)}X}\) implies injectivity of \(d_{(x,y,z)}\Phi\), so that \(\Phi\) is an immersion. Finally it only remains to check that \(\Phi\) is proper, but this follows immediately from \(X\) compact. To see that \(X\) is compact, we can identify \(X \subset \overline{B_{0,1}(\mathbb{R}^{21})}\), where \(X\) is given by the intersection of zero-loci of polynomial equations. This realizes \(X\) as a closed compact subset of Euclidean space, hence compact. \qed

    \end{pb}
\end{document}