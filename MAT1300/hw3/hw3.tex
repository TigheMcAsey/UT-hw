\documentclass[10.5pt]{article}
\usepackage{amsmath, amsfonts, amssymb,amsthm}
\usepackage[includeheadfoot]{geometry} % For page dimensions
\usepackage{fancyhdr}
\usepackage{enumerate} % For custom lists
\usepackage{tikz-cd}
\usepackage{graphicx}

\fancyhf{}
\lhead{MAT1300 hw3}
\rhead{Tighe McAsey - 1008309420}
\pagestyle{fancy}

% Page dimensions
\geometry{a4paper, margin=1in}

\theoremstyle{definition}
\newtheorem{pb}{}
\usepackage{tikz-cd, stackengine}

% Commands:

\newcommand{\set}[1]{\{#1\}}
\newcommand{\gen}[1]{\langle#1\rangle}
\newcommand{\abs}[1]{\lvert#1\rvert}
\newcommand{\norm}[1]{\lvert\lvert#1\rvert\rvert}
\newcommand{\tand}{\text{ and }}
\newcommand{\tor}{\text{ or }}
\newcommand{\pd}{\frac{\partial}{\partial x_j}}

\begin{document}
    \begin{pb}
        \textbf{(a)} From the definition of a lie group we know that \(\mu: G\times G \to G\) is smooth, then \(\mu_g = \mu \circ \iota_g\), where \(\iota_g: G \to G \times G\) via \(h \mapsto (g,h)\) is the incusion into the product manifold, we have seen previously the inclusion is smooth, so that \(\mu_g = \mu \circ\iota_g\) is smooth. Now we can also see that
        \begin{align*}
            \mu_{g^{-1}}\mu_g = 1_G = \mu_g\mu_{g^{-1}}
        \end{align*}
        and \(\mu_{g^{-1}}\) is smooth for the same reason \(\mu_g\) is, so that \(\mu_g\) is in fact a diffeomorphism, this implies that \(d_e\mu_g\) is an isomorphism.

        \textbf{(b)} 
        
        \textbf{(Lemma)} Let \((\rho,E), (\widehat{\rho},\widehat{E})\) be two vector bundles on the same base space \(M\), and \(F: E \to \widehat{E}\) a smooth bijective map of smooth vector bundles with \(F(x,0) = (x,0)\) (i.e. \(F\) descends to the identity), then \(F\) is a diffeomorphism.
        \begin{proof}
            Being a diffeomorphism is a local property, so for a point \(x \in M\), let \(U\) be an open neighborhood of \(M\) where \(\rho^{-1}(U)\) admits a local trivialization \(\zeta\), moreover there is a second neighborhood \(x \in V \subset U\) where \(\widehat{\rho}^{-1}(V)\) admits a local trivialization \(\widehat{\zeta}\) (since the base manifold is the same by possibly shrinking the neighborhood we can assume that the two bundle charts are equal on \(V\times \set{0}\), this is not necessary but removes a lot of bloat from notation). Then \(\widehat{\zeta}\circ F\circ\zeta^{-1}: M \times \mathbb{R}^n \to M \times \mathbb{R}^n\) is smooth, linear on each fiber and bijective on each fiber, so on \(V\), we can write \(A(x) = \widehat{\zeta}\circ F\circ\zeta^{-1}(x,-)\). Then on the local trivialization \(F\) is given by
            \begin{align*}
                \widehat{\zeta}\circ F\circ\zeta^{-1}(x,v) = (x,A(x)v)
            \end{align*}
            In particular, the Jacobian \(D_{(x,v)} (\widehat{\zeta}\circ F\circ\zeta^{-1})\) is given by
            \begin{align*}
                \begin{pmatrix} 1_n & 0 \\ B(x,v) & A(x) \end{pmatrix}
            \end{align*}
            Bijectivity on each fiber implies that \(A(x)\) is full rank, so that \(\det (D_{(x,v)} \widehat{\zeta}\circ F\circ\zeta^{-1}) = \det A(x) \in \mathbb{R}^\times\), by the inverse function theorem \(\widehat{\zeta}\circ F\circ\zeta^{-1}\) has a local smooth inverse, and hence \(F\) is a diffeomorphism.
        \end{proof}
        
        Since \(T_eG\) is \(n\)-dimensional, we can identify it with \(\mathbb{R}^n\), the following diagram specifies the desired correspondence of vector bundles:
        \begin{equation*}
            \begin{tikzcd}
            G\times \mathbb{R}^n \arrow[d] \arrow[r,"F",shift left=1] &TG \arrow[l,"T", shift left = 1] \arrow[d] \\
            G \arrow[r, "1_G"] & G
            \end{tikzcd}
        \end{equation*}
        Where \(F(g,v) = (g,d_e\mu_g (v))\), 
        
        \emph{(when I originally solved the problem I tried to show \(F\) and the inverse map \(T\) which is not too hard to compute are both smooth, however, after trying to show \(F,T\) are smooth for quite some time I did the following computation which allowed me to see that \(F\) is smooth, this computation does not generalize easily to \(T\), so the lemma is intended to avoid having to do a similar computation for \(T\)).}

        In order to show \(F\) is smooth, it suffices to show that \((g,v) \mapsto d_e\mu_g(v)\) is smooth, here we can use smoothness of \(\mu\), and the identification \(T(G\times G) \longleftrightarrow TG \oplus TG\) by identifying on each fiber, we have previously computed (last homework) that \(d_p \iota_q = \begin{pmatrix}0 \\ 1\end{pmatrix}\) when \(\iota\) denotes inclusion. We have that \(d\mu: T(G\times G) \to TG\) is smooth since \(\mu\) is a smooth map, then
        \begin{align*}
            d\mu((g,v),(h,u)) &= d_{(g,h)}\mu(v,u) \\
            d_e \mu_g = d_e (\mu\circ\iota_g) (v,u) &= (d_{(g,e)}\mu) (d_e\iota_g)(v,u) = d_{(g,e)}\mu(u)
        \end{align*}
        From this computation, we can see that \(d_e \mu_g = d_{(g,e)}\mu(0,u)\) is the restriction of \(d\mu\) to \(G \times \set{0} \times T_eG\), this is clearly a submanifold directly from the definition of it being a linear subspace given by inclusion on the first \(n\) coordinates and last \(n\) coordinates. Thus the restriction of \(d\mu\) to this submanifold is smooth, and is identified with \(d_e \mu_g\). So \(F\) is smooth, and we appeal to the lemma to find that \(T\), the set theoretic inverse for \(F\) is smooth and hence \(F\) is a diffeomorphism.
    \end{pb}
    \begin{pb}
        Let \(f: X \to \mathbb{R}^m\) be a submersion, where \(X\) is a compact smooth manifold. The proof will follow if we can show submersions are open maps, assuming this, since the image of a compact set is compact (by pulling back an open cover along the map) we get that \(f(X) \subset \mathbb{R}^m\) is open, but also \(f(X) \subset \mathbb{R}^m\) is compact hence closed, so since \(X \neq \emptyset\) we have \(f(X) = \mathbb{R}^m\), contradicting compactness.

        It remains to show that a submersion is open, since \(f\) is a submersion, we can cover \(M,N\) with charts \((U_\alpha,V_\alpha,\phi_\alpha)\) and \((U'_\beta,V'_\beta,\varphi_\beta)\) respectively with the property that the following commutes (here \(\pi\) is the projection map onto the first \(n\) coordinates)
        \begin{equation*}
            \begin{tikzcd}
                U_\alpha \arrow[d,"\pi"] \arrow[r,"\phi_\alpha"] &V_\alpha \arrow[d,"f"]\\
                U_\beta' \arrow[r,"\varphi_\beta"] &V_\beta'
            \end{tikzcd}
        \end{equation*}
        Now let \(E \subset X\) be open, and write \(E_\alpha := V_\alpha \cap E\), then
        \begin{align*}
            f(E) = \bigcup_\alpha f(E_\alpha) = \bigcup_{\alpha,\beta} \varphi_\beta\pi\phi_\alpha^{-1}(E_\alpha)
        \end{align*}
        But \(\varphi_\beta\pi\phi_\alpha^{-1}\) is a composition of open maps hence open, so that \(f(E)\) is open which suffices to show \(f\) is open.
    \end{pb}
    \begin{pb}
        \textbf{(a)} From the iterated construction we get
        \begin{align*}
            \overline{[a,b,c,d,e,f,g,h]} &= [\overline{(a,b,c,d,e,f)},(-g,-h)] = [\overline{(a,b,c,d)},(-e,-f),(-g,-h)] \\
            &= [\overline{(a,b)},(-c,-d),(-e,-f),(-g,-h)] = [\overline{a}, -b, (-c,-d),(-e,-f),(-g,-h)]
        \end{align*}
        from this taking the \(i\)-th coordinate to be \(1\) and the others zero we see \(\overline{e_i} = -e_i\). \qed

        \textbf{(b)} We first note that \(A\) maps real values to real values and imaginary values to imaginary values. This can be seen since if \(\overline{x} = x \tand \overline{y} = -y\), then
        \begin{align*}
            &\overline{A(x)} = A(\overline{x}) = A(x) &\overline{A(y)} = A(\overline{y}) = A(-y) = -A(y)
        \end{align*}
        And moreover, for arbitrary \(x\), \(A(x)A(1) = A(x)\) implies \(A(1) = 1\) so linearity suffices to show that \(A(\alpha) = \alpha\) for \(\alpha \in \mathbb{R}\), once again by linearity we see that this implies \(\text{Re}(A(x)) = A(\text{\text{Re}(x)})\) for all \(x\). It follows that \(A\) is orthogonal, i.e. preserves the inner product
        \begin{align*}
            \gen{A (x), A(y)} = \text{Re}(A(x)\overline{A(y)}) = \text{Re}(A(x)A(\overline{y})) = \text{Re}(A(x \overline{y})) = \text{Re}(x \overline{y}) = \gen{x,y}
        \end{align*}
        Since \(A\) preserves the imaginary octonions, it makes sense to restrict \(A\) to acting on \(\text{Im}(\mathbb{O})\), so that identifying \(A\) with its image in \(O(7)\) is well defined, since the inner product on \(\text{Im}(\mathbb{O})\) induces the standard norm on \(\mathbb{R}^7\) (computation provided below), we know from the polarization identity that the inner products are the same since they can be recovered from the norm, so that the image of \(A\) in \(O(7)\) is still orthogonal in the euclidean sense.

        I include here the computation of equivalence of norms using the multiplication table:
        Consider the octonion given by \(a = (a_ie_i)_0^7\), of course we are only interested in the case of \(a_0 = 0\), so we have \(a = (a_ie_i)_1^7\)
        \begin{align*}
            a \overline{a} = \left(\sum_1^7 a_i e_i\right)\left(\sum_1^7-a_je_j\right) = \sum_1^7 a_i^2 + \sum_{i < j} a_ia_j e_ie_j + \sum_{i > j}a_ia_j e_ie_j
        \end{align*}
        We can read from the off diagonal of the octonions multiplication table that for \(i,j > 0\) and \(i \neq j\) that \(e_ie_j = -e_je_i\), this kills the two sums on the right to give us \(\sum_1^7 a_i^2\) the euclidean norm as desired. \qed

        \textbf{(c)} From the multiplication table we have for any \(i\), \(e_i \overline{e_i} = e_0\), it follows from definitions that \(\gen{e_i,e_i} = 1\) for \(i = 1,2,4\). Now reading from the table,
        \begin{align*}
            e_1 \overline{e_2} = -e_3, \quad e_1 \overline{e_4} = - e_5, \quad e_2 \overline{e_4} = -e_6, \quad (e_1e_2)\overline{e_4} = -e_7
        \end{align*}
        these all have zero real part, so that by taking the inner product we get zero, this suffices to show its a special triple. \qed

        \textbf{(d)} Although not explicitly stated in the question, in the lecture notes problems we have \(V_n(\mathbb{R}^m)\) is orthonormal sets of \(n\) vectors in \(\mathbb{R}^m\), to see that this is indeed a manifold we can use the regular value theorem with
        \begin{align*}
            F': \text{Mat}_{3\times 7}(\mathbb{R}) &\to \mathbb{R}^6 \\
            \begin{pmatrix} v_1&v_2&v_3 \end{pmatrix} &\mapsto (\norm{v_1}^2,\norm{v_2}^2,\norm{v_3}^2,\gen{v_1,v_2},\gen{v_1,v_3},\gen{v_2,v_3})
        \end{align*}
        \(F'\) is polynomial hence smooth. The total derivative looks like (in \(1 \times 7\) blocks)
        \begin{align*}
            d_{(v_1,v_2,v_3)}F = \begin{pmatrix} 2v_1^i & 0 & 0 \\ 0 & 2v_2^i & 0 \\ 0&0& 2v_3^i \\ v_2^i & v_1^i & 0 \\ v_3^i & 0 & v_1^i \\ 0 & v_3^i & v_2^i \end{pmatrix}
        \end{align*}
        it is straightforward to see that \((1,1,1,0,0,0)\) is a regular value for \(F'\) by the vectors being nonzero at these points (independence of rows then follows easily from orthogonality). This realizes \(V_3(\mathbb{R}^7)\) as a \(15 = 21 - 6\) dimensional manifold \(F'^{-1}\set{(1,1,1,0,0,0)}\). The only additional condition of a special triple that isnt in \(V_3(\mathbb{R}^7)\) is the equation \(\gen{xy,z} = 0\), so it will suffice to check that \(0\) is a regular value for
        \begin{align*}
            F: V_3(\mathbb{R}^7) &\to \mathbb{R} \\
            (x,y,z) &\mapsto \gen{xy,z}
        \end{align*}
        where again we are using the identification of the imaginary octonions with \(\mathbb{R}^7\), also recognize that \(F\) is polynomial hence smooth. To simplify things we can factor through \(V_2(\mathbb{R}^8)\) by taking the composition \((x,y,z) \overset{m}{\mapsto} (xy,z') \overset{\gen{}}{\mapsto} \gen{xy,z'}\) the \(z \mapsto z'\) just denotes appending a zero leading coordinate, i.e. \(z' = (0,z_1,\hdots,z_7)\), everything here is still smooth due to being polynomial in coordinates. Thus \(d_{(x,y,z)}F = (d_{(xy,z')}\gen{})(d_{(x,y,z)}m)\), since all the spaces we are dealing with are just submanifolds of \(\mathbb{R}^N\) for some \(N\) (we can identify matrices with their components) We can identify the space with its chart, and the derivative with the Jacobian on charts, this allows us to readily write down the derivatives for \(m, \gen{}\).
        \begin{align*}
            &d_{(x,y,z)}m = \begin{bmatrix} A(x,y) & 0 \\ 0 & 1_{7 \times 7} \end{bmatrix}
        \end{align*}
        (written in block form) Then to conclude that the derivative is onto, it just needs to be nonzero, so it will suffice to check that one of the last seven entries of \(d_{(xy,z)} \gen{}\) is non-zero, but in the coordinates the inner product is given by \((a,b) \mapsto (a_0b_0,\hdots,a_7b_7)\), it follows the Jacobian at \((xy,z')\) is given by \[\begin{pmatrix} 0&z_1&\cdots&z_7 &(xy)_0 &\cdots &(xy)_7 \end{pmatrix}\]
        Since \(\gen{x,y} = 0\), and \(\norm{x}^2 = \norm{y}^2 = 1\) we know that \(xy\) is nonzero (we are working in a division algebra) and \(xy\) is not real, since otherwise \(0 = \gen{x,-y} = \gen{x, \overline{y}} = \text{Re}(xy)\) which would contradict \(xy \neq 0\). It follows that one of \((xy)_j\) is nonzero for \(1 \leq j \leq 7\), but this suffices to prove that \(d_{(x,y,z)}F\) has full rank of \(1\) at \((x,y,z)\). Hence we can apply the regular value theorem so that \(F^{-1}\set{0} = X\) is a submanifold of \(V_3(\mathbb{R}^7)\). \qed
    \end{pb}
\end{document}