\documentclass[10.5pt]{article}
\usepackage{amsmath, amsfonts, amssymb,amsthm}
\usepackage[includeheadfoot]{geometry} % For page dimensions
\usepackage{fancyhdr}
\usepackage{enumerate} % For custom lists
\usepackage{tikz-cd}
\usepackage{graphicx}

\fancyhf{}
\lhead{MAT1300 hw9}
\rhead{Tighe McAsey - 1008309420}
\pagestyle{fancy}

% Page dimensions
\geometry{a4paper, margin=1in}

\theoremstyle{definition}
\newtheorem{pb}{}
\usepackage{tikz-cd, stackengine}

% Commands:

\newcommand{\set}[1]{\{#1\}}
\newcommand{\gen}[1]{\langle#1\rangle}
\newcommand{\abs}[1]{\left\vert#1\right\vert}
\newcommand{\norm}[1]{\lvert\lvert#1\rvert\rvert}
\newcommand{\tand}{\text{ and }}
\newcommand{\tor}{\text{ or }}
\newcommand{\pd}{\frac{\partial}{\partial x_j}}
\newcommand{\px}{\frac{\partial}{\partial x}}
\newcommand{\py}{\frac{\partial}{\partial y}}
\newcommand{\pz}{\frac{\partial}{\partial z}}
\newcommand{\ppx}{\frac{\partial^2}{\partial x^2}}
\newcommand{\ppy}{\frac{\partial^2}{\partial y^2}}
\newcommand{\ppz}{\frac{\partial^2}{\partial z^2}}
\setcounter{MaxMatrixCols}{20}
\tikzset{
  curarrow/.style={
  rounded corners=8pt,
  execute at begin to={every node/.style={fill=red}},
    to path={-- ([xshift=-50pt]\tikztostart.center)
    |- (#1)  {}
    -| ([xshift=50pt]\tikztotarget.center)
    -- (\tikztotarget)}
    }
}

\begin{document}
    \begin{pb}
        I will first prove a lemma, since I will use it multiple times in order to prove homotopy equivalences.

        \textbf{Lemma.} If \(e: X \hookrightarrow M\) is an embedding for manifolds \(M,X\), and there is a strong deformation retract \(H: M \to M\) with \(H(M\times\set{1}) = X\), then \(M \simeq X\).
        \begin{proof}
            Let \(r(x) = H(x,1)\), then \(e\vert_{e(X)}^{-1}r: M \to X\) is smooth, and since \(H\) is a strong deformation retract we have \(e\vert_{e(X)}^{-1}r e = 1_X\), from which it suffices to show that \(e e\vert_{e(X)}^{-1}r = r \simeq 1_M\), but \(r = H(-,1)\), so this homotopy is exhibited by \(H\) and we are done.
        \end{proof}

        Let \(V_0,\hdots,V_n\) be the standard charts on \(\mathbb{RP}^n\), now take \(V = V_0\), and let \(\text{pt.} = [0:0:\cdots:1] \in V_0^c\), then take \(U = \mathbb{RP}^n\setminus \set{\text{pt.}}\), the standard chart map \(\phi_0\) gives us \(V \cong \mathbb{R}^n\). Similarly, we find that \(U \cap V = V \setminus \set{\text{pt.}}\), so that
        \begin{align*}
            \phi_0^{-1}\vert_{U \cap V}: U \cap V \overset{\cong}{\longrightarrow} \mathbb{R}^n \setminus \set{\phi_0^{-1}(\text{pt.})} \simeq S^{n-1}
        \end{align*}
        The homotopy equivalence is given by \(\mathbb{R}^n \setminus \set{\phi_0^{-1}(\text{pt.})} \overset{\cong}{\longrightarrow} \mathbb{R}^{n} \setminus \set{0}\) via \(x \mapsto x - \phi_0^{-1}(\text{pt.})\), then taking the strong deformation retract \(H(x,t) = (1-t)x + t\frac{x}{\norm{x}}\) which gives a homotopy equivalence to \(S^{n-1}\). Now it remains to show \(U \simeq \mathbb{RP}^{n-1}\). First we consider the smooth map \(\theta: \mathbb{RP}^n\setminus \set{\text{pt.}} \to (0,\pi/2)\) via \([x_0,\hdots,x_n] \mapsto \arcsin x_n\), where we take the representative of \([x_0,\hdots,x_n]\) with \(x_n > 0\), we can do this since we removed the point \(x_n = 0\), and smoothness follows by \(\arcsin\) being smooth on \([0,1)\), so our map is smooth in coordinates, it follows that points in \(\mathbb{RP}^n \setminus \set{\text{pt.}}\), now we can define the homotopy (where once again we define the maps on the representative with \(x_n > 0\))
        \begin{align*}
            H([x],t) = \left[\cos((1-t)\theta(x))\frac{(x_0,\hdots,x_{n-1},0)}{\norm{(x_0,\hdots,x_{n-1},0)}} + \sin\theta(x)\right]
        \end{align*}
        Once again, this map is smooth since it is defined to be smooth on coordinates, and \(H(\mathbb{RP}^n\times \set{1}) = \set{[x] \in \mathbb{RP} \mid x_n = 0} \cong \mathbb{RP}^{n-1}\), where the diffeomorphism is given by the embedding \(\mathbb{RP}^{n-1} \hookrightarrow \mathbb{RP}^n\) via \([x_0,\hdots,x_{n-1}] \mapsto [x_0,\hdots,x_{n-1},0]\), this map is smooth due to being identity on the charts given by the same coordinate non-vanishing loci. Proper since \(\mathbb{RP}^{n-1}\) is compact, is clearly injective, and is an immersion since in appropriate coordinates its given by the identity. Hence the homotopy defined above gives a strong deformation retract from \(\mathbb{RP}^n\) to \(\set{[x] \in \mathbb{RP} \mid x_n = 0}\) from which we get a homotopy equivalence. This concludes the annoying details and now we can proceed with the algebraic argument.

        We first want to show that for \(0 < k < n\), we have \(H^k(\mathbb{RP}^n) = 0\). Let \(q: S^n \to \mathbb{RP}^n\) be the covering map, then since \(q\) is locally invertible and \(\mathbb{RP}^n\) is compact, we have an open cover \(U_1,\hdots,U_s\) for \(\mathbb{RP}^n\), with associated maps \(q_1,\hdots,q_s\) satisfying \(qq_j = 1_{\mathbb{RP}^n}\) for each \(j\), taking a partition of unity suboordinate to the \(U_j\), we can define \(f = \sum_1^s \eta_j\cdot q_j\), with \(q\circ f = 1_{\mathbb{RP}^n}\), it follows that \(f^*q^* = 1_{\mathbb{RP}^n}^*\). Now we want to show that \([q^*]: H^*(\mathbb{RP}^n) \to H^*(S^n)\) is injective, to do so assume that \([q^*]([\omega]) = [0]\), then \(q^* \omega = d \nu\) for some \(\omega\) representing the class \([\omega]\), and some form \(\nu\), now we can use our section to find that \[\omega = f^*q^* \omega = f^* d\nu = d f^* \nu\]
        this shows that \(\omega\) is an exact form, and hence \([\omega] = 0\). This suffices to show that \([q^*]\) is injective, but then for \(0 < k < n\), we have \([q^*]: H^k(\mathbb{RP}^n) \hookrightarrow H^k(S^n) = 0\), so that \(H^k(\mathbb{RP}^n) = 0\) for \(0 < k < n\) as desired.
        
        Since \(U \cup V\) is an open cover for \(\mathbb{RP}^n\), we get the short exact sequence of chain complexes
        \begin{equation*}
            \begin{tikzcd}
                0 \arrow[r] &\Omega^*(\mathbb{RP}^n)\arrow[r] &\Omega^*(U) \oplus \Omega^*(V) \arrow[r] &\Omega^*(U\cap V) \arrow[r] &0
            \end{tikzcd}
        \end{equation*}
        Mayer Vietoris gives us a long exact sequence on cohomology, the portion of interest is for \(n > 1\)
        \begin{equation*}
            \begin{tikzcd}[arrow style=math font,cells={nodes={text height=2ex,text depth=0.75ex}}]
            \cdots & H^{n}(U) \oplus H^{n}(V) \arrow[l] \arrow[draw=none]{d}[name=Z,shape=coordinate]{} & H^{n}(\mathbb{RP}^n) \arrow[l] \\
            H^{n-1}(U \cap V) \arrow[curarrow=Z]{urr}{} & H^{n-1}(U) \oplus H^{n-1}(V) \arrow[l] & H^{n-1}(\mathbb{RP}^n) \arrow[l]
            \end{tikzcd}
        \end{equation*}
        Since cohomology is a homotopy invariant, we may substitute in the spaces above to this LES.
        \begin{equation*}
            \begin{tikzcd}[arrow style=math font,cells={nodes={text height=2ex,text depth=0.75ex}}]
            \cdots & H^{n}(\mathbb{RP}^{n-1}) \oplus H^{n}(\mathbb{R}^n) \arrow[l] \arrow[draw=none]{d}[name=Z,shape=coordinate]{} & H^{n}(\mathbb{RP}^n) \arrow[l] \\
            H^{n-1}(S^{n-1}) \arrow[curarrow=Z]{urr}{} & H^{n-1}(\mathbb{RP}^{n-1}) \oplus H^{n-1}(\mathbb{R}^n) \arrow[l] & H^{n-1}(\mathbb{RP}^n) \arrow[l]
            \end{tikzcd}
        \end{equation*}
        Now we know the cohomology for spheres, and euclidean space, \(\mathbb{RP}^{n-1}\) is \(n-1\) dimensional so that its \(n\)-th cohomology is zero and finally we already computed that \(H^{n-1}(\mathbb{RP}^n) = 0\). Applying this we get
        \begin{equation*}
            \begin{tikzcd}[arrow style=math font,cells={nodes={text height=2ex,text depth=0.75ex}}]
            \cdots & 0 \arrow[l] \arrow[draw=none]{d}[name=Z,shape=coordinate]{} & H^{n}(\mathbb{RP}^n) \arrow[l] \\
            \mathbb{R} \arrow[curarrow=Z]{urr}{} & H^{n-1}(\mathbb{RP}^{n-1}) \arrow[l] & 0 \arrow[l]
            \end{tikzcd}
        \end{equation*}
        Exactness of this sequence gives us that \(\mathbb{R} \cong H^{n-1}(\mathbb{RP}^{n-1}) \oplus H^n(\mathbb{RP}^n)\) (the splitting is guaranteed since were working with vector spaces). Now since \(\mathbb{RP}^1 \cong S^1\), which has \(H^1(S^1) \cong \mathbb{R}\), and the above formula holds for \(n > 1\), we find recursively that for \(n \geq 1\)
        \begin{align*}
            H^n(\mathbb{RP}^n) \cong \begin{cases}
                \mathbb{R} & n \text{ odd} \\
                0 & n \text{ even}
            \end{cases}
        \end{align*}
         From this and the fact that \(\mathbb{RP}^n\) is connected giving it \(0\)-th cohomology \(\mathbb{R}\), we get the cohomology ring
         \begin{align*}
            H^*(\mathbb{RP}^n) \cong \begin{cases}
                \mathbb{R}[x_n]/(x_n^2) & n\text{ odd} \\
                \mathbb{R} & n \text{ even}
            \end{cases}
         \end{align*}
         since the zero-th cohomology class is a unit with respect to wedge, and \(x_n\) represents the \(n\)-form \([\omega]\), but \(\omega\wedge \omega = 0\) since \(H^{2n}(\mathbb{RP}^n) = 0\) by dimension considerations. \qed
        
        % consider \(S^n\) embedded in \(\mathbb{R}^{n+1}\) in the standard way and let \(A = \set{(x_0,\hdots,x_n) \in S^n \mid x_n \geq 0}\), and \(B = -A\), then projection along the \(n\)-th coordinate gives \(A \cong D^n\) with \((0,\hdots,0,1) \mapsto 0\) and \(\partial A \overset{1_{S^{n-1}}}{\longrightarrow} \partial D^n\), to get a similar diffeomorphism for \(B\), compose the diffeomorphisms of projecting along the \(n\)-th coordinate then taking \(x \mapsto -x\), by construction we get the image of \(x \in A\) agrees with the image of \(-x \in B\). Now we get that \(D^n \setminus \set{0}\) is homotopic to \(S^{n-1}\) by exhibiting the strong deformation retract \(H(x,t) = (1-t)x + t\frac{x}{\norm{x}}\), now let \(\pi_A : A \to D^n\) and \(\pi_B: B \to D^n\) be the maps described above, and let \(H^A(x,t) = \pi_A^{-1}H(\pi_A(x),t)\), \(H^B(x,t) = \pi_B^{-1}H(\pi_B(x),t)\)
    \end{pb}
    \begin{pb} 
        Let \(g\) be a Riemannian metric for \(E\) over \(M\) and \(\omega\) be a representative of \([\operatorname{Th}\pi]\) from the definition of vertically compactly supported cohomology, we have that \(\pi^{-1}(K) \cap \operatorname{supp}\omega \subset E\) is compact for every \(K \subset M\) compact. Now let
        \begin{align*}
            K_0 \subset U_1 \subset K_1 \subset U_2 \subset K_2 \subset \cdots
        \end{align*}
        Be an exuastion for \(M\) (i.e. \(\bigcup_1^\infty K_j = \bigcup_1^\infty U_j = M\)) with \(K_j\) compact, and \(U_j\) open. It follows that for each \(K_j\), we have \(\sup_{(x,v) \in \pi^{-1}(K_j)\cap \operatorname{supp}\omega}g((x,v),(x,0)) = C_j <\infty\), since compact sets are bounded with respect to \(g\). Now let \(\set{\eta_j}_1^\infty\) be a partition of unity suboordinate to the \(U_j\), and let \(D_j = \inf_{x \in K_j}g(s(x),(x,0)) > 0\) since \(s\) is nonvanishing and this is a continuous function on a compact set, hence it attains its infimum. Now denote the vector bundle coordinate of \(s\) as \(s'\), i.e. if \(s(x) = (x,v)\) then \(s'(x) = v\), note that this is smooth since its just a projection of \(s\). Now we can define a homotopy
        \begin{align*}
            H: (x,t) \mapsto \left(x, ts'(x)\sum_1^\infty \eta_j(x)\frac{C_{j+1} + 1}{D_j}\right)
        \end{align*}
        This is a homotopy from \(\iota: M \hookrightarrow M \times \set{0} \subset E\) to \(F:(x,t) \mapsto \left(x, s'(x)\sum_1^\infty \eta_j(x)\frac{C_{j+1} + 1}{D_j}\right)\) and is clearly smooth since the first coordinate is identity and the second is a product of smooth functions. We want to check that \(F^*\omega = 0\), to do so it suffices to check that \(F\) maps into \((\operatorname{supp}\omega)^c\), in which case the pullback is clearly zero. By definition of \(C_j\), it will suffice to check that \(g(F(x),(x,0)) > C_j\). Since Riemannian metrics are induced by inner products, we get that for fixed \(x\), \(\abs{a}g((x,0),(x,v)) = g((x,0),(x,av))\) (note here the sum is taken to infinity but only finitely many \(\eta_j\) are nonzero, so all manipulations work out since it is in practicality a finite sum).
        \begin{align*}
            g((x,0),F(x)) = \sum_1^\infty \eta_j(x)\frac{C_{j+1}+1}{D_j}g((x,0),(x,s'(x))) \geq \sum_1^\infty \eta_j(x)\frac{C_{j+1}+1}{D_j}D_j = \sum_1^\infty \eta_j(x)(C_{j+1} + 1)
        \end{align*}
        Now note that since our exhaustion is an increasing union, we have that \(C_{j+1} \geq C_j\) for all \(j\), and moreover if \(x \in K_N \setminus K_{N-1}\) (all \(x \in M\) are in some set of this form or \(M\) compact and \(K_0 = M\), in which case just trake \(K_0\)), then \(\sup_{\pi^{-1}(x)\cap \operatorname{supp}\omega}g((x,v),(x,0)) \leq C_N\), and all \(\eta_j\) with \(j < N-1\) are zero, it follows that
        \begin{align*}
            g((x,0),F(x)) \geq \sum_{N-1}^\infty \eta_j(x)(C_{j+1} + 1) \geq \sum_{N-1}^\infty \eta_j(x)(C_N + 1) = (C_N + 1) \sum_{N-1}^\infty \eta_j(x) = C_N + 1 > C_N
        \end{align*}
        This suffices to show that for all \(x\), \(F(x) \not \in \operatorname{supp}\omega\), so that \(F^*\omega = 0\) whence \(0 = F^*[\operatorname{Th}\pi] = \iota^*[\operatorname{Th}\pi]\), the second equality of course following from \(F \simeq \iota\). \qed
       
        % This problem becomes quite difficult if our Thom class representative has disconnected support, due to in our case forms with vertically compact support being defined as having compact support on each fiber, not necessarily on the preimage \(\pi^{-1}(K)\) for each compact \(K\), to remedy this I will first prove that we can choose a Thom class representative with connected support. It will be easiest to first do this in the case of a trivial bundle over a chart, then use partitions of unity on charts, and the fact that \(E\) is orientable to construct such a Thom class in general.
        % Now working over a chart, we get a local trivialization \(E\vert_{\pi^{-1}V} \cong U \times \mathbb{R}^d\) (let \(t_1,\hdots,t_j\) be \(U\) coordinates and \(x_1,\hdots,x_d\) be \(\mathbb{R}^d\) coordinates), in this case we can simply take \(f\) to be a bump function with support \(D^d\), and rescale it so that \(\int_{\mathbb{R}^d}f dx_1\wedge\cdots\wedge dx_d = 1\), since the Thom class is uniquely determined by a closed form with \(\pi_* \omega = 1\), we find that \(\omega = f dx_1\wedge\cdots\wedge dx_d\) is the Thom class so long as \(d \omega = 0\) (to be precise we should pull \(\omega\) back along the local trivialization but since the local trivialization is a diffeomorphism this pollutes the notation and doesn't affect any properties were interested in). Closedness is easy to check since \(\frac{\partial}{\partial t_i}f = 0\) for all \(t_i\), and
        % \begin{align*}
        %     d\omega = \sum_1^k \frac{\partial}{\partial t_i}f dt_i \wedge dx_1\wedge\cdots\wedge dx_d + \sum_1^d \pd f dx_j \wedge dx_1\wedge\cdots\wedge dx_d = 0
        % \end{align*}
        % So that indeed \(\omega\) is a Thom class representative for \(E\vert_{\pi^{-1}V}\), we actually will use a bit more than this, so note that \(\omega\) has support which is connected and contains the zero section \(U \times \set{0}\). Now we will construct a Thom class representative with connected support on all of \(E\), using these local constructions. So let \(\bigcup_\alpha V_\alpha\) be a covering of \(M\) by charts, then we can take a partition of unity \(\eta_\alpha\) suboordinate to the \(V_\alpha\), and fix \(\omega_\alpha\) to be compatibly oriented forms on the \(V_\alpha\), as constructed above. To see that \(\omega = \sum_\alpha \pi^* \eta_\alpha \wedge \omega_\alpha\) is our desired Thom class representative with connected support, we first check that it is indeed a Thom class representative, in which case it suffices to show \(\pi_* \omega = 1\) and \(\omega\) is a closed form, since this determines the Thom class. To do either of these things we invoke the projection formula \(\pi_* (\pi^*\mu\wedge\nu) = \mu\wedge\pi_*\nu\) for any forms \(\mu, \nu\). Comp
        % \begin{align*}
        %     \pi_* \sum_\alpha \pi^* \eta_\alpha \wedge \omega_\alpha
        % \end{align*}
    \end{pb}
    \begin{pb}
        \textbf{(a)} The proof of the case \(\alpha' = 0\) is identical to that of \(\alpha = 0\), but we show that the rows rather than columns are linearly independent, so assume \(v_1\wedge\cdots\wedge v_p = \alpha \neq 0\). Now, it will suffice to show by induction that if \(k < p\), then we can choose \(\omega_{k+1}\), so that
        \begin{align*}
            \left\{\begin{pmatrix}
            \gen{v_1,\omega_1} \\ \vdots  \\ \gen{v_p,\omega_1}
            \end{pmatrix},\begin{pmatrix}
            \gen{v_1,\omega_2} \\ \vdots  \\ \gen{v_p,\omega_2}
            \end{pmatrix},\hdots, \begin{pmatrix}
            \gen{v_1,\omega_{k+1}} \\ \vdots  \\ \gen{v_p,\omega_{k+1}}
            \end{pmatrix}\right\} \subset \mathbb{R}^p
        \end{align*}
        are linearly independent. Since \(\alpha = v_1\wedge\cdots\wedge v_p \neq 0\) we have necessarily that the \(v_j\) are linearly independent. Now since \(k < p\), we can choose some \((x_1,\hdots,x_p) \in \mathbb{R}^p\) linearly independent from the first \(k\)-columns, then \(\omega = \omega_{k+1}\) can be constructed as follows, start with \(\omega = \frac{x_1 v_1}{\norm{v_1}^2}\) this is the base case, now assume recursively we have \(\gen{v_1,\omega} = x_1,\hdots, \gen{v_j,\omega} = x_j\), then we can take \(u\) to be the projection of \(v_{j+1}\) to \(\text{span}\set{v_1,\hdots,v_j}^\perp\), this is nonzero since \(v_{j+1} \not \in \text{span}\set{v_1,\hdots,v_j}\). Then we have \(\gen{v_{j+1},u} = a \neq 0\) finally denote \(\gen{v_{j+1}, \omega} = b\), and now take \(\omega' = \omega + \frac{x_{j+1} - b}{a}u\), then since \(u\) is orthogonal to \(v_1,\hdots,v_j\), we still get \(\gen{v_i,\omega'} = x_i\) for \(i = 1,\hdots, j\), but now we also get that
        \begin{align*}
            \gen{v_{j+1},\omega'} = \gen{v_{j+1},\omega} + \frac{x_{j+1} - b}{a}\gen{v_{j+1},u} = b + \frac{x_{j+1} - b}{a}a = x_{j+1}
        \end{align*}
        Continuing this process we get the desired \(\omega_{k+1}\), since this holds for any \(k < p\), we can always construct some \(\omega_1\wedge\cdots\wedge \omega_p\) with the property that the columns of \((\gen{v_i,\omega_j})_{1 \leq i,j \leq p}\) are linearly independent, and hence \(\gen{\alpha,\omega_1\wedge\cdots\wedge \omega_p}_p = \det (\gen{v_i,\omega_j})_{1 \leq i,j \leq p} \neq 0\). \qed

        \textbf{(b)} Consider two positively oriented orthonormal bases \(e_1,\hdots,e_k\) and \(d_1,\hdots,d_k\). Let \(T\) be the linear map defined by \(T(e_i) = d_i\), and extending linearly, since both bases are positively oriented we get \(\det T > 0\), moreover we have \((T^\text{T}T)_{ij} = \gen{d_i,d_j} = \delta_{ij}\), so that \(T^\text{T}T = 1_V\) is orthogonal, since \(\det T^\text{T} = \det T\), this relation gives us \((\det T)^2 = 1\), so \(\det T = \pm 1\), but since we have established \(\det T > 0\), we get to conclude that \(\det T = 1\). Now we are done since
        \begin{align*}
            d_1\wedge\cdots\wedge d_k = T(e_1)\wedge \cdots \wedge T(e_k) = (\det T)(e_1\wedge\cdots\wedge e_k) = e_1\wedge\cdots\wedge e_k
        \end{align*} \qed

        \textbf{(c)} We first consider an element of the form \(\beta = e_{i_1}\wedge\cdots\wedge e_{i_{k-p}}\) with \(i_1 < i_2 < \cdots < i_{k-p}\), now we can denote \(\set{j_1,\hdots,j_p} = \set{1,\hdots,k} \setminus \set{i_1,\hdots,i_{k-p}}\) with \(j_1 < \cdots < j_p\). It follows that \(e_{j_1}\wedge\cdots\wedge e_{j_p} \wedge \beta = (-1)^\ell \omega\) for some \(\ell\). I claim that \(\star \beta = (-1)^\ell e_{j_1}\wedge\cdots\wedge e_{j_p}\) satisfies \(\lambda_\beta(\alpha) = \gen{\alpha,\star\beta}_p\). We first check this for \(\alpha\) of the form \(e_{r_1}\wedge\cdots\wedge e_{r_p}\), since if it holds for elements of this form we get general elements of \(\Lambda^p(V) = \sum a_i \alpha_i\) for \(a_i\) of this form, so that since \(\lambda_\beta\) is linear we get
        \begin{align*}
            \lambda_\beta(\sum a_i \alpha_i) = \sum a_i \lambda_\beta(\alpha_i) = \sum a_i \gen{\alpha_i,\star\beta} = \gen{\sum a_i \alpha_i, \star \beta}
        \end{align*}
        so it suffices to check in this simplified case. Now if \(\set{r_1,\hdots,r_p} \cap \set{i_1,\hdots,i_{k-p}} = \set{i_z} \neq \emptyset\), then we get \(\alpha \wedge \beta = 0\), hence \(\lambda_\beta(\alpha) = 0\), as well as the matrix with determinant \(\gen{\alpha,\star\beta}_p\) having a row corresponding to \((\gen{e_{i_z},e_{j_1}},\hdots,\gen{e_{i_z},e_{j_p}}) = (0,\hdots,0)\), so that \(\gen{\alpha,\star\beta}_p = 0\), now in the case that \(\set{r_1,\hdots,r_p} \cap \set{i_1,\hdots,i_{k-p}} = \emptyset\), we get that \(r_1,\hdots,r_p = \sigma(j_1),\hdots,\sigma(j_p)\) for \(\sigma \in S_p\), then \(e_{r_1}\wedge\cdots\wedge e_{r_p} = \text{sgn}(\sigma)e_{j_1}\wedge\cdots\wedge e_{j_p}\), so that \(\alpha \wedge \beta = \text{sgn}(\sigma)(-1)^\ell \omega\), and \(\gen{\alpha\wedge\beta,\omega}_k = \text{sgn}(\sigma)(-1)^\ell\), moreover \(\gen{\alpha,\star \beta} = (-1)^\ell \det P_\sigma\) where \(P_\sigma\) denotes the permutation matrix taking \(j_1 \mapsto \sigma(j_1)\), of course this is also equal to \((-1)^\ell \text{sgn}(\sigma)\), so we have provided existence of \(\star\beta\) for \(\beta\) of the form \(e_{i_1}\wedge\cdots\wedge e_{i_{k-p}}\), from this we can establish existence for all \(\beta\), since any \(\beta \in \Lambda^{k-p}(V)\) can be written as \(\sum a_i \beta_i\) for \(\beta_i\) of this form, this allows us to define \(\star \beta = \sum a_i \star \beta_i\) then for any \(\alpha \in \Lambda^p(V)\) we get
        \begin{align*}
            \lambda_\beta(\alpha) &= \gen{\alpha \wedge \sum a_i \beta_i, \omega}_k = \gen{\sum a_i \alpha \wedge \beta_i, \omega}_k = \sum a_i \gen{\alpha \wedge \beta_i, \omega}_k \\ &= \sum a_i \gen{\alpha, \star \beta_i}_p = \gen{\alpha, \sum a_i \star \beta_i}_p = \gen{\alpha, \star \beta}_p
        \end{align*}
        Which suffices to prove existence for any \(\beta \in \Lambda^{k-p}(V)\). Now we need to check uniqueness Suppose \(\star \beta' = \star \beta\), then \(\alpha \mapsto \gen{\alpha \wedge (\beta - \beta'),\omega}_k = 0\) for all \(\alpha\). Suppose now that \(\beta \neq \beta'\), we can write \(\beta = \sum a_i \beta_i\), and \(\beta' = \sum b_i \beta'_i\) where \(\beta_i, \beta'_i\) are of the form \(e_{i_1}\wedge\cdots\wedge e_{i_{k-p}}\) for \(i_1 < \cdots < k-p\), it follows that the multiplicity of one of these summands must differ between \(\beta\) and \(\beta'\), otherwise the two will be equal. So assume without loss of generality that \(\beta_1 = \beta'_1\), but \(a_1 \neq b_1\), moreover since one of them must be nonzero we can assume \(a_1 \neq 0\). Now denote \(\beta_1 = e_{i_i}\wedge\cdots\wedge e_{i_{k-p}}\), and once again define \(\set{j_1,\hdots,j_p} = \set{1,\hdots,k} \setminus \set{i_1,\hdots,i_{k-p}}\) with \(j_1 < \cdots < j_p\), it follows that for \(\alpha = e_{j_1}\wedge\cdots\wedge e_{j_p}\) we have \(\alpha \wedge \beta_\ell = 0\) for any \(\ell \neq 1\), and same for \(\beta'_\ell\), since some \(j_z\) must appear in the wedge terms of \(\beta_\ell\) (or respectively \(\beta_\ell'\)) by virtue of \(\beta_\ell\) (resp. \(\beta'_\ell\)) not being identical to \(\beta_1 = \beta'_1\). Moreover, we get \(\alpha \wedge \beta_1 = (-1)^r \omega\) for some \(r\). It follows that
        \begin{align*}
            \gen{\alpha\wedge (\beta - \beta'), \omega}_k &= \gen{\alpha \wedge \beta, \omega}_k - \gen{\alpha \wedge \beta', \omega}_k = \sum a_i \gen{\alpha \wedge \beta_i, \omega}_k - \sum b_i \gen{\alpha \wedge \beta',\omega}_k \\
            &= a_1 \gen{\alpha \wedge \beta_1, \omega}_k - b_1 \gen{\alpha \wedge \beta'_1, \omega}_k = (a_1 - b_1)\gen{\alpha \wedge \beta_1, \omega}_k \\ 
            &= (-1)^r(a_1 - b_1) \gen{\omega, \omega}_k = (-1)^r(a_1 - b_1) \neq 0
        \end{align*}
        Which contradicts \(\star \beta = \star \beta'\), so this suffices to show uniqueness.

        Now that we have existence and uniqueness, linearity is quite easy. Let \(\beta, \gamma \in \Lambda^{k-p}(V)\), then for any \(\alpha\) we have
        \begin{align*}
            \gen{\alpha,\star (a \beta + b\gamma)}_p &= \gen{\alpha\wedge (a\beta + b\gamma),\omega}_k = \gen{a \alpha \wedge \beta + b \alpha \wedge \gamma,\omega}_k = a\gen{\alpha \wedge \beta, \omega}_k + b\gen{\alpha \wedge \gamma, \omega}_k \\
            &= a \gen{\alpha, \star \beta}_p + b \gen{\alpha, \star \gamma}_p = \gen{\alpha, a\star \beta + b\star\gamma}
        \end{align*}
        Uniqueness then tells us that \(\star (a \beta + b\gamma) = a\star \beta + b\star\gamma\). \qed

        \textbf{(d)} To extend the \(\star\) operator to forms on \(M\), we would like to apply \(\star\) fiberwise. To make sense of this, we first require \(\gen{-,-}_p\) for each \(1\leq p \leq k\) on \(\Lambda^p T^*M \times \Lambda^p T^*M\). Since \(\gen{-,-}_p\) is defined using a metric on \(V\), we only need a metric on \(T^*M\), this is of course given by a Riemannian metric, since in order for \(\star\beta\) to be a smooth form, we will require the metric to be smoothly varying. Now in order to define \(\star\beta\) fiberwise as satisfying
        \begin{align*}
            \gen{\alpha\wedge\beta,\omega}_k = \gen{\alpha,\star\beta}_p
        \end{align*}
        we need a choice for \(\omega \in \Lambda^k T^*M\), since \(\omega\) must be the wedge of \(k\)-basis elements in each fiber, we require a nonvanishing element of \(\omega \in \Lambda^k T^*M\), this is of course an orientation (by identifying the cotangent bundle with the tangent bundle), so that in order to have this \(M\) must be orientable. Now if \(M\) is orientable, we can choose our orthonormal basis on each fiber (which will be smooth since our Riemannian metric is) to agree with our orientation, in order to get our element \(\omega\). Finally, since we have \(\omega\), restricting to the desired form in each fiber, our fiberwise construction of \(\star\beta\) is smooth using the orthonormal frame, so that the construction carries over fiberwise to manifolds. \qed

        \textbf{(e)} We can use that \(\star\) agrees fiberwise with the original fiberwise definition, and in this case identify \(dx,dy,dz \leftrightarrow e_1,e_2,e_3\). This in particular means our proof from part (c) shows that if \(h \in C^{\infty}(M, \mathbb{R})\), we get
        \begin{align*}
            \star h dx &= h \star dx = h dy \wedge dz \\
            \star h dy &= h \star dy = -h dx \wedge dz \\
            \star h dz &= h \star dz = h dx \wedge dy
        \end{align*}
        Now applying this to \(df\), we get
        \begin{align*}
            d\star df &= d\star \px fdx + \py fdy + \pz fdz = d(\px f dy \wedge dz - \py f dx\wedge dz + \pz f dx \wedge dy) \\
            &= \ppx f dx \wedge dy \wedge dz  - \ppy fdy\wedge dx \wedge dz + \ppz fdz\wedge dx\wedge dy \\ &= (\ppx f + \ppy f + \ppz f)dx \wedge dy \wedge dz
        \end{align*} \qed
    \end{pb}
\end{document}