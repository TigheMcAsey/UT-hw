\documentclass[10.5pt]{article}
\usepackage{amsmath, amsfonts, amssymb,amsthm}
\usepackage[includeheadfoot]{geometry} % For page dimensions
\usepackage{fancyhdr}
\usepackage{enumerate} % For custom lists
\usepackage{tikz-cd}
\usepackage{graphicx}

\fancyhf{}
\lhead{MAT1300 hw5}
\rhead{Tighe McAsey - 1008309420}
\pagestyle{fancy}

% Page dimensions
\geometry{a4paper, margin=1in}

\theoremstyle{definition}
\newtheorem{pb}{}
\usepackage{tikz-cd, stackengine}

% Commands:

\newcommand{\set}[1]{\{#1\}}
\newcommand{\gen}[1]{\langle#1\rangle}
\newcommand{\abs}[1]{\lvert#1\rvert}
\newcommand{\norm}[1]{\lvert\lvert#1\rvert\rvert}
\newcommand{\tand}{\text{ and }}
\newcommand{\tor}{\text{ or }}
\newcommand{\pd}{\frac{\partial}{\partial x_j}}
\setcounter{MaxMatrixCols}{20}

\begin{document}
    \begin{pb}
        We start by computing the Jacobian of the map \(F: x \mapsto \frac{x}{\norm{x}}\)
        \begin{align*}
            \pd \frac{x_i}{\sqrt{\sum_1^{n+1} x_k^2}} = \frac{\delta_{ij}\norm{x} - x_ix_j\norm{x}^{-1}}{\norm{x}^2} = \frac{\delta_{ij}}{\norm{x}} - \frac{x_ix_j}{\norm{x}^3}
        \end{align*}
        So the Jacobian looks like
        \begin{align*}
            \frac{1}{\norm{x}}1 - \frac{1}{\norm{x}^3}xx^T
        \end{align*}

        Now beginning the actual proof, let \(\set{f_1,\hdots,f_{n-m}}\) be a basis for \((\text{Im}\,A)^\perp\), then define \(T: \mathbb{R}^{n-m+1} \to \mathbb{R}^{n+1}\) via \(e_i \mapsto f_i\) when \(1 \leq i \leq n-m\) and \(e_{n-m+1} \mapsto Ae_1\), then \(\overline{T}: \mathbb{RP}^{n-m}\to \mathbb{RP}^n\) is an embedding, so we can refer to its image as the submanifold \(X \subset \mathbb{RP}^n\). Now we get that \(\text{Im}(\overline{A})\cap X = q(\text{Im}\,A \cap \text{Im}\,T)\), where \(q: \mathbb{R}^{n+1}\setminus\set{0} \to \mathbb{RP}^n\) is the quotient. By construction this intersection is \([Ae_1]\), so that once we verify \(\overline{A}\pitchfork_{[Ae_1]}X\) we will get that \(I_2(\overline{A},X) = 1\). Now checking transversality, we will use the following maps, where the \(\pi\) maps are the indiced quotients under the action by the discrete group. Note that in particular the \(\pi\) maps are submersions of manifolds of equal dimension and hence local diffeomorphisms by the inverse function theorem.
        \begin{align*}
            &\widehat{A}: S^m \to S^n &\widehat{T}: S^{m-n} \to S^n\\
            &v \mapsto \frac{Av}{\norm{Av}} &v \mapsto \frac{Tv}{\norm{Tv}} \\
            &\pi_m: S^m \to \mathbb{RP}^m &\pi_{n-m}: S^{n-m} \to \mathbb{RP}^{n-m} \\
            &\pi: S^n \to \mathbb{RP}^n
        \end{align*}
        Moreover, the following diagrams commute by definition of \(\widehat{A},\widehat{T}\)
        \begin{equation*}
            \begin{tikzcd}
            S^m \arrow[r,"{\widehat{A}}"] \arrow[d,"{\pi_m}"] &S^n \arrow[d,"{\pi}"] & S^{n-m} \arrow[r,"{\widehat{T}}"] \arrow[d,"{\pi_{n-m}}"] &S^n \arrow[d,"{\pi}"]\\
            \mathbb{RP}^m \arrow[r,"{\overline{A}}"] &\mathbb{RP}^n &\mathbb{RP}^{n-m} \arrow[r,"\overline{T}"] &\mathbb{RP}^n
            \end{tikzcd}
        \end{equation*}
        We will verify later that \(\text{Im}\,d_{e_1}\widehat{A} + \text{Im}\,d_{e_{m-n+1}}\widehat{T} = T_{A(e_1)}S^n\), but assuming it for now we find that (using repeatedly the submersion properties of the projections)
        \begin{align*}
            T_{\overline{A}e_1}\mathbb{RP}^n &= d_{\widehat{A}e_1}\pi(\text{Im}\,d_{e_1}\widehat{A} + \text{Im}\,d_{e_{m-n+1}}\widehat{T}) \\
            &= \text{Im}\,d_{e_1}(\pi\circ\widehat{A}) + \text{Im}\,d_{e_{n-m+1}}(\pi\circ\widehat{T}) \\
            &= \text{Im}\,d_{e_1}(\overline{A}\circ\pi_m) + \text{Im}\,d_{e_{n-m+1}}(\overline{T}\circ\pi_{n-m}) \\
            &= \text{Im}\,(d_{[e_1]}\overline{A}) + \text{Im}\,(d_{[e_{n-m+1}]}\overline{T}) \\
            &= \text{Im}\,(d_{[e_1]}\overline{A}) + T_{[Ae_1]}X
        \end{align*}
        This verifies that indeed \(\overline{A}\pitchfork_{[Ae_1]}X\), now to complete the proof, note that we have some \([p] \in \mathbb{RP}^n \setminus X\), since \(\mathbb{RP}^n\) is connected (therefore path connected), any constant map \(\mathbb{RP}^m \to \mathbb{RP}^n\) is homotopic to the map \(c:\mathbb{RP}^m \to [p]\), where \(I_2(c,X) = 0\) trivially, since intersection number is a homotopy invarient this completes the proof.

        \textbf{(Proof of \(\mathbf{d_{e_1}\widehat{A} + d_{e_{m-n+1}}\widehat{T} = T_{A(e_1)}S^n}\)):} To show this, we will compute the derivatives as maps of \(\mathbb{R}^k\setminus\set{0} \to \mathbb{R}^{n+1}\), then use the characterization of the tangent space \(T_pS^k = p^\perp \cap T_p \mathbb{R}^{k+1}\). To compute the derivative note that the maps are of the form \(\widehat{A} = F\circ A \tand \widehat{T} = F \circ T\), where we computed the derivative of \(F\) prior to tackling the problem, by the chain rule we have
        \begin{align*}
            d_{e_1}\widehat{A} &= \left(\frac{1}{\norm{Ae_1}}1 - \frac{1}{\norm{Ae_1}^3}(Ae_1)\cdot(Ae_1)^T\right)d_{e_1}A\\
            d_{e_{n-m+1}}\widehat{T} &= \left(\frac{1}{\norm{Ae_1}}1 - \frac{1}{\norm{Ae_1}^3}(Ae_1)\cdot(Ae_1)^T\right)d_{e_{n+m-1}}T
        \end{align*}
        restricting to the orthogonal compliment of \(Ae_1\), \(\frac{1}{\norm{Ae_1}^3}(Ae_1)\cdot(Ae_1)^T \equiv 0\), so that
        \begin{align*}
            d_{e_1}\widehat{A} \equiv \frac{1}{\norm{Ae_1}}d_{e_1}A \tand d_{e_{n-m+1}}\widehat{T} \equiv \frac{1}{\norm{Ae_1}}d_{e_{n-m+1}}T
        \end{align*}
        the derivative should also have restricted domain since these are maps of spheres, restricting the domain of \(d_{e_1}\widehat{A}\) to \(e_1^\perp\) and \(d_{e_{n-m+1}}\widehat{T}\) to \(e_{n-m+1}^\perp\) and taking \(\rho: \mathbb{R}^{n+1} \to (Ae_1)^\perp\) to be the orthogonal projection we find the images of either differential have respective bases
        \begin{align*}
            \set{\rho(Ae_2),\hdots,\rho(Ae_{m+1})} \tand \set{\rho(Te_1),\hdots,\rho(Te_{n-m})}
        \end{align*}
        By definition of \(T\), and injectivity of both \(A\) and \(T\) (which have \(Ae_1\) in their image), this collection of \(n\) vectors forms a basis for \(T_{Ae_1}S^n\), this is easiest to see by writing it as
        \begin{align*}
            \rho(\gen{Ae_2,\hdots,Ae_{m+1},f_1,\hdots,f_{n-m}})
        \end{align*}
        where \(\rho\) has no kernel on this subspace, and this space has dimension \(n\) by definition of the \(f_i\) and injectivity of \(A\). \qed
        
        % Consider the translated vectorspace \(W \subset \mathbb{R}^{n+1} = (\text{Im}\,A)^\perp + \frac{A(e_1)}{2\norm{A e_1}}\), then this defines a submanidold \(X\) of \(\mathbb{RP}^n\) after taking the quotient (check!). Moreover, \(\overline{A}(\mathbb{RP}^m) \cap X = [Ae_1]\), and the intersection is transverse, \(A \pitchfork_{[Ae_1]} X\) (check!). This gives that \(I_2(A,X) = 1\), we will be done once we shot the straightforward result that a constant map has zero intersection number with \(X\). Since \(\mathbb{RP}^n\) is connected and therefore path connected, and \(X \subsetneq \mathbb{RP}^n\), we can take \([z] \in X^c\) so that any constant map is homotopic to \(c:\mathbb{RP}^m \to [z]\) which has \([z] \cap X = \emptyset\) so that \(c \pitchfork X\) trivially with \(I_2(c,X) = 0\).

        % (check! 1):  We can realize \(X\) as a submanifold via an embedding, namely take \(W' = W\cap S^n\), then the quotient map \(q: S^n \to \mathbb{RP}^n\) is an embedding when restricted to \(W'\), this follows from \(q\) being a submersion between spaces of equal dimension and hence also an immersion, moreover \(q\vert_{W'}\) is injective since \(0 \not \in W'\) implies that only one representative of each equivalence class in \(\mathbb{RP}^n\) lies in \(W'\) (if \(x,-x \in W'\) then \(x = v + Ae_1\), so that \(-x + v = -Ae_1 \in W' \implies -2Ae_1 \in (\text{Im}A)^\perp\) a contradiction), to conclude that \(q\vert_{W'}\) is an immersion it will suffice to check that \(W'\) is closed (hence compact by being a subset of the sphere), to see this note that . We can easily check that the image of \(\overline{A}\) intersected with \(X\) is \([Ae_1]\), since \(W'\cap \text{Im}\,A = Ae_1\), so there is also only one point of intersection after taking the quotient.

        % (check! 2):
    \end{pb}
    \begin{pb}
        \textbf{(a)} From the Tubular neighborhood theorem we have a diffeomorphism \(\phi: NM \to W\) where \(W\) is an open neighborhood of \(M\) in \(N\), then we take take \(s: M \to NM\), where \(s(x) = (x,\tilde{s}(x))\), We define the isotopy as follows:
        \begin{align*}
            e(t,x) = \phi(x,t\tilde{s}(x))
        \end{align*}
        Clearly \(e\) is smooth, disjointness of \(e_1(M)\) from \(M = e_0(M)\) follows from injectivity of \(\phi\), and their disjointness in the normal bundle. Now we only need check that \(e\) is an isotopy, but it is clear that for each \(t\), we have \(x \mapsto (x,t\tilde{s}(x))\) is an embedding, since it is a smooth section of the normal bundle (so has smooth inverse \(\pi_N\vert_{e_t(M)}\), i.e. is diffeomorphic to its image), then post composing with the diffeomorphism \(\phi\) giving \(e_t\) of course still gives an embedding.

        \textbf{(b)} From the classification of \(1\) dimensional manifolds, \(M\) must be \(S^1\) Here I will identify points on \(S^1\) with \([-\frac12,\frac12)\) in the obvious way. Then since \(M\) has codimension \(1\), \(NM\) is a line bundle for \(M\), so by the classification of line bundles on \(S^1\) must either be the trivial bundle or the M\"obius bundle (note that non-vanishing smooth sections are preserved by bundle isomorphisms, so we can work concretely in these cases), in the case of the trivial bundle we are done by taking the \(M \times \set{1}\) section. Now I will show the case of the M\"obius bundle cannot happen. Take the section family of sections indexed by \(\epsilon \in (0,1]\), \(s_\epsilon(x) = (x,\epsilon\sin(\pi x))\), which is a smooth section (and hence an embedding since it has smooth inverse on its image given by the bundle projection), moreover we have \(M \cap s_\epsilon(M) = (0,0)\), to see that \(s_\epsilon \pitchfork M\), simply notice that taking \(\gamma(t) = t\) has \(\frac{d}{dt}\vert_{t=0}s_\epsilon\circ\gamma = \begin{pmatrix} 1 \\ \epsilon\pi \end{pmatrix}\), and since \[\begin{pmatrix} 1 \\ \epsilon\pi \end{pmatrix} \not \in T_{(0,0)}M\times\set{0} = \text{span}\begin{pmatrix}
            1 \\0
        \end{pmatrix}\]
        The intersection is transverse and thus \(I_2(s,M) = 1\). To get a contradiction from this, let \(g\) be a riemmannian metric on \(NM\) and \(\phi: NM \overset{\cong}{\longrightarrow} W \supset M\) be the diffeomorphism given by the tubular neighborhood theorem. Then since \(M\) and \(e_1(M)\) are compact and disjoint we have \(d(M,e_1(M)) = \delta > 0\), Now take \(\phi^{-1}(B_{\delta/2}(M) \cap W)\) where \(B_{\delta/2}(M) := \bigcup_{x \in M} B_{\delta/2}(x)\), this is an open neighborhood of \(M\) in the normal bundle, so for each \(x \in M\), there is some \(U_x \supset \set{x}\), and some \(\epsilon_x > 0\) with \(U_x \times (-\epsilon_x,\epsilon_x) \subset \phi^{-1}(B_{\delta/2}(M) \cap W)\), by compactness there is a finite subcover, and hence some \(\epsilon > 0\) so that \(B_\epsilon(M) \subset NM\) (where we define \(B_\epsilon(M) := \bigcup_{x \in M}\set{x}\times(-\epsilon,\epsilon)\)) has the property that \(\phi(B_\epsilon(M)) \subset B_{\delta/2}(M) \subset N\), and therefore \(I_2(s_{\epsilon/\pi},e_1(M)) = 0\), since they are disjoint. This of course contradicts intersection number being isotopy invarient. \qed
        
        
        % This proof is a bit more involved than previous homework problems, so I will provide an outline before diving in, here we want to first Whitney embed \(N \hookrightarrow \mathbb{R}^r\) so we can use metrics. The main idea here is to reconstruct the normal bundle of \(M\) by taking the normal bundle of embeddings of \(S^1\) into \(M\), and then restricting their normal bundles to \(NM\). Once we show that this reconstruction is possible, we will use the classification of line bundles on \(S^1\), if one of the line bundles is a M\"obius bundle, then a section \(s\) will give a transverse intersection with \(M\), in this case we can take the section to be closer to \(M\) than \(d(e_1(M),M)\), so that \(I_2(e_1,s) = 0 \neq 1 = I_2(e_0,s)\), which will reduce the problem to all of the line bundles being trivial, in which case we can show that we can take sections on line bundles from the embeddings which are compatible and give a nonvanishing global section on \(M\).

        % As stated in the preamble, we can consider \(N \subset \mathbb{R}^r\) by taking a Whitney embedding, this also allows us to deal with normal bundles less formally. Now consider the collection of embeddings \(\set{\gamma_\alpha}_{\alpha \in I}\) where each \(\gamma_\alpha: S^1 \hookrightarrow M\), then we may consider \(NM\vert_{\gamma_\alpha(S^1)} \subset N\gamma_\alpha(S^1) \subset TN\vert_{\gamma_{\alpha}(S^1)}\), so that \(NM\vert_{\gamma_\alpha(S^1)}\) is a line bundle over \(S^1\), we first assume for the sake of contradiction that there exists some \(\alpha\), with \(NM\vert_{\gamma_\alpha(S^1)} \cong \text{Mobius}(S^1)\), we can take the section on the mobius bundle \(s(x) = (x,\sin(\pi x))\), which is a smooth section with intersection number \(1\),
    \end{pb}
    \begin{pb}
        Suppose \(n > 2(r+1)\), and \(e: S^r \hookrightarrow \mathbb{R}^n\), we define a sort of smooth mapping cone \(E: D^{r+1} \to \mathbb{R}^n \times \mathbb{R}\) by writing points in \(D^{r+1}\) as \((s,t)\) for \(s \in S^r\) and \(t \in [0,1]\), and taking the map (unfortunately to get \(S^r\) along the zero section we have to flip our convention for radius)
        \begin{align*}
            E:(s,t) \mapsto \left(e(s)\cos\frac{\pi t}{2},\sin\frac{\pi t}{2}\right)
        \end{align*}
        This is well defined since when \(t = 1\) the expression is not dependent on \(s\). It is also immediate from construction that \(E\) is an embedding. Now we want to modify the proof of the strong Whitney embedding theorem to turn this into a map into \(\mathbb{R}^n\) respecting the original embedding, let \(\iota: \mathbb{R}^n \to \mathbb{R}^n\times \mathbb{R}\) be the inclusion into \(\mathbb{R}^n \times \set{0}\). The main labour involved in this proof is to show that for some \(x \in S^n\), denoting \(\pi_x\) to be the projection onto \(x^\perp\) the following properties hold
        \begin{enumerate}
            \item \(\pi_x \circ E\) is injective
            \item \(\pi_x \circ E\) is an immersion
            \item \(\pi_x \circ \iota\) is a diffeomorphism \(\mathbb{R}^n \to \mathbb{R}^n\)
        \end{enumerate}
        For now we assume such an \(x\) exists, then \(\pi_x \circ E\) is an embedding \(D^{r+1} \to \mathbb{R}^n\) (properness follows from the \(r+1\) disc being compact). This implies that \[(\pi_x \circ \iota)^{-1}\circ(\pi_x\circ E): D^{r+1} \to \mathbb{R}^n\] is also an embedding, in fact it is the desired embedding since for \(s \in S^r\) we have
        \begin{align*}
            \pi_x\circ E(s) = \pi_x(e(s),0) = \pi_x\circ\iota\circ e(s)
        \end{align*}
        so that \((\pi_x\circ \iota)^{-1}\pi_x\circ E(s) = e(s)\).

        Now we need to verify properties (1-3), it will be easiest to show using Sard's theorem that values of \(x\) failing each of these individual properties have measure zero, so that some \(x\) satisfying all three must exist. Letting \(x \in S^n\) note that \(\pi_x\circ \iota\) is a compositon of linear maps, hence linear, by taking the dimensions into account, it suffices to show that it has zero kernel to be an isomorphism (then since the inverse is linear it will be a diffeomorphism), having zero kernel is equivalent to \(x\) not being in \(\iota(\mathbb{R}^n)\), but \(\iota(\mathbb{R}^n)\) is measure zero, so it remains to check properties one and two are fulfilled on sets with compliment measure zero. Property \(2\) follows directly from the proof of Whitney embedding theorem, taking \begin{align*}
            f^{\text{tang}} : &TD^{r+1} \setminus D^{r+1} \times \set{0} \to S^n \\
            &v \mapsto \frac{dE(v)}{\norm{dE(v)}}
        \end{align*}
        Now if \(x \not \in \text{Im}\,f^{\text{tang}}\), then \(d(\pi_x\circ E)\) is injective, hence \(\pi_x \circ E\) is an immersion, since \(\dim TD^{r+1} = 2r+2 < n\), a point \(x\) is a regular value for \(f^{\text{tang}}\) exactly when it is disjoint from the image, so \(\text{Im}\,f^\text{\text{tang}}\) has measure zero by Sard's theorem, as desired. Finally, we show condition 1, due to the constraints of manifolds with boundary (i.e. not being able to take products) we will actually need to use multiple maps to do so, define the following maps here we again identify points in \(D^{r+1}\) with their polar coordinates \((s,t)\), where \(S^r = \set{(s,0) \mid s \in S^r}\)
        \begin{align*}
            f^\text{inj}_1: &(D^{r+1} \times S^r) \setminus \set{(s,0),s} \to S^n  \\
            &(x,y,t) \mapsto \frac{E(x) - E(y,0)}{\norm{E(x) - E(y,0)}} \\
            f^{\text{inj}}_2: &(D^{r+1} \times S^r \times (0,1)) \setminus \set{(s,t),(s,t) \mid s \in S^r, t\in (0,1)} \to S^n  \\
            &(x,y) \mapsto \frac{E(x) - E(y,t)}{\norm{E(x) - E(y,t)}}\\
            f^\text{inj}_3: &D^{r+1} \setminus \set{1} \to S^n  \\
            &x \mapsto \frac{E(x) - E(1)}{\norm{E(x) - E(1)}}
        \end{align*}
        Now, by construction, if \(x \in \text{Im}(f_1^\text{inj})^c \cap \text{Im}(f_2^\text{inj})^c \cap \text{Im}(f_3^\text{inj})^c\), then \(\pi_x\circ E\) is injective. The domains of \(f_i^\text{inj}\) have dimensions \(2r+1,2r+2 \tand r+1\) which are all less than \(n\), so the regular values of each of these maps are the points disjoint from their image. Since the critical values of each of these maps have measure zero, the union of their critical values has measure zero as desired, thus property 1 holds everywhere apart from a set of measure zero and we are done. \qed
    \end{pb}
\end{document}