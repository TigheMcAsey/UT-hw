\documentclass[10.5pt]{article}
\usepackage{amsmath, amsfonts, amssymb,amsthm}
\usepackage[includeheadfoot]{geometry} % For page dimensions
\usepackage{fancyhdr}
\usepackage{enumerate} % For custom lists
\usepackage{tikz-cd}
\usepackage{graphicx}

\fancyhf{}
\lhead{MAT1300 hw6}
\rhead{Tighe McAsey - 1008309420}
\pagestyle{fancy}

% Page dimensions
\geometry{a4paper, margin=1in}

\theoremstyle{definition}
\newtheorem{pb}{}
\usepackage{tikz-cd, stackengine}

% Commands:

\newcommand{\set}[1]{\{#1\}}
\newcommand{\gen}[1]{\langle#1\rangle}
\newcommand{\abs}[1]{\lvert#1\rvert}
\newcommand{\norm}[1]{\lvert\lvert#1\rvert\rvert}
\newcommand{\tand}{\text{ and }}
\newcommand{\tor}{\text{ or }}
\newcommand{\pd}{\frac{\partial}{\partial x_j}}
\setcounter{MaxMatrixCols}{20}

\begin{document}
    \begin{pb}
        
    \end{pb}
    \begin{pb}
        
    \end{pb}
    \begin{pb}
        \textbf{Lemma.} If \(G\) a (finite) discrete group acts on an orientable manifold \(M\) such that the action is smooth, free and proper, such that for each \(g \in G\) and \(p \in M\), \(\det (d_p g) > 0\) then there is an induced orientation on \(M/G\).
        \begin{proof}
            For convenience, take the section \(s: M \to \Lambda^n TM\) so that \(s > 0\). Let \(q: M \to M/G\) be the quotient map induced by the group action, and let \(\set{V_\alpha}_{\alpha \in \mathcal{A}}\) be an open cover for \(M\) with \(q^{-1}(V_\alpha) = \bigsqcup_1^r U_\alpha^i\) and \(q\vert_{U_\alpha^i}: U_\alpha^i \overset{\cong}{\longrightarrow} V_\alpha\). Now let \(\set{\eta_\alpha}_\mathcal{A}\) be a partition of unity subordinate to the \(V_\alpha\)
        \end{proof}

        Denote \(j:S^d \to S^d\) as the antipodal map. When \(d\) is odd, we have an isotopy \(1_{S^d} \sim j\) via \(H((z_1,\hdots,z_{\frac{d+1}{2}}),t) = (e^{i\pi t}z_1,\hdots,e^{i\pi t}z_{\frac{d+1}{2}})\), 
    \end{pb}
\end{document}